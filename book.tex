%%%%%%%%%%%%%%%%%%%% book.tex %%%%%%%%%%%%%%%%%%%%%%%%%%%%%
%
% sample root file for the chapters of your "monograph."
%
% Use this file as a template for your input.
%
%%%%%%%%%%%%%%%% Springer-Verlag %%%%%%%%%%%%%%%%%%%%%%%%%%


% RECOMMENDED %%%%%%%%%%%%%%%%%%%%%%%%%%%%%%%%%%%%%%%%%%%%%%%%%%%
\documentclass[graybox,envcountchap]{svmono}

% choose options for [] as required from the list
% in the Reference Guide

%\usepackage{mathptmx}
%\usepackage{helvet}
%\usepackage{courier}
%
\usepackage[utf8]{inputenc}
\usepackage{type1cm}  
\usepackage{mathtools}  

\usepackage{makeidx}         % allows index generation
\usepackage{graphicx}        % standard LaTeX graphics tool
                             % when including figure files
\usepackage{multicol}        % used for the two-column index
\usepackage[bottom]{footmisc}% places footnotes at page bottom

\usepackage{newtxtext}       % 
\usepackage[varvw]{newtxmath}       % selects Times Roman as basic font

\usepackage{xspace}
\usepackage{xcolor}
\usepackage{xparse}
\usepackage{enumitem}

\usepackage{amsmath}

\bibliographystyle{spmpsci}


\renewcommand{\epsilon}{\varepsilon}
\renewcommand{\phi}{\varphi}

\newcommand{\fix}[1]{\textcolor{red}{#1}\xspace}
\newcommand{\pw}{Pila-Wilkie\xspace}
\newcommand{\pwt}{\pw Theorem\xspace}
\newcommand{\om}{o-minimal\xspace}
\newcommand{\Om}{O-minimal\xspace}
\newcommand{\omy}{o-minimality\xspace}
\newcommand{\Omy}{O-minimality\xspace}
\newcommand{\homeomic}{homeomorphic\xspace}
\newcommand{\homeom}{homeomorphism\xspace}
\newcommand{\cd}{cell-decomposition\xspace}
\newcommand{\cds}{cell-decompositions\xspace}
\newcommand{\CD}{Cell-Decomposition\xspace}
\newcommand{\CDs}{Cell-Decompositions\xspace}
\newcommand{\cmptble}{compatible\xspace}
\newcommand{\cmptblety}{compatibility\xspace}
\newcommand{\cont}{continuous\xspace}
\newcommand{\contty}{continuity\xspace}
\newcommand{\cnctd}{connected\xspace}
\newcommand{\cnctdns}{connectedness\xspace}
\newcommand{\defnb}{definable\xspace}
\newcommand{\defnbly}{definably\xspace}
\newcommand{\defnbty}{definability\xspace}
\newcommand{\defnblycnctd}{definably-connected\xspace}
\newcommand{\defnbcnctdns}{definable-connectedness\xspace}
\newcommand{\Mlt}{(M, <)}
\newcommand{\Mltp}{(M, <, \hdots)}
\newcommand{\M}{\mathcal{M}}
\newcommand{\Lstr}{\mathcal{L}\mathrm{-Structure}\xspace}
\newcommand{\Lstrs}{\mathcal{L}\mathrm{-Structures}\xspace}

\newcommand{\N}{\mathbb{N}}
\newcommand{\Q}{\mathbb{Q}}
\newcommand{\R}{\mathbb{R}}
\newcommand{\Z}{\mathbb{Z}}
\newcommand{\zeroset}{\{ 0 \}}
\newcommand{\fancyD}{\mathfrak{D}}
\newcommand{\Zgeq}[1]{\mathbb{Z}_{\geq #1}}
\newcommand{\Zleq}[1]{\mathbb{Z}_{\leq #1}}
\newcommand{\Jp}{J^{\prime}}
\newcommand{\Xprime}[1][X]{#1^{\prime}}
\newcommand{\In}[1][n]{(I)_{#1}}
\newcommand{\IIn}[1][n]{(II)_{#1}}
\newcommand{\vonethrum}[2]{#1_{1}, \hdots, #1_{#2}}
\newcommand{\Minf}{M \cup \{ \pm \infty \}}
\newcommand{\ionein}{i_1, \hdots, i_n}
\newcommand{\aoneak}{a_1, \hdots, a_k}
\newcommand{\vonevm}[2]{#1_1, \hdots, #1_{#2}}
\newcommand{\izerocell}{(\ionein, 0)\textrm{-cell}}
\newcommand{\ionecell}{(\ionein, 1)\textrm{-cell}}
\newcommand{\incell}{(\ionein)\textrm{-cell}}
\newcommand{\incells}{(\ionein)\textrm{-cells}}
\newcommand{\XsubseteqMn}[1][n]{X \subseteq M^{#1}}
\newcommand{\inhb}{non-empty\xspace}
% \newcommand{\inhb}{inhabited\xspace}
\newcommand{\inj}{injective\xspace}
\newcommand{\injtvty}{injectivity\xspace}
\newcommand{\surj}{surjective\xspace}
\newcommand{\surjtvty}{surjectivity\xspace}
\newcommand{\bij}{bijective\xspace}
\newcommand{\bijtion}{bijection\xspace}
\newcommand{\bijtions}{bijections\xspace}

\newcommand{\XsubsetMn}[1][n]{X \subset M^{#1}}

\newcommand{\image}[1]{\operatorname{image}\left(#1\right)}
\newcommand{\graph}[1]{\operatorname{graph}\left(#1\right)}
\newcommand{\dom}[1]{\operatorname{dom}\left(#1\right)}
\newcommand{\codom}[1]{\operatorname{codom}\left(#1\right)}
\newcommand{\inter}[1]{\operatorname{int}\left(#1\right)}
\newcommand{\cl}[1]{\operatorname{cl}\left(#1\right)}
\newcommand{\dcl}[1]{\operatorname{dcl}\left(#1\right)}
\newcommand{\acl}[1]{\operatorname{acl}\left(#1\right)}
\renewcommand{\dim}[1]{\operatorname{dim}\left(#1\right)}
\newcommand{\defnbdim}[1]{\operatorname{dim_{defn}}\left(#1\right)}
\newcommand{\algdim}[1]{\operatorname{dim_{alg}}\left(#1\right)}
\newcommand{\fr}[1]{\operatorname{fr}\left(#1\right)}
\newcommand{\funcdom}[3]{#1 \colon #2 \to #3}
\newcommand{\funcdef}[3]{#1 \colon #2 \mapsto #3}
\newcommand{\funcrestr}[2]{#1 \vert_{#2}}

%Sets
\DeclarePairedDelimiterX{\Set}[2]\{\}{%
  \, #1 \;\delimsize\vert\; #2 \,
}

%cardinality
\DeclarePairedDelimiter\card{\vert}{\vert}

%abs
\DeclarePairedDelimiter\abs{\vert}{\vert}

%norm
\DeclarePairedDelimiter\norm{\vert\vert}{\vert\vert}

%subproof
\newenvironment{subproof}[1][\proofname]{%
  \renewcommand{\qedsymbol}{$\blacksquare$}%
  \begin{proof}[#1]%
}{%
  \end{proof}%
}

%Fix example
\makeatletter
\let\example\relax
\spn@wtheorem{example}{Example}{\itshape}{\rmfamily}
\makeatletter

% See the list of further useful packages
% in the Reference Guide

\makeindex             % used for the subject index
                       %, please use the style and. ist with
                       % your makeindex program

%%%%%%%%%%%%%%%%%%%%%%%%%%%%%%%%%%%%%%%%%%%%%%%%%%%%%%%%%%%%%%%%%%%%%

\begin{document}

\author{Notes written by Alun Stokes}
\title{Course Notes on \Omy and the \pwt}
\subtitle{Presented by Gareth Jones as part of the Fields Institute Graduate Course on \Omy and Applications}
\maketitle
\frontmatter%%%%%%%%%%%%%%%%%%%%%%%%%%%%%%%%%%%%%%%%%%%%%%%%%%%%%%


%%%%%%%%%%%%%%%%%%%%%%% dedic.tex %%%%%%%%%%%%%%%%%%%%%%%%%%%%%%%%%
%
% sample dedication
%
% Use this file as a template for your own input.
%
%%%%%%%%%%%%%%%%%%%%%%%% Springer %%%%%%%%%%%%%%%%%%%%%%%%%%

\begin{dedication}
As the instructor for this module, these lecture notes are dedicated to Dr Gareth Jones, whose presentation of the Pila-Wilkie Theorem and its proof was both illuminating and intellectually challenging for this young graduate student.
\end{dedication}





%\include{foreword}
%%%%%%%%%%%%%%%%%%%%%%preface.tex%%%%%%%%%%%%%%%%%%%%%%%%%%%%%%%%%%%%%%%%%
% sample preface
%
% Use this file as a template for your own input.
%
%%%%%%%%%%%%%%%%%%%%%%%% Springer %%%%%%%%%%%%%%%%%%%%%%%%%%

\preface

This collection of notes results from a series of 8 lectures given by Dr Gareth Jones at the Fields Institute over the course of 19 Jan to 11 Feb 2022. The vast majority of the material presented here was given as part of this first module in the three-module course on \omy and Applications. Primarily, attempt was made to uniformize notation and formatting across the various lectures, and some extra details added where otherwise missing or left as an exercise to the viewer -- or even more likely, of some confusion to this author.

This current iteration of the notes is to be considered preliminary, and the style and particularities of their presentation are by no means final. As the owner of the sole pair of eyes to have read this document, no assurances are made that it is free of mistakes or oversights (in fact, one might be more willing to make promises to the contrary) -- but to the best of our ability, we have tried our best to minimize the sure-to-come lists of errata and corrigenda upon review.
 
This document intends to act as an easily (to an early graduate student in mathematics) accessible introduction to \omy in the context of expansions of the real field and their application in proving the \pwt. In particular, \omy is not discussed in full generality -- rather, the role the property of \omy of ordered fields plays in results on semi-algebraic sets, cell decompositions and parameterization by these decompositions, and of course, how this all comes together to prove the eponymous theorem of Pila and Wilkie. The ordering of content presented is not strictly adherent to the delineation given in the lectures, but it does broadly follow.

These notes are written, as most things are, from particular authorial perspective -- which is to say, not by those with more than little to any background in model theory. There may be at points a belabouring of ideas that another would find trivial, unnecessary, or otherwise not necessarily worth the space they take up on the page. The purpose of compiling these notes is not just to archive the lecture series given but also to make it somewhat more accessible by clarifying what we ourselves had to further investigate and understand as we attended the course. This does not mean any significant amount of material is added above and beyond the course content itself -- rather, more so that some `one-liners' in the original presentation are afforded just a few more in these notes, and some contextualization added in the first chapter to bridge the gap for a student for whom \omy is completely novel.

Where not otherwise cited, facts should be supposed to have been taken from the lectures — which themselves will periodically include references or suggestions for additional reading material. All citations are included for exteriorly sourced information, with the full list of references listed at the end of the document. This is contrary to the Springer Nature style of chapter-end citations, but the struggle with incorporating both this feature and BibTeX at the hands of this authorial team was nothing short of nightmarish. And, as with so many things in life, BibTeX should always come first.

Although at times dry, the content here, especially once one delves into it and splashes about in the waters of its intrigue, is quite the scene to behold. However, we would be remiss to assume this view is shared by one new to the subject area, or only acquainted with logics and not number theory, vice versa -- or (may the good lord help you) neither. Attempt at levity is made throughout in order to keep things interesting, whilst of course still rigorous, complete, and unobtrusive. Appropriateness of this attempt is yet to be determined, but it is the philosophy of this author that texts are to be enjoyed -- perhaps even the otherwise driest more so than any other; how else can we be expected to pay sufficient and adequate attention unless we are sufficiently engaged, amused, and interested all at once. Perhaps this is just one author's opinion unshared by the mathematical community at large -- but either way, you, my reader dearest, have no choice in the matter, and so shall have to decide for yourself the merit and suitability of the approach for the matters at hand. With all that in mind, please enjoy.
%\medskip

%\fix{An aside for Dr Speissegger: throughout, there are bits of red text indicating the non-inclusion or reduction of detail from what should otherwise be in these notes. This is just to clarify where some corners were cut in the interest of time -- as I tended to take a breadth-over-depth approach to attempt to put the course material into notes, feeling that would be a more `complete' version, despite being similarly unfinished either way.}

\vspace{\baselineskip}
\begin{flushright}\noindent
Stokes,  Alun C. \hfill {\it McMaster University}\\
\monthyeardate\today \hfill {\it MATH 712}\\
\end{flushright}



%%%%%%%%%%%%%%%%%%%%%%acknow.tex%%%%%%%%%%%%%%%%%%%%%%%%%%%%%%%%%%%%%%%%%
% sample acknowledgement chapter
%
% Use this file as a template for your own input.
%
%%%%%%%%%%%%%%%%%%%%%%%% Springer %%%%%%%%%%%%%%%%%%%%%%%%%%

\extrachap{Acknowledgements}

I would like to acknowledge Dr Gareth Jones for his wonderful series on the content presented hence, along with Drs Ya'acov Peterzil and Jacob Tsimerman for jointly organizing the course. I would like to thank the Fields Institute for hosting this graduate course, and finally Dr Patrick Speissegger as the intructor of record for MATH 712 at McMaster University.


\tableofcontents

%%%%%%%%%%%%%%%%%%%%%%acronym.tex%%%%%%%%%%%%%%%%%%%%%%%%%%%%%%%%%%%%%%%%%
% sample list of acronyms
%
% Use this file as a template for your own input.
%
%%%%%%%%%%%%%%%%%%%%%%%% Springer %%%%%%%%%%%%%%%%%%%%%%%%%%

\extrachap{Acronyms}

The following are used varyingly judiciously throughout.
\begin{description}
\item[CD]{Cell Decomposition}
\item[PW]{Pila-Wilkie}
\item[DLO]{Dense Linear Order}
\item[FOL]{First-Order Language}
\item[MT]{Monotonicity Theorem}
\item[FT]{Finiteness Theorem}
\item[UF]{Uniform Finiteness}
\item[QE]{Quantifier Elimination}
\item[DC]{Definable Choice}
\item[CS]{Curve Selection}
\end{description}



\mainmatter%%%%%%%%%%%%%%%%%%%%%%%%%%%%%%%%%%%%%%%%%%%%%%%%%%%%%%%
%%%%%%%%%%%%%%%%%%%%%part.tex%%%%%%%%%%%%%%%%%%%%%%%%%%%%%%%%%%
% 
% sample part title
%
% Use this file as a template for your own input.
%
%%%%%%%%%%%%%%%%%%%%%%%% Springer %%%%%%%%%%%%%%%%%%%%%%%%%%

\begin{partbacktext}
\part{\Omy and Necessary Concepts}
\noindent This introductory part will focus primarily on setting up the setting in which we find ourselves working for most of the course. Highlights include ``\emph{what even is \omy?}'' and ``\emph{but how does that imply \pw?}'' -- and perhaps a favourite of mine: ``\emph{what is the \pwt?}''. While the latter two are answered in much more detail, later on, this part will take us quickly through some of the significant results we'll find ourselves needing in a bit. In particular, we prove the monotonicity theorem (MT), define cell decompositions (CD) and much later on \emph{smooth} cell decompositions -- then prove the Cell Decomposition theorem itself, Later on, we discuss dimensionality and the agreement between the definable (or geometric) and algebraic (or model-theoretic) notions of the quantity -- something that will come to be quite useful to us later on. We finally end off with that promised discussion of \scds, although in comparison to other sections, it really is more a series of results and sketches (if that) of how one might prove these things, as they broadly build on either proof structures we will have seen before, or shouldn't be too hard in conjunction with a dusting off of evertone's favourite: Calculus: Early Transcendentals -- of which I'm sure we all have a copy somewhere.

The prepared reader should find themselves acquainted with preliminary ideas in mathematical logic (nothing further than one would find in an undergraduate class on the subject) and the basics of field theory. Again, nothing further than an undergraduate would necessarily be expected to have encountered.

A reader should leave this section feeling themselves reasonably well-acquainted with some of the tools they might see themselves using in general sorts of proofs about \om structures and how they may go about proving or disproving something to be \om. Some preliminary notions of how this all fits into counting points of bounded height on curves may be coming to surface by the end of this part, but the novice reader would be well-forgiven were that not the case. They should, however, feel comfortable identifying what is `clearly' a definable set and be able to do some rudimentary reasoning on how we can use the finiteness and definability of one of a pair of complementary sets to say something about the definability of the other. The idea of cells and cell decompositions should be reasonably understood (at least \emph{intuitively} if not in full technical detail). The reader should have a map of sorts in their mind that sequentializes and connects the discussed matter in a reasonable and meaningful way -- as these preliminary ideas form the basis for the larger \lemmas and theorems to come. By the end of this part, the well-established reader should find the proof presented rather intuitive, and even (hopefully) find that they belabour the points they make \emph{too} much for how clearly obvious they are.

\end{partbacktext}

%%%%%%%%%%%%%%%%%%%%% chapter.tex %%%%%%%%%%%%%%%%%%%%%%%%%%%%%%%%%
%
% sample chapter
%
% Use this file as a template for your own input.
%
%%%%%%%%%%%%%%%%%%%%%%%% Springer-Verlag %%%%%%%%%%%%%%%%%%%%%%%%%%
%\motto{Use the template \emph{chapter.tex} to style the various elements of your chapter content.}
\chapter{A Bit Before we Begin}
\label{bit-before} % Always give a unique label
% use \chaptermark{}
% to alter or adjust the chapter heading in the running head
%\chaptermark{Some Introductions}

\abstract*{We begin with just a few words in preparation for what is to come; some definitions, expectations of the experience (or lack thereof) on the part of the reader, and a general outline are given. The overly excited reader may feel free to skip right onto Chapter \ref{setting}, but this section serves as just a bit of an \textit{amuse-bouche} for those not so ready to jump right in.}

\abstract{We begin with just a few words in preparation for what is to come; some definitions, expectations of the experience (or lack thereof) on the part of the reader, and a general outline are given. The overly excited reader may feel free to skip right onto Chapter \ref{setting}, but this section serves as just a bit of an \textit{amuse-bouche} for those not so ready to jump right in.}

\section{Before the Lectures Proper}
We feel somewhat compelled to address an aspect of this topic that we felt was slightly neglected in the curriculum. Had one seen the list of attendees to these lectures, the reason for skipping over such `trivialities' as we are about to point out briefly is clear — with several attendees being former students of Pila or Wilkie themselves. Still, for the \emph{not-even-amateur} logician, the following certainly bears some explicit mention.

In the absence of the results we come to find, \omy may appear a relatively unmotivated idea to study. Of course, as the pure mathematicians we are (or hope one day to be), why \emph{shouldn't} the mere concept of further understanding be enough to compel our interest? Still, the progression from introductory mathematical logic and the importance and usefulness of quantifier elimination to \omy is one that was unapparent to this author until being elsewhere noted. We don't claim this to be a failure of the course so much as a failure in personal preparation, but should the reader find themselves similarly underprepared, then they will find themselves thankful for this little pretext.

To keep things brief, we will say just this: in a predicate logical system, we are interested in quantifier elimination. The result of the elimination of quantifiers is essentially the answer to the question a quantified statement asks. Perhaps the most famous example of this is the existence of real roots of quadratic equations. We ask the quantified `question':
\begin{align}
  \exists x \in \R . (a \cdot x^2 + b \cdot x + c = 0 \wedge a \neq 0)
\end{align}
— that is, does there exist such an $x$? The quantifier eliminated equivalent form is
\begin{align}
  b^2 - 4 \cdot a \cdot c \geq 0 \wedge a \neq 0,
\end{align}
the first half of which should be recognized as the quadratic discriminant (and the second just to ensure non-degeneracy). Here, quantifier elimination gives the exact and deterministic characterization of the answer to the quantified statement -- and it is this property that motivates its study. We trust at this point that the motivation has been sufficiently belaboured.

It is known that first-order theories with quantifier elimination (that is, decidability for the theory can be reduced to the question of satisfaction of quantifier-free sentence in the theory) are model complete. In the interest of not straying too far, we leave it to the reader to believe or convince themselves that this is a desirable property. While this is not the focus of how we define \omy in this course, it is true that a structure is \om exactly if every formula given by no more than one free variable and some subset of $M$-parameters is equivalent to a quantifier-free formula defined only by these parameters, and the ordering on the structure \cite{marker_model_2002}. Thus, for anyone finding themselves perhaps unconvinced upfront of the merit of some of the ideas explored here (outside the Pila-Wilkie Theorem), then we hope this motivates the sequence we are about to take on. And for everyone else, we hope that this section did not bore too thoroughly.


\section{Preliminary Definitions}
\label{sec:prelim-defns}
Throughout, we will be working with models $ \M = \Mlt$ of the theory of dense linear orders (DLO) \emph{without endpoints}. For now, $M$ will be fixed, but we will look at some specific instances later on. Perhaps then, one of the most important definitions, to begin with, is that of \emph{definability}.

%\begin{definition}[Definability of sets (without parameters) \cite{marker_basic_2002}]
\begin{definition}[Definability of sets (without parameters)]
For $n \in \Zgeq{1}$, we say a set $A \subseteq M^n$ is \emph{definable without parameters} if there exists some formula in our model, $ \phi$, satisfied exactly by the elements of $A$.
\end{definition}

%\begin{definition}[Definability of sets \cite{marker_basic_2002}]
\begin{definition}[Definability of sets]
For $X \subseteq M$ and $n \in \Zgeq{1}$, then we say a set $A \subseteq M^n$ is \emph{definable with parameters from $X$} if there exists some formula in our model, $ \phi$, and elements $b_1, \hdots, b_m$, such that $ \phi$ is satisfied exactly by the elements of $A$ along with the parameters in $X$.
\end{definition}

Notice then that definability without a parameter is simply the case of definability with parameters coming from the empty set. These definitions immediately and naturally induce the idea of definable functions and definable points. In particular, a function is definable in parameters if its graph is definable by those same parameters in $\M = \Mlt$. Similarly, an element $a$ is definable in $\M$ (with parameters) if the singleton $\{a\}$ is definable in $ \M$ by those same parameters. This isn't something we will need to consider too extensively.

As ever, when introduced to a novel space, we are interested in what its open intervals look like. We have the following characterization:

\begin{definition}[Open Interval]
  A set, $A \subset M$ is an open interval in $\Mlt$ if $A$ is of one of the following forms:
  \begin{itemize}
    \item $(a, b)$ with $a < b \in M$
    \item $(- \infty, a)$ with $a \in M$
    \item $(a, \infty)$ with $a \in M$
  \end{itemize}
\end{definition}

We say further that intervals of the first type — that is, those having finite bounds — are \emph{bounded}. Easy to miss but important to note is that the endpoints must sit inside our domain. So, for example, in $(\Q, <)$, the set $(-5, \sqrt{7})$ is \emph{not} an open interval.

We imbue $M$ with the order topology and $M^n$ with the product topology. We then define what it means to be an \om expansion.

\begin{definition}[\Om expansion]
  Taking $\mathcal{M} = \Mltp$ an expansion of $\Mlt$, we say $\mathcal{M}$ is \om if every definable (with parameters) subset of $M$ is given by a finite union of open intervals and points.
\end{definition}

\begin{svgraybox}
  If we weaken the above and ask only for \emph{convex} sets (which are a superset of our open intervals) in place of open intervals, then the above would define \emph{weak \omy} — but that won't be a topic of discussion here.
\end{svgraybox}

For the etymologically inclined, it is noted that the `o' in \om comes from the shortening of `order-minimality'. For more information on the history and development of the idea of \omy, one may reference
\textbf{Tame Topology \& \Om Structures} \cite{dries_tame_1998} or \textbf{Definable Sets in Ordered Structures I} \cite{pillay_definable_1986} and \textbf{II} \cite{knight_definable_1986}.


Some (arguably) simple examples of \om structures are given by expansions of the real field. Consider, for example, $\overline{\R} = (\R, <, +, -, \cdot, 0, 1)$ and the further extension $\R_{\textrm{exp}} = (\overline{\R}, \exp)$, both of which are \om. Don't mistake my use of `simplicity' as an indication that these were trivial or did not require particular and considerable consideration — rather, just that they have a relatively simple-seeming form. For now and going forward, we fix $\M$ an \om structure and move on to our first theorem.


%%%%%%%%%%%%%%%%%%%%%%%% referenc.tex %%%%%%%%%%%%%%%%%%%%%%%%%%%%%%
% sample references
% %
% Use this file as a template for your own input.
%
%%%%%%%%%%%%%%%%%%%%%%%% Springer-Verlag %%%%%%%%%%%%%%%%%%%%%%%%%%
%
% BibTeX users please use

\bibliography{chapter_11}


%%%%%%%%%%%%%%%%%%%%% chapter.tex %%%%%%%%%%%%%%%%%%%%%%%%%%%%%%%%%
%
% sample chapter
%
% Use this file as a template for your own input.
%
%%%%%%%%%%%%%%%%%%%%%%%% Springer-Verlag %%%%%%%%%%%%%%%%%%%%%%%%%%
%\motto{Use the template \emph{chapter.tex} to style the various elements of your chapter content.}
\chapter{Setting it all up}
\label{setting} % Always give a unique label
% use \chaptermark{}
% to alter or adjust the chapter heading in the running head
%\chaptermark{Some Introductions}

\abstract*{We now begin properly with a from-the-basics definition of the objects at play: field expansions, monotonicity, cells and decompositions into them, semi-algebraicity and similarly fundamental ideas are each defined and contextualized. Note that we will not be discussing topological definitions in general. That is to say, the reader is assumed to be familiar with the basic point-set topology and the ordinary sorts of topologies we see cropping up (e.g. order, product) -- not that topological ideas won't be discussed. As well, basic knowledge of mathematical logic is assumed; first-order languages (FOL), $ \Lstrs$, relations, and satisfiability are all presumed familiarities. With definability now a part of our tool-set, we start by proving a few theorems fundamental to results later in this course.}

\abstract{We now begin properly with a from-the-basics definition of the objects at play: field expansions, monotonicity, cells and decompositions into them, semi-algebraicity and similarly fundamental ideas are each defined and contextualized. Note that we will not be discussing topological definitions in general. That is to say, the reader is assumed to be familiar with the basic point-set topology and the ordinary sorts of topologies we see cropping up (e.g. order, product) -- not that topological ideas won't be discussed. As well, basic knowledge of mathematical logic is assumed; first-order languages (FOL), $ \Lstrs$, relations, and satisfiability are all presumed familiarities. With definability now a part of our tool-set, we start by proving a few theorems fundamental to results later in this course.}


\section{On Monotonicity}
\chaptermark{Monotonicity}
What constitutes a `nice' property of a function is generally non-contentious; injectivity and surjectivity are often useful -- together even more so -- and it would be the odd mathematician to turn their nose up at a function being bounded, supposing they weren't chasing a nasty counterexample or engaging in some other such endeavour. At present, we will focus on the property of \emph{monotonicity} and when we can determine a definable function to be monotonic in the context of open intervals. The following was proved in \cite{pillay_definable_1986} by Pillay and Steinhorn:

\begin{theorem}[The Monotonicity Theorem]
	\label{thm:monotonicity}
  Suppose $f \colon I \to M$ is a definable function for $I \subset M$ an open interval. Then there exist $a_1, \hdots, a_k \in I$ such that on each adjacent interval, $(a_j, a_{j+1})$ (where $I = (a_0, a_{k+1})$) $f$ is either constant, or strictly monotonic and continuous. Further, if $f$ is definable over some $A \subseteq M$, then so too are $a_1, \hdots, a_{k}$ definable over $A$.
\end{theorem}

Hence, we will refer to this simply as the Monotonicity Theorem, abbreviated by MT. It is perhaps not immediately apparent why this should be true, or even that we should be interested that it is. The answer to the second point is that this piece-wise continuity and monotonicity of definable functions is a relatively rigid condition, and this (not just here but for structures in general) allows us to say a good bit about them. Observe that if we have some $X \subseteq M$ definable and infinite, then X must contain some open interval. This should be relatively intuitive, even if a proof doesn't come to you immediately, given what we've covered thus far. As for why the Monotonicity Theorem holds, we show this by piecing together three lemmata that should make the picture a bit more clear. Throughout, take $J \subset I$ as an open interval. To not get bogged down in the minutiae of their proofs as we go through — not that they are particularly challenging — but in any case, we will state all three and then prove them sequentially.

\begin{lemma}
\label{lemma:monotonic-1}
  There is an open interval, $\Jp \subseteq J$, on which $f$ is constant or injective.
\end{lemma}

\begin{lemma}
\label{lemma:monotonic-2}
  If $f$ is injective on $J$, then there is an open interval, $\Jp \subseteq J$ on which $f$ is strictly monotonic.
\end{lemma}

and finally,

\begin{lemma}
\label{lemma:monotonic-3}
  If $f$ is a strictly monotonic function on $J$, then there exists some open interval $ \Jp \subseteq J$ on which $J$ is continuous.
\end{lemma}

Taking these lemmata for granted, it is not terribly difficult to see how the Monotonicity Theorem falls out. The fun then is in proving these three facts — which is nice, as they are not terribly complicated.

We start where any sensible person would.

\begin{proof}[Lemma \ref{lemma:monotonic-1}]
  Suppose there is some $y \in M$ such that its preimage under $f$ intersected with $J$ is infinite. This necessarily implies the existence of $ \Jp \subseteq J$ an open interval on which $f$ takes constant value, and so we can assume for any $y \in M$ that we have $f^{-1}(y) \cap J$ is finite. Then, we must have f(J) infinite, and so contains interior with subset $ (a, b) $, for $a < b$. Taking
  \begin{align*}
    q \colon (a, b) &\to J \\
    q \colon y &\mapsto \min{\Set{x \in J}{f(x) = y}},
  \end{align*}
  we get q injective — and so this is an open interval $ \Jp \subseteq q((a, b))$ on which $f$ is injective.
\end{proof}


\begin{proof}[Lemma \ref{lemma:monotonic-2}]
  This we can get quite quickly. Suppose such a strictly monotone function exists on $J$. Clearly, $f$ cannot be constant (else monotonicity would be non-strict), and so by \omy of $f$, we get that the image of $J$ under $f$ contains some open interval, $ \Jp \subseteq \image{f}$, on which we have preimage a sub-interval of $J$. We get monotonicity on this interval by Lemma \ref{lemma:monotonic-1} and non-constancy (and thus monotonicity) of $f$; this must be a bijection (either order-preserving or reversing, but bijective either way), and so we are finished.
\end{proof}

\fix{
\begin{proof}[Lemma \ref{lemma:monotonic-3}]
  Write me.
\end{proof}
}


\begin{proof}[Theorem \ref{thm:monotonicity}]
  We now combine these three lemmata to get our result. Take $A$ the set of all $x \in I$ (coming from our original theorem statement) such that $f$ is both continuous and strictly monotone at $x$. We know that taking the restriction of $f$ to some open sub-interval on which $f$ is defined maintains both continuity and monotonicity by Lemmata 2 and 3 — and so taking the set difference of A from I, the original open interval, we cannot have \emph{any} open intervals. There are then thus only finitely many points, and the theorem follows.
\end{proof}

\begin{svgraybox}
	Take note that the proof provided here is \emph{not} precisely the one that was given in the lecture, but rather a bit more condensed, less roundabout method of achieving the result. The strategy is the same, however, differing only in presentation.
\end{svgraybox}

\fix{
Two Exercises Lec1 pg 4
}

The following result is a special case in 2 dimensions of what is referred to as the \emph{Finiteness Theorem}, abbreviated FT. We first prove this special case and then take a brief detour to talk about cell decompositions before we can address the more general theorem.

\section{The (Planar) Finiteness Theorem}
\chaptermark{Special Finiteness}

\begin{theorem}[Finiteness Theorem in $M^2$]
	\label{thm:2finiteness}
	Suppose $A \subseteq M^2$ and that for each $x \in M$, the fibre $A_x$ above $x$ — that is, the set of $y$ with $(x, y) \in A$ — is finite. Then, there exists some $N \in \Zgeq{1}$ such that $\card{A_x} \leq N$ for all $x \in M$
\end{theorem}

\begin{proof}[Finiteness Theorem in $M^2$ (Theorem \ref{thm:2finiteness})]
	We define a point $(a, b) \in M^2$ to be \emph{normal} if it sits in an open box, $I \times J$ satisfying
	\begin{itemize}
		\item $(I \times J) \cap A = \emptyset$
		\item $(a, b) \in A$
		\item There exists a continuous $f \colon I \to M$ such that $(I \times J) \cap A = \graph{f}$.
	\end{itemize}

	Similarly, for points with only one finite endpoint, we say some $(a, \infty)$ (resp. $(a, - \infty)$) is \emph{normal} if there exists open interval $I$ such that $a \in I$ and some $b \in M$ such that
	\begin{align*}
		(I \times (b, \infty)) \cap A = \emptyset
	\end{align*}
	and again, respectively taking $(b, - \infty)$ for the other case.

	Supposing we take the set $\Set{(a, b) \in M^2}{(a, b) \ \textrm{is normal}}$, it easily follows that this set is definable, and similarly so for the $\{\pm \infty\}$ cases. We now define functions $f_1, f_2, \hdots, f_n$ by the property that
	\begin{align*}
		\dom{f_k} = \Set{x \in M}{\card{A_x} \geq k}.
	\end{align*}
	We have the property that $f_k(x)$ is the $k$-th element of $A_x$ — and so we get the definability of each $f_k$ by the finiteness of each fibre.

	Fixing some $a \in M$ and taking $n \geq 0$ maximal such that all of $f_1, \hdots, f_n$ are defined and \emph{continuous} on an open interval around $a$. We then say that $a$ is
	\begin{itemize}
		\item \textbf{good} if  $a \notin \cl{\dom{f_{n + 1}}}$ and otherwise
		\item \textbf{bad} if $a$ \emph{is} in this closure.
	\end{itemize}
	We partition into $G = \Set{a \in M}{a \ \textrm{is good}}$ and $B = \Set{a \in M}{a \ \textrm{is bad}}$. What we will now show is that $G$ is definable — which we do by showing that for any $a \in B$, there is a minimal $b \in M \cup \{\pm \infty \}$ such that $(a, b)$ is \emph{not} normal.

	Let $a \in B$. We use the following notation for convenience:
	\begin{description}
		\item
			\begin{align*}
						\lambda(a, -) = \begin{cases}
									      \displaystyle\lim_{x \to a^{-}} f_{n + 1}(a) & \colon \ \textrm{$f_{n+1}$ defined on $(t, a)$ for some $t < a$.} \\
									      \infty & \colon \ \textrm{else}
									   \end{cases}
			\end{align*}

		\item
			\begin{align*}
						\lambda(a, 0) = \begin{cases}
									      f_{n + 1}(a) & x \in \dom{f_{n+1}} \\
									      \infty & \colon \ \textrm{else}
									   \end{cases}
			\end{align*}

		\item
			\begin{align*}
						\lambda(a, +) = \begin{cases}
									      \displaystyle\lim_{x \to a^{+}} f_{n + 1}(a) & \colon \ \textrm{$f_{n+1}$ defined on $(a, t)$ for some $a < t$.} \\
									      \infty & \colon \ \textrm{else}
									   \end{cases}
			\end{align*}
	\end{description}

	Take $\beta(a) = \min{\{ \lambda(a, -),\ \lambda(a, 0),\ \lambda(a, +) \}}$. It is not difficult to see then that $\beta(a)$ is simply the least $b \in \Minf$ such that $(a, b)$ is not normal. Were we instead to take some $a \in G$, then $(a, b)$ must \emph{always} be normal for any $b \in \Minf$. So, $B$ can be given as
	\begin{align*}
		B = \Set{a \in M}{\exists b \in \Minf \ \textrm{s.t.} \ (a, b) \ \textrm{is not normal}},
	\end{align*}
	and as such, is definable.

	If we take some $a \in G$, then $\card{A_x}$ is constant on an open interval about $a$ by definition of $G$. By showing that $B$ is finite, we get our desired result. Supposing $B$ to be \emph{infinite}, we can partition $B$ into
	\begin{align*}
		B_+ &= \Set{a \in B}{\exists y \ \textrm{s.t.} \ y > \beta(a), \ (a, y) \in A} \\
		B_- &= \Set{a \in B}{\exists y \ \textrm{s.t.} \ y < \beta(a), \ (a, y) \in A},
	\end{align*}
	both evidently definable sets. By the infinitude of $B$, so too must at least one of $B_-, B_+$ be infinite — and further, so must one of
	\begin{itemize}
		\item $B_+ \cap B_-$
		\item $B_+ \setminus B_-$
		\item $B_- \setminus B_+$
		\item $B \setminus (B_+ \cup B_-)$.
	\end{itemize}
	We can then apply MT (Theorem \ref{thm:monotonicity}) to each case to reach a contradiction by showing that assuming non-finiteness, we \emph{should} be able to find a normal point with first coordinate $a$ -- contradicting the `badness' of any point in $B$. Thus, $B$ is \emph{finite}, and so there must be some finite upper bound on the cardinality of all fibres, $A_x$, and our proof is complete.
\end{proof}



\section{Cell Decompositions}
\chaptermark{Cells and Decompositions Into Them}

We start with a few definitions that should hopefully feel motivated in anticipation of the higher-dimensional analogues of what we have seen already.

\begin{definition}[Cells in $M^n$]
  For a sequence $(i_1, \hdots, i_n)$ for each $i_j \in \{0, 1\}$, we define $(i_1, \hdots, i_n)$\emph{-cells} of $M^n$ inductively as follows:
  \begin{enumerate}
    \item A $0$-cell is a point in $M$, and a $1$-cell an open interval (both in $M^1$).
    \item Supposing $\incells$ are defined for $M^n$,
	\begin{enumerate}
	  \item we define an $\izerocell$ to be a definable set given by $\graph{f}$ for $f$ a continuous, definable function on an $\incell$.
	  \item Perhaps predictably then, we define an $\ionecell$ to be a definable set of the form $(f, g)_C = \Set{(x, y) \in C \times M}{f(x) < y < g(x)}$ for $f, g$ continuous, definable functions on an $\incell$, with $C \subset M^n$. Note that we may also allow $f \equiv - \infty$ or $g \equiv \infty$.
	\end{enumerate}

  \end{enumerate}

  As usual, we denote a projection map by $\pi$, and for any $\incell$ we can define the projection
  \begin{align*}
    \pi \colon M^n \to M^k
  \end{align*}
  for $k$ the sum of $\ionein$, such that the restriction of $\pi$ to our $\incell$ is a homeomorphism onto its image.
\end{definition}

It is not hard to see that what we are doing here is just projecting away from the coordinate 0 parts of the cell. This can be thought of as a canonical coordinate projection that any cell comes naturally equipped with, which is quite a fine thing to have at hand.

In what should hopefully be predictable at this point, we wish now to define what it means to \emph{decompose} our space into cells. At some point, we will cease prefacing these definitions with `as usual, we do so by induction' -- but that point is yet to come. So, as usual, we proceed by defining \cds by induction.

\begin{definition}[Cell Decomposition \emph{of $M$}]
  A \emph{\cd} of $M$ is a finite set defined by some strictly increasing finite sequence $\aoneak$ that form the set
  \begin{align*}
    \{(- \infty, a_1), (a_1, a_2), \hdots, (a_k, \infty), \{a_1\}, \{a_2\}, \hdots \{a_k\} \}.
  \end{align*}
  That is, all the sequential open intervals (including those with infinite endpoints) plus the singleton sets. Just for the sake of belabouring the point, this is a definitionally \emph{definable} set.
\end{definition}

As we have time and time before, we now up the dimension by induction to define general \cds:

\begin{definition}[Cell Decomposition \emph{of $M^{n+1}$}]
  A \cd of $M^{n+1}$ is a finite partition, $\fancyD$, of $M^{n+1}$ into cells, such that
  \begin{align*}
    \Set{\pi (C)}{C \in \fancyD}
  \end{align*}
  is itself a decomposition of $M^n$, with each $\pi$ the respective projection as discussed above.
\end{definition}


We trust the conjunction of these two definitions into an understanding of \cds in arbitrary dimensions is clear by induction.
An idea that may seem a bit apropos until a bit later on is that of \emph{compatibility} -- but rest assured, dear reader, that this will all come together shortly.

\begin{definition}[Compatibility]
  We call a \cd $\fancyD$ of $M^n$ \emph{compatible} with a subset, $X \subseteq M^n$ if for each cell, $C \in \fancyD$, either $C \cap X$ is empty, or $C$ is a subset of $X$.
\end{definition}


\section{The \CD Theorem}
\chaptermark{The \CD Theorem}

The importance of this last point will be made clear in the theorem we have built up to the \CD Theorem -- which essentially says that these compatible decompositions exist and are compatible with any finite collection of definable sets and, most importantly, that definable functions are continuous on each cell in such a decomposition (defined in the domain of the function, of course). Properly, and as proved by Knight, Pillay, and Steinhorn \cite{knight_definable_1986}:

\begin{theorem}[\CD]
  \label{thm:cell-decomposition}
  Take $n \in \Zgeq{1}$.
  \begin{svgraybox}
    Note the use (about to be made) of subscripts on statements $(I)_n$ and $(II)_n$ to denote the dimension of $M^n$ to which each statement refers. This is going to be notationally useful in the proof, but may seem a bit queer at present, and without introduction.
  \end{svgraybox}
 Then
  \begin{enumerate}[label={}]
    \item[$\In$ ] Suppose $\vonevm{X}{k} \subseteq M^n$ are \defnb sets. Then there is a \cd of $M^n$ \cmptble with each $X_j$.
    \item[$\IIn$ ] If $\funcdom{f}{X}{M}$ is \defnb, then there is a \cd, $\fancyD$ of $M^n$ \cmptble with $X$ s.t. the restriction $\funcrestr{f}{C}$ is \cont for each $C \in \fancyD$.
  \end{enumerate}

  Further, and in analogy to the Monotonicity Theorem, if our $\vonevm{X}{k}$ or $f$ (depending on case $\In[]$ or $\IIn[]$) are \defnb over $A \subset M$, then we can take the cells in  $\fancyD$ to be similarly \defnb over $A$; that is, with the \emph{same} parameters.

  \begin{svgraybox}
    This last point is perhaps a bit unfair to mention, as we will not be providing a proof for it -- though for the sake of interest, it would feel incomplete to not at least analogize with Theorem \ref{thm:monotonicity}. In truth, what follows is not a full proof of \CD, but a special case where we take $M$ to be $\R$ and use the yet unproven (or even stated) result of \emph{uniform finiteness}. We take this result entirely for granted in the lectures due to the oddity that a complete (unassuming) proof somewhat `bootstraps' uniform finiteness into the induction we do on $\In[j]$ and $\IIn[j]$, proving it as we go along. This is because uniform finiteness is actually itself an immediate consequence of the \CD Theorem (which makes the proof a fun little oddity). For our purposes, we take it as assumedly true -- in part due to the length of this proof even \emph{with} that assumption -- and trust that our dear intelligent reader sees plainly how we could fix this in the absence of the assumption.
  \end{svgraybox}

\end{theorem}

Uniform finiteness is a generalization of the finiteness theorem we proved earlier (Theorem \ref{thm:2finiteness}), but with potentially many parameters and higher dimensions. As with the argument for the \CD Theorem, we will similarly restrict our attention to the case where $M = \R$. This special case is as follows:

\begin{proposition}[Uniform Finiteness (for $\R$)]
  \label{prop:unif-finiteness}
  Suppose $X \subset \R^{n+1}$ is \defnb with each fibre $X_x$ finite for $x \in \R^n$. Then there is some $N \in \Zgeq{1}$ such that $\card{X_x} \leq N$ for all $x \in \R^n$.
\end{proposition}


The following proof is due to van den Dries \cite{dries_remarks_1984} which, for no reason other than interest's sake, we mention went on to inspire the later work of Pillay and Steinhorn in \cite{pillay_definable_1986}.

\begin{proof}[\CD (Theorem \ref{thm:cell-decomposition})]
  We proceed by induction on parameter $n$. The base cases are both already done for us; $\In[1]$ is immediate from the definition of \omy, and $\IIn[1]$ is given by the Monotonicity Theorem. What we go on to show is two inductive facts that `bounce off' one another in a sense, to allow us to prove both $\In$ and $\IIn$ for all $n$. These are
  \begin{enumerate}[label=(\alph*)]
    \item \label{pf:cd:a} Given $\In[1], \hdots, \In$ and $\IIn[1], \hdots, \IIn[n-1]$, we can conclude $\IIn$; and
    \item \label{pf:cd:b} Given $\In[1], \hdots, \In$ and $\IIn[1], \hdots, \IIn[n]$, we can conclude $\In[n+1]$.
  \end{enumerate}
  That these two facts together give us the desired result should be clear.
  Getting there requires a bit more effort, and so we simply begin with \ref{pf:cd:a}.
  Thus we wish to prove $\IIn[n]$: that for a \defnb $\funcdom{f}{X}{M}$, there is a \cd $\fancyD$ of $M^n$ \cmptble with $X$ and having \contty of $\funcrestr{f}{C}$ for each cell, $C \in \fancyD$.

  Suppose $\funcdom{f}{X}{M}$ is such a definable function. We assume (because we already have) $\In[1]$. By this, we may assume $X$ is a cell. If $X$ is not already an open cell, then recall that we can take its image under the canonical projection away from zero coordinates. Since we do not refer here to the dimension of $X$, we assume that it is open or has been made so as described and then use our inductive hypothesis to conclude. So, we suppose $X \in \fancyD$ is an open cell on which $f$ is \cont. Take
  \begin{align*}
	\Xprime = \Set{x \in X}{f \ \textrm{is \cont and \defnb at} \ x}.
  \end{align*}
  Clearly, $\Xprime$ is \defnb, and we are supposing that we know $\Xprime$ to be open in $X$. Using inductive assumption $\In[n]$, we get a \cd, $\fancyD$ with $\R^n$ \cmptble with $X \setminus \Xprime$ and with $\Xprime$. If some $C \in \fancyD$ is an open cell contained in $X$, we get \contty of $f$ on $C$  by density; that is, $C \cap \Xprime \neq \emptyset$, and so $C \subseteq \Xprime$ and it follows that $\funcrestr{f}{C}$ is \cont. Supposing however that $C$ was \emph{not} an open cell, we apply the aforementioned projection construction, and the argument just presented holds (up to a change in dimension).

  This would be all well and good to end off \ref{pf:cd:a} with, were it not predicated on the yet unjustified density of $\Xprime$ in $X$ -- and so we now prove this. Suppose $B \subseteq X$ is an open box. We will show that there must exist a point in $B$ at which $f$ is \cont. In analogy to our proof of monotonicity, we know that if $\Xprime[B]$ is an open box contained inside of $B$, then $f$ takes on infinitely-many values on $\Xprime[B]$ (following from $\In$). This is the obvious case. Supposing otherwise, we proceed as follows:

  Construct a sequence of open boxes, $(B_j)_{1 \leq j \leq n}$ in $B$, and sequence $(I_j)_{1 \leq j \leq n}$, of open intervals, each $I_j$ having length less than $\frac{1}{j}$, with the closure, $\cl{B_{n+1}} \subseteq B$, and $f(B_n) \subseteq I_n$. Then, by compactness. we get that the intersection of all $B_n$ is non-empty, and at some point in this intersection, $f$ is continuous. This is of course just our claim -- we now go on to \emph{prove} this by construction.

  To get $I_1$, simply consider $f(B) \subseteq \R$ -- meaning
  \begin{align*}
    f(B) = \bigcup_{p \in \Zgeq{1}} J_p \cup F
  \end{align*}
  for $F$ a finite set, and $J_p$ a countable set of open intervals of length less than 1. Then, $B$ is given by
  \begin{align*}
    B = \left( \bigcup_{p \in \Zgeq{1}} f^{-1}(J_p) \cap B \right) \cup \left( \bigcup_{r \in F} f^{-1}(r) \cap B \right).
  \end{align*}

  To each half of the middle cup, we can apply $\In$ to determine the contents of each of the respective \emph{big} cups to be a finite union of cells -- and so $B$ must be an \emph{countable} union of cells, each of which is contained in one of these sets. Perhaps coming a bit out of left field, we apply the Baire Category Theorem to conclude that by the openness of $B$, so too must be one of these cells be open.
  \begin{svgraybox}
    Notice that this is one reason we restrict ourselves to working over $\R$ -- the Baire Category Theorem simply does not hold in any DLO model. So this argument could not be broadened beyond the reals (or compact spaces) as we are currently undertaking it.
  \end{svgraybox}
  This \emph{cannot} be one of $f^{-1}(r) \cap B$, as it would then contain a box on which we took the value of $r$, an so this open cell must be in one of $f^{-1}(J_p) \cap B$ for some $p$. Taking $J_1$ to be that $J_p$, and $B_1$ to be an open box contained in $f^{-1}(J_1) \cap B$, with $\cl{B_1} \subseteq B$. As desired, we then have $f(B_1) \subset I_1$. Clearly the first step in an induction, we then (incompletely) note that, having $\vonevm{I}{n}$, $\vonevm{B}{n}$ constructed, we repeat exactly as above to finish the induction.

  And with that, we can give ourselves a \emph{light} patting on the back -- for as much as we've done so far, this is just the end of the proof of \ref{pf:cd:a}. To get the `bounced-back' half of the induction, we now go on to prove \ref{pf:cd:b}; that is, given $\vonevm{(I)}{n}$, $\vonevm{(II)}{n}$, we may derive $(II)_{n+1}$.


  For reasons of breaking up this lengthy proof into its two constituent sections, please enjoy the following horizontal line:

  \centerline{\rule{0.6667\linewidth}{.2pt}}
  \medskip

  Try not to have too much fun with that, now. We move on to proving \ref{pf:cd:b}; recall our assumptions that $\vonevm{(I)}{n}$ and $\vonevm{(II)}{n}$ hold. We want now to prove $\IIn[n+1]$. First, we start with a small proposition.

  \begin{proposition}
    \label{prop:cell-refinement}
    Suppose $\fancyD_1$, $\fancyD_2$ are \cds of $\R^{n+1}$ with a common refinement -- that is, another \cd, $\fancyD$ of $R^{n+1}$ compatible with all cells in each of $\fancyD_1$ and $\fancyD_2$. Terminology-wise, we say that $\fancyD$ \emph{refines} $\fancyD_1$ and $\fancyD_2$, or that is is an \emph{refinement} of $\fancyD_1$ and $\fancyD_2$.
  \end{proposition}

  \begin{svgraybox}
    A frustrated author's aside: please take no notice of the $\square$ that sits just above this box and proceeds Proposition \ref{prop:cell-refinement}. Its presence is a mystery, and the method to remove it proves elusive, even after what would be not ungenerously called a \emph{cursory} amount of investigation. We will simply pretend it does not exist and trust that the honest, caring reader does so as well.
  \end{svgraybox}


  \begin{subproof}[Cell Refinement (Proposition \ref{prop:cell-refinement})]
    For the purpose of transparency, we note that this proof was left out of the lecture and as an exercise for the interested (or obligated) viewer. The following takes inspiration from van den Dries \cite{dries_tame_1998}. We note that this could be made a bit cuter if we had the machinery of \emph{dimension} that we will soon define, but in either case, this proof is relatively trivial. We have our $\fancyD_1$ and $\fancyD_2$ two decompositions of the trivially definable subset of, $\R^{n+1}$: $\R^{n+1}$ itself. We can then simply take a decomposition of the ambient space (which here is the \emph{whole} space) containing our definable subset. We have seen previously that we can take this decomposition to partition each cell of $\fancyD_1 \cup \fancyD_2$ -- and the `restriction' of this decomposition to our definable set (again, just to ensure this is sufficiently belaboured, this is not actually a restriction since our definable set is $\R^{n+1}$), we are left with our everywhere (on cells in $\fancyD_1 \cup \fancyD_2$) compatible decomposition.
  \end{subproof}

  \begin{svgraybox}
    A much less frustrated author's aside: let us all take a moment and appreciate the appropriately placed $\square$ above. We can move forth pretending all is well again, and we hope this has not caused the reader \emph{too} much undue stress -- beyond, of course, the normal, cursory amount.
  \end{svgraybox}

  Now, if some $A \subseteq \R$ is \defnb, we define its type, $\tau(A)$ as follows:
  \begin{description}
    \item Let $\vonevm{a}{L}$ strictly increasing be the points in the boundary of $A$. We let $\tau(A)$ then act as an indicator function on sequential intervals, $(a_j, a_{j+1})$, defined as the positive unit (1) when that interval sits inside $A$, and otherwise the negative unit ($-1$). We set $a_0 = - \infty$ and $a_{L+1} = \infty$ (which is starting to become sort of an out-of-bounds normalcy), and define
    \begin{align*}
      \tau_{2j + 1} = 1
    \end{align*}
    if $(a_j, a_{j+1}) \subseteq A$ (and of course $-1$ otherwise). Note, of course, that this would then mean that the given interval is contained in the complement of $A$. For even numbers, we have
    \begin{align*}
      \tau_{2j} = 1
    \end{align*}
    if $a_j \in A$ and naturally, $-1$ if $a_j \not\in A$. Then, we have $\tau(A) = (\vonevm{\tau}{2 \cdot L + 1}) \in \{0, 1\}^{2 \cdot L + 1}$ a sequence of length $2 \cdot L + 1$ consisting of $\pm 1$s.

  \end{description}

    To ground ourselves for a moment, consider $$\tau((1, 2] \cup \{ 3 \} )) = (-1,\ -1,\ +1,\ +1,\ -1,\ +1,\ -1),$$
    which can immediately convince ourselves is non-unique, as this is the same sequence induced by $\tau((9, 10] \cup \{5,\ 7\} )$.

    At this point, you may be a bit confused (if the recollection is even still there) as to why we made such a fuss early on about \emph{uniform finiteness} -- when we've yet to see it used. Well, for the anxious amongst you, satisfaction will come soon, as we now make use of that perhaps unjustified assumption.

    By UF, we get the following (and again, \underline{\textbf{please}} ignore the spurious QED-type symbol that appears at the end on this proposition),

    \begin{proposition}
      \label{prop:definable-fibre-types}
      If we have a \defnb $X \subseteq \R^{n+1}$, then the set of the types of fibres -- that is
      \begin{align*}
        \Set{\tau(X_x)}{x \in \R^n}
      \end{align*}
      -- is finite. Further, for each given choice of type, the set of fibres giving rise to that type is \defnb.
    \end{proposition}
    Perhaps starting this sentence with `of course' would be unfair, but it shouldn't be hard to see, intuit, or at least \emph{guess} that the set of $x \in \R^n$ gives rise to any \emph{particular} type is usually empty. As before, this proof was left as an exercise to the responsible party, and so please forgive any clear indications of amateurism -- were they absent, there may be something a little suspect going on.

    \begin{proof}(Proof of Proposition \ref{prop:definable-fibre-types})
      This we will not belabour even slightly. We have assumed UF and so simply appeal to UF in the case of $M^{n}$, by which we get that each fibre is finite -- and so must have finite types belonging to its elements. In the case of each type, it is given by some point or open interval in a fibre and so is defined by points and open intervals. Supposing there were infinitely-many \emph{different} such types, we would have to be working in a non-finite dimension. Thus, \defnbty falls out, almost as if by accident.
    \end{proof}

    Now, with these two propositions, we are just about ready to put together our proof of \ref{pf:cd:b}. That is, we now prove $\In[n+1]$.

    By Proposition \ref{prop:cell-refinement} (on cell-refinement), we can assume $k = 1$ -- which you'll recall is the number of sets we have from our original statement of the theorem. Then, by Proposition \ref{prop:definable-fibre-types} and $\In$, we get a \cd, $\fancyD$ of $\R^n$ such that for each $C \in \fancyD$ there is an $L$ and a $\tau \in \{\pm 1\}^{2 \cdot L + 1} $ such that
    $$
      \tau(X_x) = \tau
    $$
    for all $x \in C$. Fixing such $C$, $\tau$, and $L$, we get \defnb functions $\vonevm{f}{L}$ with each $f_1 < \hdots < f_L$ and such that either
    \begin{enumerate}
      \item $(f_i, f_{i+1})_C \subseteq X$; or
      \item $(f_i, f_{i+1})_C \cap X = \emptyset$,
    \end{enumerate}
    with the normal condition of $f_0 = - \infty$, $f_{L+1} = \infty$. Notice also that the same holds for the graphs of $f_j$ -- that they are either contained in or disjoint from $X$, excluding of course the cases at infinities (which we allowed above). Now, with our small army of \defnb functions, all defined of this cell $C \in \R^n$, we can use $\IIn$ (which we proved in our induction in part \ref{pf:cd:a}) to conclude that we may partition $C$ into finitely-many cells, such that $f$ is \cont on each cell. With the end very nearly in sight, we apply $\In$ to get a \cd of $\R^n$ (and in fact of all cells from the above) \cmptble with all the resulting cells. Finally, taking graphs over those cells will give us the desired \cd of $\R^{n+1}$. And with that, we have `bounced back' such that we may repeat this induction \textit{ad infinitum}, and the \CD Theorem is proved (in our special case) for all $n$. Please note the now properly-placed well-deserved QED-symbol. Revel a little, if you must.
\end{proof}

It is now at \emph{this} point that the reader not only \emph{may} but is encouraged to give themselves their well-deserved, no-bars-held pat on the back for the fortitude it took to get through that. Perhaps also giving the above a quick re-read wouldn't be such a bad idea, as there are some bits and subtleties that this author needed a few passes to feel entirely comfortable with. For a proof that doesn't make the assumptions we did here, the reader is directed to \cite{knight_definable_1986}, but by no means necessarily encouraged towards it —- just made aware of its existence. With that done, we are through one of the more laborious parts of this first part of the course. For the masochists in the audience, we note that there is more length and labour to come of this variety. Still, for the normal amongst us, it is with a relaxation that we should move on to discuss dimensionality, and the problem that mathematicians have for coming up with distinct words. But first -- which is a phrase we perhaps begin sentences with all too often -- we have a bit of miscellany to address.

%%%%%%%%%%%%%%%%%%%%% chapter.tex %%%%%%%%%%%%%%%%%%%%%%%%%%%%%%%%%
%
% sample chapter
%
% Use this file as a template for your own input.
%
%%%%%%%%%%%%%%%%%%%%%%%% Springer-Verlag %%%%%%%%%%%%%%%%%%%%%%%%%%
%\motto{Use the template \emph{chapter.tex} to style the various elements of your chapter content.}
\chapter{In Absences of Aproposia}
\label{chap:apropos} % Always give a unique label
% use \chaptermark{}
% to alter or adjust the chapter heading in the running head
%\chaptermark{Some Introductions}

\abstract*{Here, we mention a point or two that would otherwise go overlooked. Necessity is entirely eschewed, and this chapter can be safely skipped without any loss in understanding the course as it was intended to be presented. You are encouraged to leave. This short chapter exists only for the interested, dedicated, obliged, or otherwise neurotic reader. In short, this encapsulates that which has no other place thus far, nor is deserving of its own section.}

\abstract{Here, we mention a point or two that would otherwise go overlooked. Necessity is entirely eschewed, and this chapter can be safely skipped without any loss in understanding the course as it was intended to be presented. You are encouraged to leave. This short chapter exists only for the interested, dedicated, obliged, or otherwise neurotic reader. In short, this encapsulates that which has no other place thus far, nor is deserving of its own section.}

\section{Let's Just Get it Out of the Way}
\label{sec:aproposia-is-not-a-word}
\noindent One of the first thoughts the critical or unassuming reader -- which is likely all of you -- had when reading this chapter title was whether \emph{that} word was really even a word \emph{at all}. If you are at all a curious person, searching the internet or your favourite etymological website or reference text for the word `aproposia,' you would have failed in your task -- unless the task you set out was to assure yourself that it is, in fact, \emph{not} a common or even extant word. Were `apropos' of Latin root, then perhaps this linguistic abomination may make more sense, but considering the French origin of `apropos,' no such logic applies. Nonetheless, there is little the reader can do about this choice of wording (save for one particular professor). And since we felt its hopefully clear meaning and appealing sound appropriate to the nature of the topic, you should then be glad at all, dear reader, that Latin is no longer the language of science you would be expected to learn to have what would be considered `valid' opinions on its workings. If anything, feel free to use it as word to befuddle and confuse your dear friends and colleagues. While we will let you get away with calling everything and anything `normal', we insist you take `aproposia' as valid and unilaterally call that a fair trade.

\section{Onto the Miscellany}

In which we again abuse language, in the sense that there is only one fact of miscellaneous variety. Sometimes, certain benign ridiculousness must be allowed to amuse your readers or even just oneself. What we do, after all, is exciting not in its protracted execution, but in the few moments where it all comes together. It's how we prevent the onset of early insanity when getting into the thick of these sorts of ideas.

\begin{remark}
  Recall that an \om structure requires all \defnb sets \emph{with parameters} to be given by finite unions of points and open intervals (as defined by the model). If we only assume this for sets \defnb \emph{without} parameters, the resulting theory is \emph{legitimately} and provably weaker than what we get with \omy. This was in passing mentioned to be potentially true earlier on, but in one of the question and answer sessions held for this course, it was pointed out that it is \emph{in fact} true by a gentleman with the given name Chris. In a moment, we will be referring to him by his family name, Miller. This awkward wording will be clear in just a moment.

  An easy (in the sense of being counter-exemplary) way to show this is due to Dolich, Miller, and our old friend Steinhorn \cite{dolich_structures_2009}. This can be expressed (though perhaps not proven, as the length of their paper implies) quite compactly by constructing the model
  $$
    \M = (\R, <, V)
  $$
  for $V$ the Vitali set (defined by the Vitali \emph{relation}):
  $$
    V = \Set{(x, y) \in \R^2}{x - y \in \Q}.
  $$
  The only $\emptyset$-\defnb subsets of this are $\emptyset$ itself and $\R$ -- and so this fits the definition of being $\emptyset$ \om, but given any defined parameter, we end up with the rationals \defnb; clearly, this is \emph{not} \om. If you recall the short mention made earlier, it may interest the reader to note that this is \emph{weakly} \om.

  \begin{exercise}
    As an exercise of \emph{this} author to the reader, attempt to prove that the above expansion admits QE. \textit{Hint: You would be well-served to first read Chapter \ref{chap:dimensionality} on dimensionality and definable closures.}
  \end{exercise}
\end{remark}

Putting an end to this brief foray into intrigue with a splash of ridiculousness, we now move back on to a consequential idea once one defines \cds: connectedness.


%%%%%%%%%%%%%%%%%%%%% chapter.tex %%%%%%%%%%%%%%%%%%%%%%%%%%%%%%%%%
%
% sample chapter
%
% Use this file as a template for your own input.
%
%%%%%%%%%%%%%%%%%%%%%%%% Springer-Verlag %%%%%%%%%%%%%%%%%%%%%%%%%%
%\motto{Use the template \emph{chapter.tex} to style the various elements of your chapter content.}
\chapter{Connectedness, and What it Entails}
\label{connectedness} % Always give a unique label
% use \chaptermark{}
% to alter or adjust the chapter heading in the running head
%\chaptermark{Some Introductions}

\abstract*{It is likely that one unacquainted too thoroughly with this topic of study has been imagining in their minds open intervals as they might imagine them in $\R$. Unfortunately for such a reader, they are about to be disabused of that rather idyllic notion -- and we will speak to when we legitimately \emph{may} presume that things are (what we will come to call) \emph{\defnbly \cnctd}. At this point, we should have an acronym to express that the importance of this will not be immediately apparent but will soon come to be and be worked into our standard model of understanding these objects we find ourselves working with. The creation of such an acronym is left as an exercise to the reader.}

\abstract{It is likely that one unacquainted too thoroughly with this topic of study has been imagining in their minds open intervals as they might imagine them in $\R$. Unfortunately for such a reader, they are about to be disabused of that rather idyllic notion -- and we will speak to when we legitimately \emph{may} presume that things are (what we will come to call) \emph{\defnbly \cnctd}. At this point, we should have an acronym to express that the importance of this will not be immediately apparent but will soon come to be and be worked into our standard model of understanding these objects we find ourselves working with. The creation of such an acronym is left as an exercise to the reader.}


\section{But \emph{Why} Aren't Things (Always) Connected?}
\label{sec:connectedness-defn}

The easy answer is: sometimes we just don't work over connected domains. Take, for example, $(\Q, <)$ -- where we may write $\Q = (- \infty, \sqrt{2}) \cup (\sqrt{2}, \infty)$.

\begin{svgraybox}
  Recall, of course, that neither interval is open in $\Q$ by non-membership of the irrational endpoints in the domain.
\end{svgraybox}

In particular, we find that a convenient definition of a \defnblycnctd set, $X \in M^n$ is

\begin{definition}[Definable Connectedness]
  Some $X \in M^n$ is \defnblycnctd if $X$ is \emph{not} the union of two disjoint non-empty \defnb open subsets of X.
\end{definition}


\begin{example}
  Open intervals, we claim, are \defnblycnctd. So too are cells -- although this one we reason a bit about. By \CD, we know that \defnb sets have finitely-many \defnblycnctd components which are maximal \defnbly \cnctd subsets. So by uniformity, it seems that we can always take a cell to not be the union of two disjoint non-empty \defnb open subsets of itself.

\end{example}

We formalize this idea with the following proposition.

\begin{proposition}
  \label{prop:dfnblycnctd}
  Support some $X \in M^{m + n}$ is \defnb. Then, there exists some $N \in \Zgeq{1}$ such that if $x \in M^m$. then $X_x$ has \emph{at most} $N$ \defnblycnctd components.
\end{proposition}

\begin{corollary}
  Given $\mathcal{N}$, a structure elementarily equivalent to $\M$ (satisfies exactly the same first order sentences in our language) for $M$ \om, then $N$ is also \om. In short,
  \begin{align*}
    \M \equiv \mathcal{N} \ \wedge \M \ \mathrm{\om} \implies \mathcal{N} \ \mathrm{\om}.
  \end{align*}
\end{corollary}

\begin{svgraybox}
  For the interested and more informed reader than expected necessarily, it is noted that the property of \emph{minimality} (not \omy) is \emph{not} preserved under elementary equivalence as \omy is. This is one of the motivations for the idea of \emph{strong}-minimality, but for all intents and purposes here, we pretend there is no notion of a \emph{strong}-\omy.
\end{svgraybox}

This corollary will come to be quite important later, so if nothing else from here, keep that fact in the back of the mind as we go forward.

\section{Definable Choice \& Curve Selection}
\noindent In the interest of time, space, audience, and simply relevance, in most cases, we will not be providing examples as we have before (as in the case of expansion by the Vitali relation) and instead just assume we take $\M$ an \om expansion of an \emph{ordered} field, $(M, <, +, \cdot, 0, 1)$. Not addressed here, but as a good exercise to the interested reader, attempt to show that this must necessarily be a real closed field. We will think in abstractness only, and the reader who even tries to think of an example should be rather a bit ashamed of what they've done.

Without any faff, we get right into the point of this section.
 \begin{proposition}[Definable Choice]
  \label{defnb-choice}
  \begin{enumerate}
    \item \label{defnb-choice-1} Given a \defnb family, $X \subseteq M^{n+m}$ with $\pi$ the projection map onto the first $n$ coordinates, then there is a \defnb map, $\funcdom{f}{\pi X}{M^n}$ with $\graph{f} \subseteq X$.
    \item \label{defnb-choice-2} Given $E$ a \defnb equivalence relation on a \defnb set, $X \subseteq M^n$, then $E$ has a set of representatives.
  \end{enumerate}

\end{proposition}
\begin{proof}[Proofs of the above (Proposition \ref{defnb-choice}]
  First, we go on to show that if some $\XsubsetMn[n]$ is \defnb and \inhb, then we may \defnbly pick some element, $e(X) \in X$. As ever, we induct on $n$. 
  
  Suppose $n=1$, then either
  \begin{enumerate}
    \item $X$ has a least element, and so we let $e(X)$ be that; and otherwise
    \item $X$ has a left-most interval, (with respect to our order), and splitting by cases, we can take
      \begin{itemize}
        \item $e(X) = 0$ if $(a, b) = (- \infty, \infty)$;
        \item $e(X) = b - 1$ if $a = - \infty$, $b \in M$;
        \item $e(X) = a + 1$ if $a \in M$, $b = \infty$; and
        \item $e(X) = \cfrac{a + b}{2}$ if both are finite.
      \end{itemize}
  \end{enumerate}
  Note, of course, that our arithmetic is well-defined here due to the field expansion we are working with. We now induct a bit differently to prove each of cases \ref{defnb-choice-1} and \ref{defnb-choice-2};
  
  \begin{itemize}
    \item In the first case (definable without parameters), we put $f(x) = e(X_n)$ for $x \in \pi X$ (since our fibre is \inhb; and then for 
    \item we take $\Set{e(A)}{A \ \textrm{is an equivalence class of} \ E}$ a \defnb set or representation, as desired.
  \end{itemize}
\end{proof}

We may go on to refer to \emph{definable choice} as DC, stated in the acronym section upfront. From this, we can then go on to prove a neat and useful little result called \emph{curve selection}.

\begin{proposition}[Curve Selection]
  \label{prop:curve-selection}
  Suppose $\XsubsetMn$ is \defnb and $a \in \fr{X}$ (the frontier of $X$, defined by to the closure of $X$ less $X$). Then, there is a \cont \defnb \inj $\funcdom{\gamma}{(0, \epsilon)}{X}$ for some $\epsilon > 0$ with
  \begin{align*}
    \lim_{t \to 0^{+}} \gamma (t) = a.
  \end{align*}
\end{proposition}

Predictably, this is going to use the result we just proved on \defnb choice, so we just jump right in.

\begin{proof}[of Proposition \ref{prop:curve-selection}]
 Let $\abs{x} = \max \{ \abs{x_1}, \abs{x_2}, \hdots, \abs{x_n} \}$ . Since $a \in \fr{X}$, the set 
   \begin{align*}
     \Set{\abs{a - x}}{x \in X}
   \end{align*}
   is a \defnb set with arbitrarily small positive elements, and contains some interval $(0, \epsilon)$.
   
   If $t$ is in this $(0, \epsilon$), then the set 
   \begin{align*}
     \Set{x \in X}{\abs{a - x} = t}
   \end{align*}
  is \inhb. Thus, by DC, we get a \defnb $\funcdom{\gamma}{(0, \epsilon)}{X}$ such that $\abs{a - \gam(t)} = t$ for some $t$ in our interval. Clearly then, $t$ is \inj with limit $a$ as $t$ approaches 0 (on the right). As a throwback, we can apply the Monotonicity Theorem and reduce $\epsilon$ to reach the assumption that $\gamma$ is \cont.
\end{proof}

We've perhaps teased at it now for a bit, and the particularly knowledgable or prescient reader will have seen this coming, but it is at this point we move on to a spot of dimension theory. The discussion here is interesting in that we approach it both from the angle of definability (as expected) and algebraicity and see what comes to pass.

%%%%%%%%%%%%%%%%%%%%% chapter.tex %%%%%%%%%%%%%%%%%%%%%%%%%%%%%%%%%
%
% sample chapter
%
% Use this file as a template for your own input.
%
%%%%%%%%%%%%%%%%%%%%%%%% Springer-Verlag %%%%%%%%%%%%%%%%%%%%%%%%%%
%\motto{Use the template \emph{chapter.tex} to style the various elements of your chapter content.}
\chapter{Dimensionality}
\label{chap:dimensionality} % Always give a unique label
% use \chaptermark{}
% to alter or adjust the chapter heading in the running head
%\chaptermark{Some Introductions}

\abstract*{As so often one finds in mathematics, dimension is one of those ideas that it almost seems each individual mathematician has a notion of how it should be defined -- and more truthfully, it has hugely varying meanings across the vast spectra of mathematical disciplines. And perhaps even more, unfortunately, it so often seems the case that the average mathematician was raised on a dictionary of no more than 30 or so words -- which they go on to use and reuse and smash together into all-new words, for the most part equally inequivalently to how their office-mate might be doing the same thing in their preferred area of abstractness. To say the same in so many fewer words: we often find that the same words mean different things to different mathematicians -- and here, we are going to study two interpretations of one such word -- that being \emph{dimension}. We look at both the algebraist's and logician's definition (at least in the context of \omy) of the dimension of a structure and then happily go on to find out that they actually intersect. This goes on to be fundamental, and this fact is part of what allows us to connect the seemingly non-number-theoretic ideas examined hence, to those that allow us to go on to prove a theorem about points of bounded height on curves; a very number-theoretic idea indeed.}

\abstract{As so often one finds in mathematics, dimension is one of those ideas that it almost seems each individual mathematician has a notion of how it should be defined -- and more truthfully, has hugely varying meanings across the vast spectra of mathematical disciplines. And perhaps even more, unfortunately, it so often seems the case that the average mathematician was raised on a dictionary of no more than 30 or so words -- which they go on to use and reuse and smash together into all-new words, for the most part equally inequivalently to how their office-mate might be doing the same thing in their preferred area of abstractness. To say the same in so many fewer words: we often find that the same words mean different things to different mathematicians -- and here, we are going to study two interpretations of one such word -- that being \emph{dimension}. We look at both the algebraist's and logician's definition (at least in the context of \omy) of the dimension of a structure and then happily go on to find out that they actually intersect. This goes on to be fundamental, and this fact is part of what allows us to connect the seemingly non-number-theoretic ideas examined hence, to those that allow us to go on to prove a theorem about points of bounded height on curves; a very number-theoretic idea indeed.}


\begin{svgraybox}
  We continue in our assumption from the previous section -- that is, that we take $\M$ an \om expansion of an \emph{ordered} field, $(M, <, +, \cdot, 0, 1)$, and prove things about this construction in generality. Strictly speaking, this is not necessary here, even speaking less strictly than we have about this assumption earlier on. It is simply a matter of convenience and lack of loss of generality.
\end{svgraybox}


\section{Dimension: So How Does One Define That?}
\label{sec:dim-defn}

\noindent Tempting though it is to answer ``depends on who you ask'' and move on with our day, the question does bear significant thought. We start with the following definition.

\begin{definition}
  Suppose $\XsubsetMn$ is \defnb and \inhb. Then we set 
  \begin{align*}
    \dim{X}  = \max{\Set{\sum_{j = \vonethrum{j}{n}} i_{j}}{X \ \textrm{contains an} \ \incell}}
  \end{align*}
  to be the dimension of our subset of $M^n$, with the dimension of $\emptyset$ being $- \infty$.
\end{definition}
At first blush, this seems to be a reasonable definition of dimension given the manner in which we've been defining the rest of our toolkit -- and with further blushing, we will come to find it even more reasonable than it may even have initially appeared. For those finding this an \emph{unreasonable}, we would encourage a re-reading of some of the earlier definitions, specifically on cells and their decompositions or, barring that, just setting fire to this manuscript and going on about your day. Either way, we say a little something about what subsets with non-empty interior may tell us.

\begin{lemma}
  \label{lemma:interior-open-cell}
  If $\XsubsetMn{n}$ has interior, $\inter{X}$ \inhb, there thesis is a \defnb \inj map, $\funcdom{f}{X}{M^n}$ with image of $X under f$ ($f(X)$) containing an open cell.
\end{lemma}
This lemma will go one to be quite useful in a moment when we wish to say some useful things about, for example (and in particular), the \emph{dimension invariant of \defnb \bijtions }.

\begin{proof}[of Lemma \ref{lemma:interior-open-cell}]
  We promise you now, barring life-threatening or otherwise dire circumstances, that this will be the last preface to a proof of this sort -- and trust you, dear reader, to recognize when we are setting up a proof by inductions on $n$. But for now, we proceed by induction on $n$.
  
  When $n = 1$, then $\XsubsetMn{}$ is a \inhb subset of $M$, and so is infinite. Since $f$ is injective, $f(X)$ so too is it infinite (relying as well on \refnb to conclude this), and so $f(X)$ contains an open interval -- which we know to be an open-cell. By CD, we will simply assume $X$ is itself already an open-cell
\end{proof}
%\include{appendix}



\bibliography{biblio}

\backmatter%%%%%%%%%%%%%%%%%%%%%%%%%%%%%%%%%%%%%%%%%%%%%%%%%%%%%%%

\printindex

%%%%%%%%%%%%%%%%%%%%%%%%%%%%%%%%%%%%%%%%%%%%%%%%%%%%%%%%%%%%%%%%%%%%%%

\end{document}

