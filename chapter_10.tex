%%%%%%%%%%%%%%%%%%%%% chapter.tex %%%%%%%%%%%%%%%%%%%%%%%%%%%%%%%%%
%
% sample chapter
%
% Use this file as a template for your own input.
%
%%%%%%%%%%%%%%%%%%%%%%%% Springer-Verlag %%%%%%%%%%%%%%%%%%%%%%%%%%
%\motto{Use the template \emph{chapter.tex} to style the various elements of your chapter content.}
\chapter{Mapping the Landscape and Telegraphing our Journey} 
\chaptermark{Let's Orient Ourselves}
\label{orientation} % Always give a unique label
% use \chaptermark{}
% to alter or adjust the chapter heading in the running head
%\chaptermark{Some Introductions}

\abstract*{How one starts with the Taskian ideas introduced in their first course in mathematical logic and ends up studying \omy to the end of proving the \pwr is perhaps and unsurprisingly eminently unclear to any who do not already know the methodology. So we don't become too disenfranchised and uninspired as we work our way through the logical results to the end of a number-theoretic result. We layout a brief map of what is to come and how each part logically follows from its antecedent. This is not perhaps the most interesting course for one interested solely in number theory or simply mathematical logics, but where these two unlikely friends collide to create something wondrous and beautiful.}

\abstract{How one starts with the Taskian ideas introduced in their first course in mathematical logic and ends up studying \omy to the end of proving the \pwt is perhaps and unsurprisingly eminently unclear to any who do not already know the methodology. So we don't become too disenfranchised and uninspired as we work our way through the logical results to the end of a number-theoretic result. We layout a brief map of what is to come and how each part logically follows from its antecedent. This is not perhaps the most interesting course for one interested solely in number theory or simply mathematical logics, but where these two unlikely friends collide to create something wondrous and beautiful.}

\section{And \pw is?}
\noindent Perhaps the best and most prudent question to be asking one's self currently, if for no other reason than determine the worth of their time in reading this whole affair, is what the statement of the \pwt \emph{actually} is? And what, supposing the reader knows the context in which we define \omy, could that have to do with a \ntc result like the \pwt? Well, dear reader, we hope in this brief first section to enlighten you to the big ideas upon which we will ruminate for the remainder of this course, and provide a coarse outline of how these come to build on one other in order to start from the relatively basic, to the phenomenal. To live up to this section's name, however, we now state the \pwt -- first informally, and then as we will come to prove it. In all cases, however, note that we are speaking to \om expansions of the \emph{real} field, and we will not be covering the theorem and all that leads up to it in full generality.

\begin{theorem}[\pwt (Informal)]
  \label{thm:pwt_informal}
  Let $\tilde{\R}$ an \om expansion of $(\R, <)$. Then \emph{transcendental} \defnb sets have very \emph{few} rational points.
\end{theorem}

Easily understandable, seems reasonable, and (maybe?) doable without too much fuss, doesn't it seem? Here now is the formal statement, which requires the following crash course in notation; we denote by $H$ the usual rational multiplicative height function -- but it also doubles, when not used as a function, as an upper-bound on rational numbers we are interested in (I didn't decide the notation). As well, when taking vectorial input, the height is given by the element-wise maximum. Suppose $X \subseteq \R^n$. Denote $X(\Q) \coloneqq X \cap \Q$ and further $\XQH \subseteq X(\Q)$ with element height as given by the function, $H$, and bounded by the constant, $H$. Finally, the superscripts $\emph{tr}$ and $\emph{alg}$ refer to the transcendental and algebraic parts of the set they sit atop (note of course that necessarily we have $X \setminus X^{\textrm{tr}} = X^{\textrm{alg}}$). What we then want, and what \pw gives us, are good bounds on $\XQH$

