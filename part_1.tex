%%%%%%%%%%%%%%%%%%%%%part.tex%%%%%%%%%%%%%%%%%%%%%%%%%%%%%%%%%%
% 
% sample part title
%
% Use this file as a template for your own input.
%
%%%%%%%%%%%%%%%%%%%%%%%% Springer %%%%%%%%%%%%%%%%%%%%%%%%%%

\begin{partbacktext}
\part{\Omy and Necessary Concepts}
\noindent This introductory part will focus primarily on setting up the setting in which we find ourselves working for most of the course. Highlights include ``\emph{what even is \omy?}'' and ``\emph{but how does that imply \pw?}'' -- and perhaps a favourite of mine: ``\emph{what is the \pwt?}''. While the latter two are answered in much more detail, later on, this part will take us quickly through some of the significant results we'll find ourselves needing in a bit. In particular, we prove the monotonicity theorem, define cell decompositions (CD), then prove the cell decomposition theorem, and top it off with a discussion of dimensionality and the agreement between the geometric and algebraic notions of the quantity.

The prepared reader should find themselves acquainted with preliminary ideas in mathematical logic (nothing further than one would find in an undergraduate class on the subject) and the basics of field theory. Again, nothing further than an undergraduate would necessarily be expected to have encountered.

A reader should leave this section feeling themselves reasonably well-acquainted with some of the tools they might see themselves using in general sorts of proofs about \om structures and how they may go about proving or disproving something to be \om. Some preliminary notions of how this all fits into counting points of bounded height on curves may be coming to surface by the end of this part, but the novice reader would be well-forgiven were that not the case. They should, however, feel comfortable identifying what is `clearly' a definable set and be able to do some rudimentary reasoning on how we can use the finiteness and definability of one of a pair of complementary sets to say something about the definability of the other. The idea of cells and cell decompositions should be reasonably understood (at least \emph{intuitively} if not in full technical detail). The reader should have a map of sorts in their mind that sequentializes and connects the discussed matter in a reasonable and meaningful way -- as these preliminary ideas form the basis for the larger \lemmas and theorems to come.

\end{partbacktext}
