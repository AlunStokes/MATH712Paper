%%%%%%%%%%%%%%%%%%%%% chapter.tex %%%%%%%%%%%%%%%%%%%%%%%%%%%%%%%%%
%
% sample chapter
%
% Use this file as a template for your input.
%
%%%%%%%%%%%%%%%%%%%%%%%% Springer-Verlag %%%%%%%%%%%%%%%%%%%%%%%%%%
%\motto{Use the template \emph{chapter.tex} to style the various elements of your chapter content.}

\subsection{Some References and Applications}
\sectionmark{What We Don't See}
\label{chap:unfinished}

And that's about it, really. After this point in the course, there were two main applications discussed and in relatively (but not excessively) considerable detail. For the reader's sake, we do include the references given in the lectures, should they be interested in straying away on their own. These are
  \begin{itemize}
    \item Thomas Scanlon (\textit{Note: Not T.M. Scanlon, despite sharing a given and family name}) who has written a number of papers in this area, but whom we cite for his survey paper of the topic in general \cite{Scanlon2012CountingSP}.
    \item Philipp Habegger (who you'll recall from before) and Martin Orr, each of whom has writings in the proceedings from a summer school at the University of Manchester (and for which we do not have a citation here).
  \end{itemize}

There is a good bit of discussion of elliptic curves and manages to cover such theorems as the Ax-Lindeman-Weierstrass theorem, the Manin-Mumford conjecture, and the Andr\'e-Oort conjecture. Further on, there is some discussion of results to do with Galois bounds (bounds on Galois orbits of torsion points) broadly following results originally devised by Schmidt \cite{schmidt2018} (but relying on \pw, of course). The applications lecture is chock full of wonderfully interesting results that are omitted here only for reasons of time (on the part of the authorial team) and not relevance or interest. The interested reader is highly encouraged to take a watch, should they find the time, of the final archived lecture.

\subsection{\textit{Venimus, Vidimus, Vicimus} (mostly) -- and Now We Are Done}
This has been quite a journey, and from where we stand now, it may feel as if we spent much of it wading about in the shallow end before a jam-packed and speedy tour through the deep end that actually led us directly into proving the \pw theorem proper. However, for the young graduate student with little experience in the area, this author, at least, can say definitively that as much was to be gained in those slower, early sections of Part I as was in the whirlwind proofs of Part II. Overall, we hope these notes are useful, elucidative, interesting, or at the very least, searchable for any who may find themselves in want or need of a write-up of this first of three modules in the lectures series on \omy and the \pw theorem. Thank you for bearing with us, and deepest apologies for what is left out in what might otherwise have been a very interesting part IV: Applications.