%%%%%%%%%%%%%%%%%%%%%%preface.tex%%%%%%%%%%%%%%%%%%%%%%%%%%%%%%%%%%%%%%%%%
% sample preface
%
% Use this file as a template for your own input.
%
%%%%%%%%%%%%%%%%%%%%%%%% Springer %%%%%%%%%%%%%%%%%%%%%%%%%%

\preface

This collection of notes results from a series of 8 lectures given by Dr Gareth Jones at the Fields Institute over the course of 19 Jan to 11 Feb 2022. All the material presented here was given as part of this first module in the three-module course on \Omy and Applications. Primarily, attempt was made to uniformize notation and formatting across the various lectures, and some extra details added where otherwise missing or left as an exercise to the viewer.

This current iteration of the notes is to be considered preliminary, and the style and particularities of their presentation are by no means final. As the owner of the sole pair of eyes to have read this document, no assurances are made that it is free of mistakes or oversights -- but to the best of our ability, we have tried our best to minimize the sure-to-come lists of errata and corrigenda upon review.
 
This document intends to act as an easily (to an early graduate student in mathematics) accessible introduction to \omy in the context of expansions of the real field and their application in proving the \pwt. In particular, \omy is not discussed in full generality -- rather, the role the property of \omy of ordered fields plays in results on semi-algebraic sets, cell decompositions and parameterization by these decompositions, and of course, how this all comes together to prove the eponymous theorem of Pila and Wilkie. The ordering of content presented is not strictly adherent to the delineation given in the lectures, but it does broadly follow.

These notes are written, as most things are, from the author's perspective -- which is to say, someone with little to no background in model theory. There may be at points a belabouring of ideas that another would find trivial, unnecessary, or otherwise not necessarily worth the space they take up on the page. The purpose of compiling these notes is not just to archive the lecture series given but also to make it somewhat more accessible by clarifying what we ourselves had to do as we attended the course. This does not mean any significant amount of material is added above and beyond the course content itself -- rather, more so that some `one-liners' in the original presentation are afforded just a few more in these notes.

Where not otherwise cited, facts should be supposed to have been taken from the lectures — which themselves will periodically include references or suggestions for additional reading material. The citations are included for exteriorly sourced information, and the rest at the end of the document.

%\medskip

%\fix{An aside for Dr Speissegger: throughout, there are bits of red text indicating the non-inclusion or reduction of detail from what should otherwise be in these notes. This is just to clarify where some corners were cut in the interest of time -- as I tended to take a breadth-over-depth approach to attempt to put the course material into notes, feeling that would be a more `complete' version, despite being similarly unfinished either way.}

\vspace{\baselineskip}
\begin{flushright}\noindent
McMaster University,\hfill {\it Alun  Stokes}\\
April 2022\hfill {\it MATH 712}\\
\end{flushright}


