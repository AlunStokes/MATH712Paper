%%%%%%%%%%%%%%%%%%%%% chapter.tex %%%%%%%%%%%%%%%%%%%%%%%%%%%%%%%%%
%
% sample chapter
%
% Use this file as a template for your own input.
%
%%%%%%%%%%%%%%%%%%%%%%%% Springer-Verlag %%%%%%%%%%%%%%%%%%%%%%%%%%
%\motto{Use the template \emph{chapter.tex} to style the various elements of your chapter content.}
\subsection{In Absence of Aproposia: A Brief Foray}
\label{chap:apropos} % Always give a unique label
% use \sectionmark{}
% to alter or adjust the chapter heading in the running head
\sectionmark{Some Introductions}

Here, we mention a point or two (just one actually) that would otherwise go overlooked. Necessity is entirely eschewed, and this section can be safely skipped without any loss in the understanding of the course as it was intended to be presented. One is perhaps encouraged to leave. This short section exists only for the interested, dedicated, obliged, or otherwise neurotic reader. In short, this encapsulates that which has no other place thus far, nor is deserving too much attention.

\bigskip
\centerline{\rule{0.3333\linewidth}{.2pt}}
\smallskip

\subsubsection{Let's Get it Out of the Way}
\label{sec:aproposia-is-not-a-word}
\noindent One of the first thoughts the critical or unassuming reader -- which is likely all of you -- had when reading this section title was whether \emph{that} word was even a word \emph{at all}. If you are at all a curious person, searching the internet or your favourite etymological website or reference text for the word `aproposia,' you would have failed in your task -- unless the task you set out was to assure yourself that it is, in fact, \emph{not} a common or even extant word. Were `apropos' of Latin root, then perhaps this linguistic abomination may make more sense, but considering the French origin of `apropos,' no such logic applies. Nonetheless, there is little the reader can do about this choice of wording (save for one particular professor). And since we felt its hopefully clear meaning and appealing sound a nice title considering the nature of the topic, you should then be glad at all, dear reader, that Latin is no longer the language of science you would be expected to learn to have what would be considered `valid' opinions on its workings. If anything, feel free to use it as a word to confound and confuse your dear friends and colleagues. While we will let you get away with calling everything and anything `normal,' we insist you take `aproposia' as valid and unilaterally call that a fair trade.

\subsubsection{Onto the Miscellany}

In which we again abuse language, in the sense that there is only one fact of miscellaneous variety. Sometimes, certain benign ridiculousness must be allowed to amuse your readers or even just oneself. We hope you are amused. What we do, after all, is exciting not in its protracted execution but in the few moments where it all comes together. It's how we prevent the onset of early insanity when getting into the thick of these sorts of ideas.

\begin{remark}
  Recall that an \om structure requires all \defnb sets \textbf{with parameters} to be given by finite unions of points and open intervals (as defined by the model). If we only assume this for sets \defnb \textbf{without} parameters, the resulting theory is \emph{legitimately} and provably weaker than what we get with \omy. This was in passing mentioned to be potentially true earlier on, but in one of the question and answer sessions held for this course, it was pointed out that it is \emph{in fact} true by a gentleman with the given name Chris. In a moment, we will be referring to him by his family name, Miller. This awkward wording will be clear in just a moment.

  An easy (in the sense of being counter-exemplary) way to show this is due to Dolich, Miller, and our old friend Steinhorn \cite{dolich_structures_2009}. This can be expressed (though perhaps not proven, as the length of their paper implies) quite compactly by constructing the model.
  $$
    \M = (\R, <, V)
  $$
  for $V$ the Vitali set (defined by the Vitali \emph{relation}):
  $$
    V = \Set{(x, y) \in \R^2}{x - y \in \Q}.
  $$
  The only $\emptyset$-\defnb subsets of this are $\emptyset$ itself and $\R$ -- and so this fits the definition of being $\emptyset$ \om, but given any defined parameter, we end up with the rationals \defnb; clearly, this is \emph{not} \om. If you recall the short mention made earlier, it may interest the reader to note that this is \emph{weakly} \om.

  \begin{exercise}
    As an exercise of \emph{this} author to the reader, attempt to prove that the above expansion admits QE. \textit{Hint: You would be well-served to first read section \ref{chap:defn_dimensionality} on dimensionality and definable closures.}
  \end{exercise}
\end{remark}

Putting an end to this brief foray into intrigue with a splash of ridiculousness, we now move back to a consequential idea once one defines \cds: connectedness.

