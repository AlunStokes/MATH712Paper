%%%%%%%%%%%%%%%%%%%%% chapter.tex %%%%%%%%%%%%%%%%%%%%%%%%%%%%%%%%%
%
% sample chapter
%
% Use this file as a template for your own input.
%
%%%%%%%%%%%%%%%%%%%%%%%% Springer-Verlag %%%%%%%%%%%%%%%%%%%%%%%%%%
%\motto{Use the template \emph{chapter.tex} to style the various elements of your chapter content.}
\chapter{Finally: A New Beginning}
\chaptermark{Starting Again}
\label{starting_pw} % Always give a unique label
% use \chaptermark{}
% to alter or adjust the chapter heading in the running head
%\chaptermark{Some Introductions}

\abstract*{One can imagine the collective sighs of all that read through the previous section to get to this, the \emph{exciting} part of the module, almost-deafening. It is here we will first start our proper discussion of the \pwt, along with some bits and bobs as to how we will go about actually proving it, and some previous of what it entails. There are two broad \emph{bits} to the proof we will discuss -- although each is deserving of a chapter (or perhaps even more) on its own, and so there proofs are outsourced to later to come parts. For now, we just start by dipping our toes into what we've been building to through the previous entire first part.}

\abstract{One can imagine the collective sighs of all that read through the previous section to get to this, the \emph{exciting} part of the module, almost-deafening. It is here we will first start our proper discussion of the \pwt, along with some bits and bobs as to how we will go about actually proving it, and some previous of what it entails. There are two broad \emph{bits} to the proof we will discuss -- although each is deserving of a chapter (or perhaps even more) on its own, and so there proofs are outsourced to later to come parts. For now, we just start by dipping our toes into what we've been building to through the previous entire first part.}

As mentioned before -- but bears repeating -- our approach to proving \pwt (in our special little case) broadly follows that of Bhardwaj and van den Dries \cite{bhardwaj_pilawilkie_2022}; a recent paper that only came out this year (should you be reading this in 2022). Their major contribution includes the use of \sacds to simplify several arguments of the original proof by Pila and Wilkie. They also provide a full treatment of a recent variant \cite{binyamini_yomdingromov_2021} of the (classical) Yomdin-Gromov theorem \cite{gromov_entropy_1987} (again, assuming your readership takes places not too soon after these words reach the page) in 2021, and a result of Bombieri and Pila \cite{bombieri_number_1989}, which is also made use of.

\centerline{\rule{0.6667\linewidth}{.2pt}}
\bigskip

Let us now begin with a recollection of our setting. We are no longer going to speak to the generality of any ordered group, and instead an \om expansion $\Rtildefull$ of the ordered field of $\R$. Ultimately, the \pwt is about rational points on definable sets -- so taking definable sets over this expansion (supposing points have $n$ dimensions) and intersecting with $\Q^n$, then \pw tells us something about the number of such points we can expect; in short, how \emph{few} of them we should expect up to a certain \emph{height}. This concept was briefly touched upon early in the first part, but just to recap, we can define certain functions that characterize the height of a number (\`a la multiplicative height for rationals), and then bound the size of this set from above as a function of that height parameter. Without use of a height, there could of course be infinitely-many points, so this concept is crucial (and broadly the \emph{point} of the theorem).

Let's begin by again defining the multiplicative height on $\Q$.
\begin{definition}[Multiplicative Height on $\Q$]
  Suppose some $q = \sfrac{a}{b}$ for $a, b$ coprime, $a, b \in \Z$, and $b \neq 0$. Then, the height function is denoted by $\funcdom{H}{\Q}{\Zgeq{0}}$, and defined
    \begin{align*}
      H(q) = \max{ \left\{ \ \abs{a}, \ \abs{b} \ \right\} }
    \end{align*}
    for $\abs{\cdot}$ simply the absolute value on $\Q$.
  \label{defn:Q_height}
\end{definition}

\begin{svgraybox}
  For reasons of preserving syntax used consistently (rather than some off-handed syntax we have been replacing) throughout this part of the course, we overload the use of $H$ as both a function on the ration numbers that maps some $q \in \Q^n$ to its height, \emph{as well as} the integral height itself. You will often find sentences of the form `fixing some $H \geq 1$, we take $X(S, H)$ to be the set $\{ s \in S \ \colon \ H(s) \leq H \}$'. While a bit confusing, especially at first blush, this choice for syntactic overloading was made due to its persistence throughout Dr Jones' lectures.
\end{svgraybox}


\begin{corollary}
  Fixing some height, $H$, there are only finitely many $q \in \Q$ with height $H$.
\end{corollary}
\begin{proof}
  This should be clear.
\end{proof}

As usual, we now address the multidimensional-case
\begin{definition}[Multiplicative Height on $\Q^n$]
  Suppose some $q = (\vonevm{q}{n}) \in \Q^n$ for $n \in \Zgeq{1}$, all in lowest terms as before, then we define the height function to be the coordinate-wise maximum:
    \begin{align*}
      H(q) = \max{ \left\{ \ H(q_j) \colon 1 \leq j \leq n \ \right\} }.
    \end{align*}
  \label{defn:Qn_height}
\end{definition}

With that in mind, we have the following definition.
\begin{definition}
  Let $\XRn[X]{n}$
\end{definition}
