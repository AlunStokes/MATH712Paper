%%%%%%%%%%%%%%%%%%%%% chapter.tex %%%%%%%%%%%%%%%%%%%%%%%%%%%%%%%%%
%
% sample chapter
%
% Use this file as a template for your own input.
%
%%%%%%%%%%%%%%%%%%%%%%%% Springer-Verlag %%%%%%%%%%%%%%%%%%%%%%%%%%
%\motto{Use the template \emph{chapter.tex} to style the various elements of your chapter content.}
\chapter{In Absence of Aproposia}
\label{chap:apropos} % Always give a unique label
% use \chaptermark{}
% to alter or adjust the chapter heading in the running head
%\chaptermark{Some Introductions}

\abstract*{Here we mention a point or two that would otherwise go overlooked. Necessity is entirely eschewed, and this chapter can be safely skipped without any loss in understanding of the course as it was intended to be presented. This short chapter exists only for the interested, dedicated, obliged, or otherwise neurotic reader. In short, this encapsulates that which has no other place.}

\abstract{Here we mention a point or two that would otherwise go overlooked. Necessity is entirely eschewed, and this chapter can be safely skipped without any loss in understanding of the course as it was intended to be presented. This short chapter exists only for the interested, dedicated, obliged, or otherwise neurotic reader. In short, this encapsulates that which has no other place.}

\section{Let's Just Get it Out of the Way}
\label{sec:aproposia-is-not-a-word}
\noindent One of the first thoughts the critical or unassuming reader -- which is likely all of you -- had when reading this chapter title was whether that word was really even a word \emph{at all}. If you are at all a curious person, searching the internet or your favourite etymological website or reference text for the word `aproposia', you would have failed in your task -- unless the task you set out was to assure yourself that it is in fact \emph{not} a common or even extant word. Were `apropos' of Latin root, then perhaps this linguistic abomination may make more sense, but considering the French origin of `apropos', no such logic applies. Nonetheless, there is little the reader can do about this choice of wording (save for one particular professor), and since we felt its hopefully clear meaning and appealing sound appropriate to the nature of the topic, then you should be glad at all, dear reader, that Latin is no longer the language of science you would be expected to learn in order to have what would be considered `valid' opinions on its workings. We will let you call everything and anything `normal', and we will take `aproposia' as valid, and unilaterally call that a fair trade.

\section{Onto the Miscellany}

In which we again abuse language, in the sense that there is only one fact of miscellaneous variety. Sometimes, to amuse your readers of even just one's self, certain benign ridiculousness must be allowed. It's how we prevent the onset of early insanity, when getting into the thick of these sorts of ideas.

\begin{remark}
  Recall that an \om structure requires all \defnb sets \emph{with parameters} to be given by finite unions of points and open intervals (as defined by the model). In fact, if we only assume this for sets \defnb \emph{without} parameters, the resulting theory is \emph{legitimately} and provably weaker than what we get with \omy. This was in passing mentioned to be potentially true earlier on, but in one of the question and answer sessions held for this course, it was pointed out that it is \emph{in fact} true by a gentleman with the given name Chris. In a moment, we will be referring to him by his family name, Miller.

  An easy (in the sense of being counter-exemplary) way to show this is due to Dolich, Miller, and our old friend Steinhorn \cite{dolich_structures_2009}. This can be expressed (though perhaps not proven, as the length of their paper implies) quite compactly, by constructing the model
  $$
    \M = (\R, <, V)
  $$
  for $V$ the Vitali set (defined by the Vitali \emph{relation}):
  $$
    V = \Set{(x, y) \in \R^2}{x - y \in \Q}.
  $$
  The only $\emptyset$-\defnb subsets of this are $\emptyset$ itself and $\R$ -- and so this fits the definition of being $\emptyset$ \om, but given any defined parameter, we end up with the rationals \defnb; clearly, this is \emph{not} \om. If you'll recall the short mention made earlier, it may interest the reader to note that this is \emph{weakly} \om.

  \begin{exercise}
    As an exercise of \emph{this} author to the reader, attempt to prove that the above expansion actually admits QE. For a proof of this fact, should one be feeling lazy, see \cite{dolich_structures_2009}.
  \end{exercise}
\end{remark}

Putting an end to this brief foray into intrigue with a splash of ridiculousness, we now move back on to a consequential idea once one defines \cds: connectedness.

