%%%%%%%%%%%%%%%%%%%%%part.tex%%%%%%%%%%%%%%%%%%%%%%%%%%%%%%%%%%
% 
% sample part title
%
% Use this file as a template for your own input.
%
%%%%%%%%%%%%%%%%%%%%%%%% Springer %%%%%%%%%%%%%%%%%%%%%%%%%%

\begin{partbacktext}
\part{The \pwT}
\noindent Now we start properly on the promise we set out to make good on from the start: an honest proof of the \pwt. The previous part, in no means brief and right to the point, was hopefully either a good refresher or first introduction to the necessary machinery we are about to put to use to prove this eponymous theorem. As mentioned before, there are two main parts to this proof: \fix{thing 1, and thing 2}. We now have everything we need to put those together in full gruesome (meant in a good way, of course) detail -- albeit using more modern proofs than originally used by Pila and Wilkie.

We use \fix{idk the source on the 1 for thing 1} first, using \fix{blank or something or the other instead of blank} for \fix{the first part}. Then, we use the similarly recent proof of the result of \fix{fuck me i forget but it was 2021 and was something-Gromov} to get \fix{Resulto numero 2}. From there, \pw itself is far from a stretch, and we will soon have built it up in (almost) all its full spectacle.

\end{partbacktext}
