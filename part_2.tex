%%%%%%%%%%%%%%%%%%%%%part.tex%%%%%%%%%%%%%%%%%%%%%%%%%%%%%%%%%%
% 
% sample part title
%
% Use this file as a template for your input.
%
%%%%%%%%%%%%%%%%%%%%%%%% Springer %%%%%%%%%%%%%%%%%%%%%%%%%%

\begin{partbacktext}
\part{The \pwT}
\noindent Now, we start properly on the promise we set out to make good on our promise from the outset of these notes: an honest proof of the \pwt. The previous part, by no means brief and right to the point, was hopefully either a good refresher or first introduction to the necessary machinery we are about to put to use to prove this eponymous theorem. As mentioned before, there are two main parts to this proof: parameterization (the \om bit) and the diophantine part (the number theory bit). We now have everything we need to put those together in full gruesome (meant in a good way, of course) detail -- albeit using more modern proofs than originally used by Pila and Wilkie. In particular, we will be providing more detail in the proofs (and full proofs themselves) in the proof of the parameterization result, as it skews more relevant to this course. While no means undetailed, the diophantine part discussion will be moreso an overview and statement of the important and necessary results and will often lack the proof required for proper rigour.

We broadly follow the recent paper of Bhardwaj and van den Dries \cite{bhardwaj_pilawilkie_2022} first but instead elect to take a few shortcuts or liberties when thought to be appropriate. Then, we use the similarly recent new proof (depending on when you're reading this) of the Yomdin-Gromov theorem by Binyamini and Novikov \cite{binyamini_yomdingromov_2021} to get our desired result. From there, \pw itself is far from a stretch, and we will soon have built it up in (almost) all its full spectacle.

\end{partbacktext}
