%%%%%%%%%%%%%%%%%%%%%part.tex%%%%%%%%%%%%%%%%%%%%%%%%%%%%%%%%%%
% 
% sample part title
%
% Use this file as a template for your own input.
%
%%%%%%%%%%%%%%%%%%%%%%%% Springer %%%%%%%%%%%%%%%%%%%%%%%%%%

\setcounter{section}{1}
\setcounter{chapter}{0}

\chapter[\Omy \& the \pwT]{\Omy \& the \pwT: Module I in Graduate Course on O-minimality and Applications}
\noindent This introductory part will focus primarily on setting up the setting in which we find ourselves working for most of the course. Highlights include ``\emph{what even is \omy?}'' and ``\emph{but how does that imply \pw?}'' -- and perhaps a favourite of mine: ``\emph{what is the \pwt?}''. While the latter two are answered in much more detail, later on, this part will take us at a reasonable pace through some of the significant results we'll find ourselves needing in a bit. In particular, we prove the monotonicity theorem (MT), define cell decompositions (CD) and much later on \emph{smooth} cell decompositions -- proving the Cell Decomposition theorem itself along the way. Later on, we discuss dimensionality and the agreement between the definable (or geometric) and algebraic (or model-theoretic) notions of the quantity -- something that will come to be quite useful to us later on. We finally end off with that promised discussion of \scds, although in comparison to other sections, it really is more a series of results and sketches (if that) of how one might prove these things, as they broadly build on either proof structures we will have seen before, or shouldn't be too hard in conjunction with a dusting off of everyone's favourite: \textit{Calculus: Early Transcendentals} -- of which I'm sure we all have a copy somewhere.

The prepared reader should find themselves acquainted with preliminary ideas in mathematical logic (nothing further than one would find in an undergraduate class on the subject) and the basics of field theory. Again, nothing further than an undergraduate would necessarily be expected to have encountered.

A reader should leave this section feeling themselves reasonably well-acquainted with some of the tools they might see themselves using in general sorts of proofs about \om structures and how they may go about proving or disproving something to be \om. Some preliminary notions of how this all fits into counting points of bounded height on curves may be coming to surface by the end of this part, but the novice reader would be well-forgiven were that not the case. They should, however, feel comfortable identifying what is `clearly' a definable set and be able to do some rudimentary reasoning on how we can use the finiteness and definability of one of a pair of complementary sets to say something about the definability of the other. The idea of cells and cell decompositions should be reasonably understood (at least \emph{intuitively} if not in full technical detail). The reader should have a map of sorts in their mind that sequentializes and connects the discussed matter in a reasonable and meaningful way -- as these preliminary ideas form the basis for the larger \lemmas and theorems to come. By the end of this part, the well-established reader should find the proof presented rather intuitive and even (hopefully) find that they belabour the points they make \emph{too} much for how clearly obvious they are.

%%%%%%%%%%%%%%%%%%%%%part.tex%%%%%%%%%%%%%%%%%%%%%%%%%%%%%%%%%%
% 
% sample part title
%
% Use this file as a template for your own input.
%
%%%%%%%%%%%%%%%%%%%%%%%% Springer %%%%%%%%%%%%%%%%%%%%%%%%%%

\begin{partbacktext}
\part{\Omy and Necessary Concepts}
\noindent This introductory part will focus primarily on setting up the setting in which we find ourselves working for most of the course. Highlights include ``\emph{what even is \omy?}'' and ``\emph{but how does that imply \pw?}'' -- and perhaps a favourite of mine: ``\emph{what is the \pwt?}''. While the latter two are answered in much more detail, later on, this part will take us quickly through some of the significant results we'll find ourselves needing in a bit. In particular, we prove the monotonicity theorem, define cell decompositions (CD), then prove the cell decomposition theorem, and top it off with a discussion of dimensionality and the agreement between the geometric and algebraic notions of the quantity.

The prepared reader should find themselves acquainted with preliminary ideas in mathematical logic (nothing further than one would find in an undergraduate class on the subject) and the basics of field theory. Again, nothing further than an undergraduate would necessarily be expected to have encountered.

A reader should leave this section feeling themselves reasonably well-acquainted with some of the tools they might see themselves using in general sorts of proofs about \om structures and how they may go about proving or disproving something to be \om. Some preliminary notions of how this all fits into counting points of bounded height on curves may be coming to surface by the end of this part, but the novice reader would be well-forgiven were that not the case. They should, however, feel comfortable identifying what is `clearly' a definable set and be able to do some rudimentary reasoning on how we can use the finiteness and definability of one of a pair of complementary sets to say something about the definability of the other.

\end{partbacktext}

  %%%%%%%%%%%%%%%%%%%%% chapter.tex %%%%%%%%%%%%%%%%%%%%%%%%%%%%%%%%%
%
% sample chapter
%
% Use this file as a template for your own input.
%
%%%%%%%%%%%%%%%%%%%%%%%% Springer-Verlag %%%%%%%%%%%%%%%%%%%%%%%%%%
%\motto{Use the template \emph{chapter.tex} to style the various elements of your chapter content.}
\subsection{Mapping the Landscape and Telegraphing our Journey}
\sectionmark{Let's Orient Ourselves}
\label{orientation} % Always give a unique label
% use \sectionmark{}
% to alter or adjust the chapter heading in the running head
%\sectionmark{Some Introductions}

How one starts with the Tarskian ideas introduced in their first course in mathematical logic and ends up studying \omy to the end of proving the \pwt is perhaps and unsurprisingly eminently unclear to any who do not already know the methodology. So we don't become too disenfranchised and uninspired as we work our way through the logical results to the end of a number-theoretic result. We lay out a brief map of what is to come and how each part logically follows from its antecedent. This is not perhaps the most interesting course for one interested solely in number theory or simply mathematical logic, but where these two unlikely friends collide creates something wondrous and beautiful. We then go on with just a few words in preparation for what is to come; some definitions, expectations of the experience (or lack thereof) on the reader's part, and a general outline are given. The overly excited reader may feel free to skip right onto Section \ref{setting}, but this section serves as just a bit of an \textit{amuse-bouche} for those not so ready to jump right in.

\bigskip
\footnote{This will come to make sense later and will seem apropos of nothing right now, but please, take no comfort in these short, horizontal lines other than the joy one may find in delineating the would-be abstract from the content proper.}
\centerline{\rule{0.3333\linewidth}{.2pt}}
\smallskip

\subsubsection{And \pw is?}
\noindent Perhaps the best and most prudent question to be asking one's self currently, if for no other reason than determining the worth of their time in reading this whole affair, is what the statement of the \pwt \emph{actually} is? And what, supposing the reader knows the context in which we define \omy, could that have to do with a \ntc result like the \pwt? Well, dear reader, we hope in this brief first section to enlighten you to the big ideas upon which we will ruminate for the remainder of this course and provide a coarse outline of how these come to build on one another in order to start from the relatively basic, to the phenomenal. To live up to this section's name, however, we now state the \pwt -- first informally and then as we will come to prove it. In all cases, however, note that we are speaking to \om expansions of the \emph{real} field, and we will not be covering the theorem and all that leads up to it in full generality.

\begin{theorem}[\pwt (Informal)]
  \label{thm:pwt_informal}
  Let $\tilde{\R}$ an \om expansion of $(\R, <)$. Then \textbf{transcendental} \defnb sets have very \textbf{few} rational points.
\end{theorem}

It is easily understandable, seems reasonable, and (maybe?) doable without too much fuss, doesn't it seem? Here now is the formal statement, which requires the following crash course in notation; we denote by $H$ the usual rational multiplicative height function -- but it also doubles, when not used as a function, as an upper-bound on rational numbers we are interested in (I didn't decide the notation). When taking vectorial input, the height is given by the element-wise maximum. Suppose $X \subseteq \R^n$. Denote $X(\Q) \coloneqq X \cap \Q$ and further $\XQH \subseteq X(\Q)$ with element height as given by the function, $H$, and bounded by the constant, $H$. Finally, the superscripts $\emph{tr}$ and $\emph{alg}$ refer to the transcendental and algebraic parts of the set they sit atop (note, of course, that necessarily we have $X \setminus \Xtr = \Xalg$). What we then want, and what \pw gives us, are good bounds on $\card{\XtrQH}$.

\begin{theorem}[\pwt (Formal)]
  \label{thm:pwt_formal}
  Let $X \subseteq \R^n$ \defnb in $\tilde{\R}$ an \om expansion of $(\R, <)$. Then for any $\epsilon$ there exists some $c$ such that for all $H$,
  \begin{align*}
    \card{\XtrQH} \leq c \cdot H^{\epsilon},
  \end{align*}
  which is a very fine thing indeed -- that last bit being an editorial note, and not (necessarily) part of the honest theorem.
\end{theorem}

\subsubsection{There but Not Back Again}

How we shall prove this is not direct by most meanings or usages of the term and is not going to follow the proof originally given by the eponyms for the theorem. Rather, we will more so be following a later, arguably nicer proof given just this year (if you are reading this in 2022) by Bhardwaj and van den Dries \cite{bhardwaj_pilawilkie_2022}, that makes clever use of semialgebraic cell decomposition for a large part of the original proof, and elsewhere use a proof of the Yomdin-Gromov theorem given by Binyamini and Novikov \cite{binyamini_yomdingromov_2021} also very recently.

The broad strokes are as follows, with some bits left out in the majority, just definitional. This encompasses most of Part I of these notes -- as this is, after all, a graduate course on this topic. We start from the very basics with \defnbly, \omy, the Finiteness Theorem in 2 variables (which goes on to be used more broadly), and then \cds. With basics in place, we define definable choice and curve selection and then go on to show that \defnb and algebraic dimension are in agreement. Closer to the majority of the matter, we go on to prove the theorems mentioned above that, when all put together, allow us to produce a proof of \pw. Some parts are taken for granted (for example, assuming uniform finiteness without proof), and in general, we treat only expansions of the real field rather than a more generic structure -- and this allows us to use a few tricks such as the Baire Category theorem at one point. On the whole, however, this set of notes constitutes almost all but some of the trickier details in a fully self-contained proof of the \pw theorem, and the reader is directed to where any gaps may be addressed when present.


\sectionmark{A Small Bit Before we Begin}
\subsubsection{Before the Lectures Proper}

We feel somewhat compelled to address an aspect of this topic that we felt was slightly neglected in the curriculum. Had one seen the list of attendees to these lectures, the reason for skipping over such `trivialities' as we are about to point out briefly is clear — with several attendees being former students of Pila or Wilkie themselves. Still, for the \emph{not-even-amateur} logician, the following certainly bears some explicit mention.

In the absence of the results we come to find, \omy may appear a relatively unmotivated idea to study. Of course, as the pure mathematicians we are (or hope one day to be), why \emph{shouldn't} the mere concept of further understanding be enough to compel our interest? Still, the progression from introductory mathematical logic and the importance and usefulness of quantifier elimination (QE) to \omy is one that was unapparent to this author until being noted elsewhere. We don't claim this to be a failure of the course so much as a failure in personal preparation, but should the reader find themselves similarly underprepared, then they will find themselves thankful for this little pretext.

To keep things brief, we will say just this: in a predicate logical system, we are interested in quantifier elimination. The result of the elimination of quantifiers is essentially the answer to the question a quantified statement asks. Perhaps the most famous example of this is the existence of real roots of quadratic equations. We ask the quantified `question':
\begin{align}
  \exists x \in \R . (a \cdot x^2 + b \cdot x + c = 0 \wedge a \neq 0)
\end{align}
— that is, does there exist such an $x$? The quantifier eliminated equivalent form is
\begin{align}
  b^2 - 4 \cdot a \cdot c \geq 0 \wedge a \neq 0,
\end{align}
the first half of which should be recognized as the quadratic discriminant (and the second just to ensure non-degeneracy). Here, quantifier elimination gives the exact and deterministic characterization of the answer to the quantified statement -- and it is this property that motivates its study. We trust at this point that the motivation has been sufficiently belaboured.

It is known that first-order theories with QE (that is, decidability for the theory can be reduced to the question of satisfaction of quantifier-free sentence in the theory) are model complete. In the interest of not straying too far, we leave it to the reader to believe or convince themselves that this is a desirable property. While this is not the focus of how we define \omy in this course, a structure is indeed \om exactly if every formula given by no more than one free variable and some subset of $M$-parameters is equivalent to a quantifier-free formula defined only by these parameters, and the ordering on the structure \cite{marker_model_2002}. Thus, for anyone finding themselves perhaps unconvinced upfront of the merit of some of the ideas explored here (outside the Pila-Wilkie Theorem), we hope this motivates the sequence we are about to take on. And for everyone else, we hope that this section did not bore you too thoroughly.


\subsubsection{Preliminary Definitions}
\label{sec:prelim-defns}
Throughout, we will be working with models $ \M = \Mlt$ of the theory of dense linear orders (DLO) \emph{without endpoints}. For now, $M$ will be fixed, but we will look at some specific instances later on. Perhaps then, one of the most important definitions, to begin with, is that of \emph{definability}.

%\begin{definition}[Definability of sets (without parameters) \cite{marker_basic_2002}]
\begin{definition}[Definability of sets (without parameters)]
For $n \in \N$, we say a set $A \subseteq M^n$ is \emph{definable without parameters} if there exists some formula in our language, $ \phi$, satisfied exactly by the elements of $A$.
\end{definition}

%\begin{definition}[Definability of sets \cite{marker_basic_2002}]
\begin{definition}[Definability of sets]
For $X \subseteq M$ and $n \in \N$, then we say a set $A \subseteq M^n$ is \emph{definable with parameters from $X$} if there exists some formula in our language, $ \phi$, and elements $b_1, \hdots, b_m$, such that $ \phi$ is satisfied exactly by the elements of $A$ along with the parameters in $X$.
\end{definition}

Notice then that definability without a parameter is simply the case of definability with parameters coming from the empty set. These definitions immediately and naturally induce the idea of definable functions and definable points. In particular, a function is definable in parameters if its graph is definable by those same parameters in $\M = \Mlt$. Similarly, an element $a$ is definable in $\M$ (with parameters) if the singleton $\{a\}$ is definable in $ \M$ by those same parameters. This isn't something we will need to consider too extensively.

When introduced to a novel space, we are often interested in what its open intervals look like. We have the following characterization:

\begin{definition}[Open Interval]
  A set, $A \subset M$ is an open interval in $\Mlt$ if $A$ is of one of the following forms:
  \begin{itemize}
    \item $(a, b)$ with $a < b \in M$
    \item $(- \infty, a)$ with $a \in M$
    \item $(a, \infty)$ with $a \in M$
  \end{itemize}
\end{definition}

We say further that intervals of the first type — that is, those having finite bounds — are \emph{bounded}. Easy to miss but important to note is that the endpoints must sit inside our domain. So, for example, in $(\Q, <)$, the set $(-5, \sqrt{7})$ is \emph{not} an open interval.

We imbue $M$ with the order topology and $M^n$ with the product topology. We then define what it means to be an \om expansion.

\begin{definition}[\Om expansion]
  Taking $\mathcal{M} = \Mltp$ an expansion of $\Mlt$, we say $\mathcal{M}$ is \om if every definable (with parameters) subset of $M$ is given by a finite union of open intervals and points.
\end{definition}

\begin{svgraybox}
  If we weaken the above and ask only for \emph{convex} sets (which are a superset of our open intervals) in place of open intervals, then the above would define \emph{weak \omy} — but that won't be a topic of discussion here.
\end{svgraybox}

For the etymologically inclined, it is noted that the `o' in \om comes from the shortening of `order-minimality'. For more information on the history and development of the idea of \omy, one may reference
\textit{Tame Topology \& \Om Structures} \cite{dries_tame_1998} or \textit{Definable Sets in Ordered Structures I} \cite{pillay_definable_1986} and \textit{II} \cite{knight_definable_1986}.


Some (arguably) simple examples of \om structures are given by expansions of the real field. Consider, for example, $\overline{\R} = (\R, <, +, -, \cdot, 0, 1)$ and the further expansion $\R_{\textrm{exp}} = (\overline{\R}, \exp)$, both of which are \om. Mind not to mistake the use of `simplicity' as an indication that these are trivial or did not require particular and considerable consideration — rather, just that they have a relatively simple-seeming form. We fix $\M$ an \om structure and move on to our first theorem for now and in the future.

  %%%%%%%%%%%%%%%%%%%%% chapter.tex %%%%%%%%%%%%%%%%%%%%%%%%%%%%%%%%%
%
% sample chapter
%
% Use this file as a template for your own input.
%
%%%%%%%%%%%%%%%%%%%%%%%% Springer-Verlag %%%%%%%%%%%%%%%%%%%%%%%%%%
%\motto{Use the template \emph{chapter.tex} to style the various elements of your chapter content.}
\subsection{Setting it all up}
\label{setting} % Always give a unique label
% use \sectionmark{}
% to alter or adjust the chapter heading in the running head
\sectionmark{Some Introductions}

We now begin properly with a from-the-basics definition of the objects at play: field expansions, monotonicity, cells and decompositions into them, and similarly fundamental ideas are each defined and contextualized. Note that we will not be discussing topological definitions in general. That is to say, the reader is assumed to be familiar with the basic point-set topology, and the ordinary sorts of topologies we see cropping up (e.g. order, product) -- not that topological ideas won't be discussed. Basic knowledge of mathematical logic is also assumed; first-order languages (FOL), $\Lstrs$, relations, and satisfiability are all presumed familiarities. With definability now a part of our tool-set, we start by proving a few theorems fundamental to results later in this course.

\bigskip
\centerline{\rule{0.3333\linewidth}{.2pt}}
\smallskip

\subsubsection{On Monotonicity}
\sectionmark{Monotonicity}
What constitutes a `nice' property of a function is generally non-contentious; injectivity and surjectivity are often useful -- together even more so -- and it would be the odd mathematician to turn their nose up at a function being bounded, supposing they weren't chasing a nasty counterexample or engaging in some other such endeavour. At present, we will focus on the property of \emph{monotonicity} and when we can determine a definable function to be monotonic in the context of open intervals. The following was proved in \cite{pillay_definable_1986} by Pillay and Steinhorn:

\begin{theorem}[The \MT]
	\label{thm:monotonicity}
  Suppose $f \colon I \to M$ is a definable function for $I \subset M$ an open interval. Then there exist $a_1, \hdots, a_k \in I$ such that on each adjacent interval, $(a_j, a_{j+1})$ (where $I = (a_0, a_{k+1})$) $f$ is either constant, or strictly monotonic and continuous. Further, if $f$ is definable over some $A \subseteq M$, then so too are $a_1, \hdots, a_{k}$ definable over $A$.
\end{theorem}

Hence, we will refer to this simply as the \Mt, abbreviated by MT. It is perhaps not immediately apparent why this should be true, or even that we should be interested that it is. The answer to the second point is that this piece-wise continuity and monotonicity of definable functions is a relatively rigid condition, and this (not just here but for structures in general) allows us to say a good bit about them. Observe that if we have some $X \subseteq M$ definable and infinite, then X must contain some open interval. This should be relatively intuitive, even if a proof doesn't come to you immediately, given what we've covered thus far. As for why the \Mt holds, we show this by piecing together three \lemmas that should make the picture a bit more clear. Throughout, take $J \subset I$ as an open interval. To not get bogged down in the minutiae of their proofs as we go through — not that they are particularly challenging — but in any case, we will state all three and then prove them sequentially.

\begin{lemma}
\label{lemma:monotonic-1}
  There is an open interval, $\Jp \subseteq J$, on which $f$ is constant or injective.
\end{lemma}

\begin{lemma}
\label{lemma:monotonic-2}
  If $f$ is injective on $J$, then there is an open interval, $\Jp \subseteq J$ on which $f$ is strictly monotonic.
\end{lemma}

and finally,

\begin{lemma}
\label{lemma:monotonic-3}
  If $f$ is a strictly monotonic function on $J$, then there exists some open interval $ \Jp \subseteq J$ on which $J$ is continuous.
\end{lemma}

Taking these \lemmas for granted, it is not terribly difficult to see how the \Mt falls out. The fun then is proving these three facts — which is nice, as they are not terribly complicated.

We start where any sensible person would.

\begin{proof}[of Lemma \ref{lemma:monotonic-1}]
  Suppose there is some $y \in M$ such that its preimage under $f$ intersected with $J$ is infinite. This necessarily implies the existence of $ \Jp \subseteq J$ an open interval on which $f$ takes constant value, and so we can assume for any $y \in M$ that we have $f^{-1}(y) \cap J$ is finite. Then, we must have $f(J)$ infinite, and so contains interior with subset $ (a, b) $, for $a < b$. Taking
  \begin{align*}
    q \colon (a, b) &\to J \\
    q \colon y &\mapsto \min{\Set{x \in J}{f(x) = y}},
  \end{align*}
  we get q injective — and so this is an open interval $ \Jp \subseteq q((a, b))$ on which $f$ is injective.
\end{proof}


\begin{proof}[of Lemma \ref{lemma:monotonic-2}]
  Suppose otherwise. Then, for every open sub-interval, $\pri{J} \subseteq J$, we have $f$ \emph{not} strictly monotonic -- and so on every such interval, there are distinct $a, b \in \pri{J}$ with $f(a) = f(b)$. To remedy this, we shrink the interval to the largest such interval not containing $b$ -- but then since this property holds for $all$ such open intervals, there now exists some $c$ with $f(a) = f(c)$. Continuing on as such, we see the interval must be the singleton set, which is not open, and a contradiction ensues. 
\end{proof}


\begin{proof}[of Lemma \ref{lemma:monotonic-3}]
  This we can get quite quickly. Suppose such a strictly monotone function exists on $J$. Clearly, $f$ cannot be constant (else monotonicity would be non-strict), and so by \omy of $f$, we get that the image of $J$ under $f$ contains some open interval, $ \Jp \subseteq \image{f}$, on which we have preimage a sub-interval of $J$. We get monotonicity on this interval by Lemma \ref{lemma:monotonic-1} and non-constancy (and thus monotonicity) of $f$; this must be a bijection (either order-preserving or reversing, but bijective either way), and so we are finished.
\end{proof}


\begin{proof}[of \Mt]
  We now combine these three \lemmas to get our result. Take $A$ the set of all $x \in I$ (coming from our original theorem statement) such that $f$ is both continuous and strictly monotone at $x$. We know that taking the restriction of $f$ to some open sub-interval on which $f$ is defined maintains both continuity and monotonicity by \Lemmas 2 and 3 — and so taking the set difference of $A$ from $I$, the original open interval, we cannot have \emph{any} open intervals. There are then thus only finitely many points, and the theorem follows.
\end{proof}

\begin{svgraybox}
	Take note that the proof provided here is \emph{not} precisely the one that was given in the lecture, but rather a bit more condensed, less roundabout method of achieving the result. The strategy is the same, however, differing only in presentation.
\end{svgraybox}

%\fix{Two Exercises Lec1 pg 4}

The following result is a special case in 2 dimensions of what is referred to as the \emph{\Ft}, abbreviated FT. We first prove this special case and then take a brief detour to talk about cell decompositions before we can address the more general theorem.

\subsubsection{The \bPFT}
\sectionmark{Special Finiteness}

\begin{theorem}[\FT in $M^2$]
	\label{thm:2finiteness}
	Suppose $A \subseteq M^2$ and that for each $x \in M$, the fibre $A_x$ above $x$ — that is, the set of $y$ with $(x, y) \in A$ — is finite. Then, there exists some $N \in \N$ such that $\card{A_x} \leq N$ for all $x \in M$
\end{theorem}

\begin{proof}[of \FT in $M^2$]
	We define a point $(a, b) \in M^2$ to be \emph{normal} if it sits in an open box, $I \times J$ satisfying
	\begin{itemize}
		\item $(I \times J) \cap A = \emptyset$
		\item $(a, b) \in A$
		\item There exists a continuous $f \colon I \to M$ such that $(I \times J) \cap A = \graph{f}$.
	\end{itemize}

	Similarly, for points with only one finite endpoint, we say some $(a, \infty)$ (resp. $(a, - \infty)$) is \emph{normal} if there exists open interval $I$ such that $a \in I$ and some $b \in M$ such that
	\begin{align*}
		(I \times (b, \infty)) \cap A = \emptyset
	\end{align*}
	and again, respectively taking $(b, - \infty)$ for the other case.

	Supposing we take the set $\Set{(a, b) \in M^2}{(a, b) \ \textrm{is normal}}$, it easily follows that this set is definable, and similarly so for the $\{\pm \infty\}$ cases. We now define functions $f_1, f_2, \hdots, f_n$ by the property that
	\begin{align*}
		\dom{f_k} = \Set{x \in M}{\card{A_x} \geq k}.
	\end{align*}
	We have the property that $f_k(x)$ is the $k$-th element of $A_x$ — and so we get the definability of each $f_k$ by the finiteness of each fibre.

	Fixing some $a \in M$ and taking $n \geq 0$ maximal such that all of $f_1, \hdots, f_n$ are defined and \emph{continuous} on an open interval around $a$. We then say that $a$ is
	\begin{itemize}
		\item \textbf{good} if  $a \notin \cl{\dom{f_{n + 1}}}$ and otherwise
		\item \textbf{bad} if $a$ \emph{is} in this closure.
	\end{itemize}
	We partition into $G = \Set{a \in M}{a \ \textrm{is good}}$ and $B = \Set{a \in M}{a \ \textrm{is bad}}$. What we will now show is that $G$ is definable — which we do by showing that for any $a \in B$, there is a minimal $b \in M \cup \{\pm \infty \}$ such that $(a, b)$ is \emph{not} normal.

	Let $a \in B$. We use the following notation for convenience:
	\begin{description}
		\item
			\begin{align*}
						\lambda(a, -) = \begin{cases}
									      \displaystyle\lim_{x \to a^{-}} f_{n + 1}(a) & \colon \ \textrm{$f_{n+1}$ defined on $(t, a)$ for some $t < a$.} \\
									      \infty & \colon \ \textrm{else}
									   \end{cases}
			\end{align*}

		\item
			\begin{align*}
						\lambda(a, 0) = \begin{cases}
									      f_{n + 1}(a) & x \in \dom{f_{n+1}} \\
									      \infty & \colon \ \textrm{else}
									   \end{cases}
			\end{align*}

		\item
			\begin{align*}
						\lambda(a, +) = \begin{cases}
									      \displaystyle\lim_{x \to a^{+}} f_{n + 1}(a) & \colon \ \textrm{$f_{n+1}$ defined on $(a, t)$ for some $a < t$.} \\
									      \infty & \colon \ \textrm{else}
									   \end{cases}
			\end{align*}
	\end{description}

	Take $\beta(a) = \min{\{ \lambda(a, -),\ \lambda(a, 0),\ \lambda(a, +) \}}$. It is not difficult to see then that $\beta(a)$ is simply the least $b \in \Minf$ such that $(a, b)$ is not normal. Were we instead to take some $a \in G$, then $(a, b)$ must \emph{always} be normal for any $b \in \Minf$. So, $B$ can be given as
	\begin{align*}
		B = \Set{a \in M}{\exists b \in \Minf \ \textrm{s.t.} \ (a, b) \ \textrm{is not normal}},
	\end{align*}
	and as such, is definable.

	If we take some $a \in G$, then $\card{A_x}$ is constant on an open interval about $a$ by definition of $G$. By showing that $B$ is finite, we get our desired result. Supposing $B$ to be \emph{infinite}, we can partition $B$ into
	\begin{align*}
		B_+ &= \Set{a \in B}{\exists y \ \textrm{s.t.} \ y > \beta(a), \ (a, y) \in A} \\
		B_- &= \Set{a \in B}{\exists y \ \textrm{s.t.} \ y < \beta(a), \ (a, y) \in A},
	\end{align*}
	both evidently definable sets. By the infinitude of $B$, so too must at least one of $B_-, B_+$ be infinite — and further, so must one of
	\begin{itemize}
		\item $B_+ \cap B_-$
		\item $B_+ \setminus B_-$
		\item $B_- \setminus B_+$
		\item $B \setminus (B_+ \cup B_-)$.
	\end{itemize}
	We can then apply the \Mt (Theorem \ref{thm:monotonicity}) to each case to reach a contradiction by showing that assuming non-finiteness, we \emph{should} be able to find a normal point with first coordinate $a$ -- contradicting the `badness' of any point in $B$. Thus, $B$ is \emph{finite}, and so there must be some finite upper bound on the cardinality of all fibres, $A_x$, and our proof is complete.
\end{proof}



\subsubsection{Cell Decompositions}
\sectionmark{Cells and Decompositions Into Them}

We start with a few definitions that should hopefully feel motivated in anticipation of the higher-dimensional analogues of what we have seen already.

\begin{definition}[Cells in $M^n$]
  For a sequence $(i_1, \hdots, i_n)$ for each $i_j \in \{0, 1\}$, we define $(i_1, \hdots, i_n)$\emph{-cells} of $M^n$ inductively as follows:
  \begin{enumerate}
    \item A $0$-cell is a point in $M$, and a $1$-cell an open interval (both in $M^1$).
    \item Supposing $\incells$ are defined for $M^n$,
	\begin{enumerate}
	  \item we define an $\izerocell$ to be a definable set given by $\graph{f}$ for $f$ a continuous, definable function on an $\incell$.
	  \item Perhaps predictably then, we define an $\ionecell$ to be a definable set of the form $(f, g)_C \coloneqq \Set{(x, y) \in C \times M}{f(x) < y < g(x)}$ for $f, g$ \cont, \defnb functions on an $\incell$, with $C \subset M^n$. Note that we may also allow $f \equiv - \infty$ or $g \equiv \infty$.
	\end{enumerate}

  \end{enumerate}

  As usual, we denote a projection map by $\pi$, and for any $\incell$ we can define the projection
  \begin{align*}
    \pi \colon M^n \to M^k
  \end{align*}
  for $k$ the sum of $\ionein$, such that the restriction of $\pi$ to our $\incell$ is a homeomorphism onto its image.
\end{definition}

It is not hard to see that what we are doing here is just projecting away from the coordinate 0 parts of the cell. This can be thought of as a canonical coordinate projection that any cell comes naturally equipped with, which is quite a fine thing to have at hand.

In what should hopefully be predictable at this point, we wish now to define what it means to \emph{decompose} our space into cells. At some point, we will cease prefacing these definitions with `as usual, we do so by induction' -- but that point is yet to come. So, as usual, we proceed by defining \cds by induction.

\begin{definition}[\CD \emph{of $M$}]
  A \emph{\cd} of $M$ is a finite set defined by some strictly increasing finite sequence $\aoneak$ that form the set
  \begin{align*}
    \{(- \infty, a_1), (a_1, a_2), \hdots, (a_k, \infty), \{a_1\}, \{a_2\}, \hdots \{a_k\} \}.
  \end{align*}
  That is all the sequential open intervals (including those with infinite endpoints) plus the singleton sets. Just for the sake of belabouring the point, this is a definitionally \emph{definable} set.
\end{definition}

As we have time and time before, we now up the dimension by induction to define general \cds:

\begin{definition}[\CD \emph{of $M^{n+1}$}]
  A \cd of $M^{n+1}$ is a finite partition, $\fancyD$, of $M^{n+1}$ into cells, such that
  \begin{align*}
    \Set{\pi (C)}{C \in \fancyD}
  \end{align*}
  is itself a decomposition of $M^n$, with each $\pi$ the respective projection as discussed above.
\end{definition}


We trust the conjunction of these two definitions into an understanding of \cds in arbitrary dimensions is clear by induction.
An idea that may seem a bit apropos until a bit later on is that of \emph{compatibility} -- but rest assured, dear reader, that this will all come together shortly.

\begin{definition}[Compatibility]
  We call a \cd $\fancyD$ of $M^n$ \emph{compatible} with a subset, $X \subseteq M^n$ if for each cell, $C \in \fancyD$, either $C \cap X$ is empty, or $C$ is a subset of $X$.
\end{definition}


\subsubsection{The \CD Theorem}
\sectionmark{The \CD Theorem}

The importance of this last point will be made clear in the theorem we have been building to: the \CD Theorem -- which essentially says that these compatible decompositions exist and are compatible with any finite collection of definable sets and, most importantly, that definable functions are continuous on each cell in such a decomposition (defined in the domain of the function, of course). Properly, and as proved by Knight, Pillay, and Steinhorn \cite{knight_definable_1986}:

\begin{svgraybox}
    Note the use (about to be made) of subscripts on statements $\In$ and $\IIn$ to denote the dimension of $M^n$ to which each statement refers. This is going to be notationally useful in the proof but may seem a bit queer at present and without introduction.
  \end{svgraybox}

\begin{theorem}[\CD]
  \label{thm:cell-decomposition}
  Take $n \in \N$. Then
  \begin{enumerate}[label={}]
    \item[$\In$ ] Suppose $\vonevm{X}{k} \subseteq M^n$ are \defnb sets. Then there is a \cd of $M^n$ \cmptble with each $X_j$.
    \item[$\IIn$ ] If $\funcdom{f}{X}{M}$ is \defnb, then there is a \cd, $\fancyD$ of $M^n$ \cmptble with $X$ s.t. the restriction $\funcrestr{f}{C}$ is \cont for each $C \in \fancyD$.
  \end{enumerate}

  Further, and in analogy to the Monotonicity theorem, if our $\vonevm{X}{k}$ or $f$ (depending on case $\In[]$ or $\IIn[]$) are \defnb over $A \subset M$, then we can take the cells in  $\fancyD$ to be similarly \defnb over $A$; that is, with the \textbf{same} parameters.
  
  \pagebreak

  \begin{svgraybox}
    This last point is perhaps a bit unfair to mention, as we will not be providing proof for it -- though for the sake of interest, it would feel incomplete to not at least analogize with Theorem \ref{thm:monotonicity}. In truth, what follows is not a full proof of \CD, but a special case where we take $M$ to be $\R$ and use the yet unproven (or even stated) result of \emph{uniform finiteness}. We take this result entirely for granted in the lectures due to the oddity that a complete (unassuming) proof somewhat `bootstraps' uniform finiteness into the induction we do on $\In[j]$ and $\IIn[j]$, proving it as we go along. This is because uniform finiteness is itself an immediate consequence of the \CD Theorem (which makes the proof a fun little oddity). For our purposes, we take it as assumedly true -- in part due to the length of this proof even \emph{with} that assumption -- and trust that our dear intelligent reader sees plainly how we could fix this in the absence of the assumption.
  \end{svgraybox}

\end{theorem}

Uniform finiteness is a generalization of the finiteness theorem we proved earlier (Theorem \ref{thm:2finiteness}), but with potentially many parameters and higher dimensions. As with the argument for the \CD theorem, we will similarly restrict our attention to the case where $M = \R$. This special case is as follows:

\begin{proposition}[Uniform Finiteness (for $\R$)]
  \label{prop:unif-finiteness}
  Suppose $X \subset \R^{n+1}$ is \defnb with each fibre $X_x$ finite for $x \in \R^n$. Then there is some $N \in \N$ such that $\card{X_x} \leq N$ for all $x \in \R^n$.
\end{proposition}


The following proof is due to van den Dries \cite{dries_remarks_1984} which, for no reason other than interest's sake, we mention went on to inspire the later work of Pillay and Steinhorn in \cite{pillay_definable_1986}.

\begin{proof}[of \CD (Theorem \ref{thm:cell-decomposition})]
  We proceed by induction on parameter $n$. The base cases are both already done for us; $\In[1]$ is immediate from the definition of \omy, and $\IIn[1]$ is given by the Monotonicity theorem. What we go on to show is two inductive facts that `bounce off' one another in a sense, to allow us to prove both $\In$ and $\IIn$ for all $n$. These are
  \begin{enumerate}[label=(\alph*)]
    \item \label{pf:cd:a} Given $\In[1], \hdots, \In$ and $\IIn[1], \hdots, \IIn[n-1]$, we can conclude $\IIn$; and
    \item \label{pf:cd:b} Given $\In[1], \hdots, \In$ and $\IIn[1], \hdots, \IIn[n]$, we can conclude $\In[n+1]$.
  \end{enumerate}
  That these two facts together give us the desired result should be clear.
  Getting there requires a bit more effort, and so we simply begin with \ref{pf:cd:a}.
  Thus we wish to prove $\IIn[n]$: that for a \defnb $\funcdom{f}{X}{M}$, there is a \cd $\fancyD$ of $M^n$ \cmptble with $X$ and having \contty of $\funcrestr{f}{C}$ for each cell, $C \in \fancyD$.

  Suppose $\funcdom{f}{X}{M}$ is such a definable function. We may assume this because we already have $\In[1]$. By this, we may assume $X$ is a cell. If $X$ is not already an open-cell, then recall that we can take its image under the canonical projection away from zero coordinates. Since we do not refer here to the dimension of $X$, we assume that it is open or has been made so as described and then use our inductive hypothesis to conclude. So, we suppose $X \in \fancyD$ is an open cell on which $f$ is \cont. Take
  \begin{align*}
	\Xprime = \Set{x \in X}{f \ \textrm{is \cont and \defnb at} \ x}.
  \end{align*}
  Clearly, $\Xprime$ is \defnb, and we are supposing that we know $\Xprime$ to be open in $X$. Using inductive assumption $\In[n]$, we get a \cd, $\fancyD$ with $\R^n$ \cmptble with $X \setminus \Xprime$ and with $\Xprime$. If some $C \in \fancyD$ is an open cell contained in $X$, we get \contty of $f$ on $C$  by density; that is, $C \cap \Xprime \neq \emptyset$, and so $C \subseteq \Xprime$ and it follows that $\funcrestr{f}{C}$ is \cont. Supposing however that $C$ was \emph{not} an open cell, we apply the aforementioned projection construction, and the argument just presented holds (up to a change in dimension).

  This would be all well and good to end off \ref{pf:cd:a} with, were it not predicated on the yet unjustified density of $\Xprime$ in $X$ -- and so we now prove this. Suppose $B \subseteq X$ is an open box. We will show that there must exist a point in $B$ at which $f$ is \cont. In analogy to our proof of monotonicity, we know that if $\Xprime[B]$ is an open box contained inside of $B$, then $f$ takes on infinitely-many values on $\Xprime[B]$ (following from $\In$). This is the obvious case. Supposing otherwise, we proceed as follows:

  Construct a sequence of open boxes, $(B_j)_{1 \leq j \leq n}$ in $B$, and sequence $(\In[j])_{1 \leq j \leq n}$, of open intervals, each $\In[j]$ having length less than $\frac{1}{j}$, with the closure, $\cl{B_{n+1}} \subseteq B$, and $f(B_n) \subseteq \In$. Then, by compactness. we get that the intersection of all $B_n$ is non-empty, and at some point in this intersection, $f$ is continuous. This is of course just our claim -- we now go on to \emph{prove} this by construction.

  To get $\In[1]$, simply consider $f(B) \subseteq \R$ -- meaning
  \begin{align*}
    f(B) = \bigcup_{p \in \N} J_p \cup F
  \end{align*}
  for $F$ a finite set, and $J_p$ a countable set of open intervals of length less than 1. Then, $B$ is given by
  \begin{align*}
    B = \left( \bigcup_{p \in \N} f^{-1}(J_p) \cap B \right) \cup \left( \bigcup_{r \in F} f^{-1}(r) \cap B \right).
  \end{align*}

  To each half of the middle cup, we can apply $\In$ to determine the contents of each of the respective \emph{big} cups to be a finite union of cells -- and so $B$ must be an \emph{countable} union of cells, each of which is contained in one of these sets. Perhaps coming a bit out of left field, we apply the Baire Category theorem to conclude that by the openness of $B$, so too must one of these cells be open.
  \begin{svgraybox}
    Notice that this is one reason we restrict ourselves to working over $\R$ -- the Baire Category theorem does not hold in every DLO model. So this argument could not be broadened beyond the reals (or compact spaces) as we are currently undertaking it.
  \end{svgraybox}
  This \emph{cannot} be one of $f^{-1}(r) \cap B$, as it would then contain a box on which we took the value of $r$, an so this open cell must be in one of $f^{-1}(J_p) \cap B$ for some $p$. We take $J_1$ to be that $J_p$, and $B_1$ to be an open box contained in $f^{-1}(J_1) \cap B$, with $\cl{B_1} \subseteq B$. As desired, we then have $f(B_1) \subset \I_1$. Clearly the first step in an induction, we then (incompletely) note that, having $\vonevm{I}{n}$, $\vonevm{B}{n}$ constructed, we repeat exactly as above to finish the induction.

  And with that, we can give ourselves a \emph{light} patting on the back -- for as much as we've done so far, this is just the end of the proof of \ref{pf:cd:a}. To get the `bounced-back' half of the induction, we now go on to prove \ref{pf:cd:b}; that is, given $\In[1], \hdots, \In[n]$, $\IIn[1], \hdots, \IIn[n]$, we may derive $\In[n+1]$.


  For reasons of breaking up this lengthy proof into its two constituent sections, please enjoy the following horizontal line:

  \medskip
  \footnote{The earlier footnote should now perhaps be more sensical in light of the use of these long, calming, and meditative horizontal lines. Please do your best not to confuse the two; we trust you limit yourself in taking any comfort in the shorter lines, as any trustworthy reader would. Remember, if ever you have trouble, think of the little rhyming aphorism: \emph{long is calm.}}
  \centerline{\rule{0.6667\linewidth}{.2pt}}
  \medskip

  Try not to have too much fun with that, now. We move on to proving \ref{pf:cd:b}; recall our assumptions that $\In[1], \hdots, \In[n]$ and $\IIn[1], \hdots, \IIn[n]$ hold. We want now to prove $\In[n+1]$. First, we start with a small proposition.

  \begin{proposition}
    Suppose $\fancyD_1$, $\fancyD_2$ are \cds of $\R^{n+1}$ with a common refinement -- that is, another \cd, $\fancyD$ of $\R^{n+1}$ compatible with all cells in each of $\fancyD_1$ and $\fancyD_2$. Terminology-wise, we say that $\fancyD$ \emph{refines} $\fancyD_1$ and $\fancyD_2$, or that $\fancyD$ is a \emph{refinement} of $\fancyD_1$ and $\fancyD_2$.
    \label{prop:cell-refinement}
  \end{proposition}

  %\begin{svgraybox}
   % A frustrated author's aside: please take no notice of the $\square$ that sits just above this box and proceeds Proposition \ref{prop:cell-refinement}. Its presence is a mystery, and the method to remove it proves elusive, even after what would be not ungenerously called a \emph{cursory} amount of investigation. We will simply pretend it does not exist and trust that the honest, caring reader does so as well.
 % \end{svgraybox}


  \begin{subproof}[of Cell Refinement (Proposition \ref{prop:cell-refinement})]
    For the purpose of transparency, we note that this proof was left out of the lecture and as an exercise for the interested (or obligated) viewer. The following takes inspiration from van den Dries \cite{dries_tame_1998}. We note that this could be made a bit cuter if we had the machinery of \emph{dimension} that we will soon define, but in either case, this proof is relatively trivial. We have our $\fancyD_1$ and $\fancyD_2$ two decompositions of the trivially definable subset of, $ \ \R^{n+1}$; namely, $ \ \R^{n+1}$ itself. We can then simply take a decomposition of the ambient space (which here is the \emph{whole} space) containing our definable subset. We have seen previously that we can take this decomposition to partition each cell of $\fancyD_1 \cup \fancyD_2$ -- and the `restriction' of this decomposition to our definable set (again, to ensure this is sufficiently belaboured, this is not actually a restriction since our definable set is $\R^{n+1}$), we are left with our everywhere (on cells in $\fancyD_1 \cup \fancyD_2$) compatible decomposition.
    \smartqed
  \end{subproof}

 % \begin{svgraybox}
  %  A much less frustrated author's aside: let us all take a moment and appreciate the appropriately placed $\square$ above. We can move forth pretending all is well again, and we hope this has not caused the reader \emph{too} much undue stress -- beyond, of course, the normal, cursory amount.
 % \end{svgraybox}

  Now, if some $A \subseteq \R$ is \defnb, we define its \emph{type}, $\tau(A)$ as follows:
  \begin{description}
    \item Let $\vonevm{a}{L}$ strictly increasing be the points in the boundary of $A$. We let $\tau(A)$ then act as an indicator function on sequential intervals, $(a_j, a_{j+1})$, defined as the positive unit (which is to say, whatever $1$ is in $\M$) when that interval sits inside $A$, and otherwise the negative unit (similarly defined). We set $a_0 = - \infty$ and $a_{L+1} = \infty$ (which is starting to become sort of an out-of-bounds normalcy), and define
    \begin{align*}
      \tau_{2j + 1} = 1
    \end{align*}
    if $(a_j, a_{j+1}) \subseteq A$ (and of course $-1$ otherwise). Note, of course, that this would then mean that the given interval is contained in the complement of $A$. For even numbers, we have
    \begin{align*}
      \tau_{2j} = 1
    \end{align*}
    if $a_j \in A$ and naturally, $-1$ if $a_j \not\in A$. Then, we have $\tau(A) = (\vonevm{\tau}{2 \cdot L + 1}) \in \{0, 1\}^{2 \cdot L + 1}$ a sequence of length $2 \cdot L + 1$ consisting of $\pm 1$s.

  \end{description}

    To ground ourselves for a moment, consider $$\tau((1, 2] \cup \{ 3 \} )) = (-1,\ -1,\ +1,\ +1,\ -1,\ +1,\ -1),$$
    which can immediately convince ourselves of non-uniqueness, as this is the same sequence induced by $\tau((9, 10] \cup \{5,\ 7\} )$.

    At this point, you may be a bit confused (if the recollection is even still there) as to why we made such a fuss early on about \emph{uniform finiteness} -- when we've yet to see it used. Well, for the anxious amongst you, satisfaction will come soon, as we now make use of that perhaps seemingly unjustified assumption.

    %By UF, we get the following (and again, \underline{\textbf{please}} ignore the spurious QED-type symbol that appears at the end of this proposition),
    By UF, we get the following:
    \begin{proposition}
      If we have a \defnb $X \subseteq \R^{n+1}$, then the set of the types of fibres -- that is
      \begin{align*}
        \Set{\tau(X_x)}{x \in \R^n}
      \end{align*}
      -- is finite. Further, for each given choice of type, the set of fibres giving rise to that type is \defnb.
      \label{prop:definable-fibre-types}
    \end{proposition}
    Perhaps starting this sentence with `of course' would be unfair, but it shouldn't be hard to see, intuit, or at least \emph{guess} that the set of $x \in \R^n$ giving rise to any \emph{particular} type is usually empty. As before, this proof was left as an exercise to the responsible party, and so please forgive any clear indications of amateurism -- were they absent, there may be something a little suspect going on.

    \begin{proof}[of Proposition \ref{prop:definable-fibre-types}]
      This we will not belabour even slightly. We have assumed UF and so simply appeal to UF in the case of $M^{n}$, by which we get that each fibre is finite -- and so must have finite types belonging to its elements. In the case of each type, it is given by some point, or open interval in a fibre and so is defined by points and open intervals. Supposing there were infinitely-many \emph{different} such types, we would have to be working in a non-finite dimension. Thus, \defnbty falls out, almost as if by accident.
      \smartqed
    \end{proof}

    Now, with these two propositions, we are just about ready to put together our proof of \ref{pf:cd:b}. That is, we now prove $\In[n+1]$ follows.

    By Proposition \ref{prop:cell-refinement} (on cell-refinement), we can assume $k = 1$ -- which you'll recall is the number of sets we have from our original statement of the theorem. Then, by Proposition \ref{prop:definable-fibre-types} and $\In$, we get a \cd, $\fancyD$ of $\R^n$ such that for each $C \in \fancyD$ there is an $L$ and a $\tau \in \{\pm 1\}^{2 \cdot L + 1} $ such that
    $$
      \tau(X_x) = \tau
    $$
    for all $x \in C$. Fixing such $C$, $\tau$, and $L$, we get \defnb functions $\vonevm{f}{L}$ with each $f_1 < \hdots < f_L$ and such that either
    \begin{enumerate}
      \item $(f_i, f_{i+1})_C \subseteq X$; or
      \item $(f_i, f_{i+1})_C \cap X = \emptyset$,
    \end{enumerate}
    with the normal condition of $f_0 = - \infty$, $f_{L+1} = \infty$. Notice also that the same holds for the graphs of $f_j$ -- that they are either contained in or disjoint from $X$, excluding, of course, the cases at infinities (which we allowed above). Now, with our small army of \defnb functions, all defined on this cell $C \in \R^n$, we can use $\IIn$ (which we proved in our induction in part \ref{pf:cd:a}) to conclude that we may partition $C$ into finitely-many cells, such that $f$ is \cont on each cell. With the end very nearly in sight, we apply $\In$ to get a \cd of $\R^n$ (and in fact of all cells from the above) \cmptble with all the resulting cells. Finally, taking graphs over those cells will give us the desired \cd of $\R^{n+1}$. And with that, we have `bounced back' such that we may repeat this induction \textit{ad infinitum} (or \textit{nauseam}, depending on how much you enjoy this sort), and the \CD Theorem is proved (in our special case) for all $n$. You are encouraged, if you've not seen something like this before, to revel just a bit in the nicely placed, well-deserved QED symbol -- as this is one of the longer proofs we've seen thus far. Not to worry you, but they will get longer and more complicated, so enjoy.
\end{proof}

\sectionmark{A Post \CD Break}

It is now at \emph{this} point that the reader not only \emph{may} but is encouraged to give themselves their well-deserved, no-bars-held, hearty pat on the back for the fortitude it took to get through that. Perhaps also giving the above a quick re-read wouldn't be such a bad idea, as there are some bits and subtleties that this author needed a few passes to feel entirely comfortable with. For a proof that doesn't make the assumptions we did here, the reader is directed to \cite{knight_definable_1986}, but by no means necessarily encouraged towards it —- just made aware of its existence. We are through one of the more laborious parts of this first part of the course with that done. For the masochists in the audience, we note that there is more length and labour to come of this variety. Still, for the normal amongst us, it is with a relaxation that we should move on to discuss \emph{\defnbcnctdns}, followed by \emph{dimensionality} -- and the problem that mathematicians have for coming up with different words for distinct concepts. But first -- which is a phrase we perhaps begin sentences with all too often -- we have a bit of miscellany to address.

  %%%%%%%%%%%%%%%%%%%%% chapter.tex %%%%%%%%%%%%%%%%%%%%%%%%%%%%%%%%%
%
% sample chapter
%
% Use this file as a template for your own input.
%
%%%%%%%%%%%%%%%%%%%%%%%% Springer-Verlag %%%%%%%%%%%%%%%%%%%%%%%%%%
%\motto{Use the template \emph{chapter.tex} to style the various elements of your chapter content.}
\subsection{In Absence of Aproposia: A Brief Foray}
\label{chap:apropos} % Always give a unique label
% use \sectionmark{}
% to alter or adjust the chapter heading in the running head
\sectionmark{Some Introductions}

Here, we mention a point or two (just one actually) that would otherwise go overlooked. Necessity is entirely eschewed, and this section can be safely skipped without any loss in the understanding of the course as it was intended to be presented. One is perhaps encouraged to leave. This short section exists only for the interested, dedicated, obliged, or otherwise neurotic reader. In short, this encapsulates that which has no other place thus far, nor is deserving too much attention.

\bigskip
\centerline{\rule{0.3333\linewidth}{.2pt}}
\smallskip

\subsubsection{Let's Get it Out of the Way}
\label{sec:aproposia-is-not-a-word}
\noindent One of the first thoughts the critical or unassuming reader -- which is likely all of you -- had when reading this section title was whether \emph{that} word was even a word \emph{at all}. If you are at all a curious person, searching the internet or your favourite etymological website or reference text for the word `aproposia,' you would have failed in your task -- unless the task you set out was to assure yourself that it is, in fact, \emph{not} a common or even extant word. Were `apropos' of Latin root, then perhaps this linguistic abomination may make more sense, but considering the French origin of `apropos,' no such logic applies. Nonetheless, there is little the reader can do about this choice of wording (save for one particular professor). And since we felt its hopefully clear meaning and appealing sound a nice title considering the nature of the topic, you should then be glad at all, dear reader, that Latin is no longer the language of science you would be expected to learn to have what would be considered `valid' opinions on its workings. If anything, feel free to use it as a word to confound and confuse your dear friends and colleagues. While we will let you get away with calling everything and anything `normal,' we insist you take `aproposia' as valid and unilaterally call that a fair trade.

\subsubsection{Onto the Miscellany}

In which we again abuse language, in the sense that there is only one fact of miscellaneous variety. Sometimes, certain benign ridiculousness must be allowed to amuse your readers or even just oneself. We hope you are amused. What we do, after all, is exciting not in its protracted execution but in the few moments where it all comes together. It's how we prevent the onset of early insanity when getting into the thick of these sorts of ideas.

\begin{remark}
  Recall that an \om structure requires all \defnb sets \textbf{with parameters} to be given by finite unions of points and open intervals (as defined by the model). If we only assume this for sets \defnb \textbf{without} parameters, the resulting theory is \emph{legitimately} and provably weaker than what we get with \omy. This was in passing mentioned to be potentially true earlier on, but in one of the question and answer sessions held for this course, it was pointed out that it is \emph{in fact} true by a gentleman with the given name Chris. In a moment, we will be referring to him by his family name, Miller. This awkward wording will be clear in just a moment.

  An easy (in the sense of being counter-exemplary) way to show this is due to Dolich, Miller, and our old friend Steinhorn \cite{dolich_structures_2009}. This can be expressed (though perhaps not proven, as the length of their paper implies) quite compactly by constructing the model.
  $$
    \M = (\R, <, V)
  $$
  for $V$ the Vitali set (defined by the Vitali \emph{relation}):
  $$
    V = \Set{(x, y) \in \R^2}{x - y \in \Q}.
  $$
  The only $\emptyset$-\defnb subsets of this are $\emptyset$ itself and $\R$ -- and so this fits the definition of being $\emptyset$ \om, but given any defined parameter, we end up with the rationals \defnb; clearly, this is \emph{not} \om. If you recall the short mention made earlier, it may interest the reader to note that this is \emph{weakly} \om.

  \begin{exercise}
    As an exercise of \emph{this} author to the reader, attempt to prove that the above expansion admits QE. \textit{Hint: You would be well-served to first read section \ref{chap:defn_dimensionality} on dimensionality and definable closures.}
  \end{exercise}
\end{remark}

Putting an end to this brief foray into intrigue with a splash of ridiculousness, we now move back to a consequential idea once one defines \cds: connectedness.


  %%%%%%%%%%%%%%%%%%%%% chapter.tex %%%%%%%%%%%%%%%%%%%%%%%%%%%%%%%%%
%
% sample chapter
%
% Use this file as a template for your own input.
%
%%%%%%%%%%%%%%%%%%%%%%%% Springer-Verlag %%%%%%%%%%%%%%%%%%%%%%%%%%
%\motto{Use the template \emph{chapter.tex} to style the various elements of your chapter content.}
\subsection{Connectedness, and What it Entails}
\label{chap:connectedness} % Always give a unique label
% use \sectionmark{}
% to alter or adjust the chapter heading in the running head
\sectionmark{Some Introductions}

It is likely that one unacquainted too thoroughly with this topic of study has been imagining in their minds open intervals as they might imagine them in $\R$. Unfortunately for such a reader, they are about to be disabused of that rather idyllic notion -- and we will speak to when we legitimately \emph{may} presume that things are (what we will come to call) \emph{\defnbly \cnctd}. At this point, we should have an acronym to express that the importance of this will not be immediately apparent but will soon come to be and be worked into our standard model of understanding these objects we find ourselves working with. Creating such an acronym is left as an exercise to the reader.

\bigskip
\centerline{\rule{0.3333\linewidth}{.2pt}}
\smallskip

\subsubsection{But \emph{Why} Aren't Things (Always) Connected?}
\label{sec:connectedness-defn}

The easy answer is: sometimes we just don't work over connected domains. Take, for example, $(\Q, <)$ -- where we may write $\Q = (- \infty, \sqrt{2}) \cup (\sqrt{2}, \infty)$.

\begin{svgraybox}
  Recall that neither interval is open in $\Q$ by non-membership of the irrational endpoints in the domain.
\end{svgraybox}

In particular, we find that a convenient definition of a \defnblycnctd set, $X \in M^n$ is

\begin{definition}[Definable Connectedness]
  Some $X \in M^n$ is \defnblycnctd if $X$ is \emph{not} the union of two disjoint non-empty \defnb open subsets of X.
\end{definition}


\begin{example}
  Open intervals, we claim, are \defnblycnctd. So too are cells -- although this one we reason a bit about. By \CD, we know that \defnb sets have finitely-many \defnblycnctd components which are maximal \defnbly \cnctd subsets. So by uniformity, it seems that we can always take a cell not to be the union of two disjoint non-empty \defnb open subsets of itself.

\end{example}

We formalize this idea with the following proposition.

\begin{proposition}
  \label{prop:dfnblycnctd}
  Support some $X \in M^{m + n}$ is \defnb. Then, there exists some $N \in \N$ such that if $x \in M^m$. then $X_x$ has \emph{at most} $N$ \defnblycnctd components.
\end{proposition}

\begin{corollary}
  Given $\mathcal{N}$, a structure elementarily equivalent to $\M$ (satisfies exactly the same first order sentences in our language) for $M$ \om, then $N$ is also \om. In short,
  \begin{align*}
    \M \equiv \mathcal{N} \ \wedge \M \ \mathrm{\om} \implies \mathcal{N} \ \mathrm{\om}.
  \end{align*}
\end{corollary}

\begin{svgraybox}
  For the interested and more informed reader than expected necessarily, it is noted that the property of \emph{minimality} (not \omy) is \emph{not} preserved under elementary equivalence as \omy is. This is one of the motivations for the idea of \emph{strong}-minimality, but here, we pretend there is no notion of a \emph{strong}-\omy.
\end{svgraybox}

This corollary will come to be quite important later, so if nothing else from here, keep that fact in the back of the mind as we go forward.

\subsubsection{Definable Choice \& Curve Selection}
\noindent In the interest of time, space, audience, and simply relevance, in most cases, we will not be providing examples as we have before (as in the case of expansion by the Vitali relation) and instead assume we take $\M$ an \om expansion of an \emph{ordered} field, $(M, <, +, \cdot, 0, 1)$. Not addressed here, but as a good exercise to the interested reader, attempt to show that this must necessarily be a real closed field. We will think in abstractness only, and the reader who even tries to think of an example should be rather a bit ashamed of what they've done.

Without any faff, we get right into the point of this section.
 \begin{proposition}[Definable Choice]
  \label{defnb-choice}
  \begin{enumerate}
    \item \label{defnb-choice-1} Given a \defnb family, $X \subseteq M^{n+m}$ with $\pi$ the projection map onto the first $n$ coordinates, then there is a \defnb map, $\funcdom{f}{\pi X}{M^n}$ with $\graph{f} \subseteq X$.
    \item \label{defnb-choice-2} Given $E$ a \defnb equivalence relation on a \defnb set, $X \subseteq M^n$, then $E$ has a set of representatives.
  \end{enumerate}

\end{proposition}
\begin{proof}[Proofs of the above (Proposition \ref{defnb-choice}]
  First, we go on to show that if some $\XsubsetMn[n]$ is \defnb and \inhb, then we may \defnbly pick some element, $e(X) \in X$. As ever, we induct on $n$. 
  
  Suppose $n=1$, then either
  \begin{enumerate}
    \item $X$ has a least element, and so we let $e(X)$ be that; and otherwise
    \item $X$ has a left-most interval, (with respect to our order), and splitting by cases, we can take
      \begin{itemize}
        \item $e(X) = 0$ if $(a, b) = (- \infty, \infty)$;
        \item $e(X) = b - 1$ if $a = - \infty$, $b \in M$;
        \item $e(X) = a + 1$ if $a \in M$, $b = \infty$; and
        \item $e(X) = \cfrac{a + b}{2}$ if both are finite.
      \end{itemize}
  \end{enumerate}
  Note, of course, that our arithmetic is well-defined here due to the field expansion we are working with. We now induct a bit differently to prove each of cases \ref{defnb-choice-1} and \ref{defnb-choice-2};
  
  \begin{itemize}
    \item In the first case (definable without parameters), we put $f(x) = e(X_n)$ for $x \in \pi X$ (since our fibre is \inhb; and then for 
    \item we take $\Set{e(A)}{A \ \textrm{is an equivalence class of} \ E}$ a \defnb set or representation, as desired.
  \end{itemize}
\end{proof}

We may go on to refer to \emph{definable choice} as DC, stated in the acronym section upfront. From this, we can then go on to prove a neat and useful little result called \emph{curve selection}.

\begin{proposition}[Curve Selection]
  \label{prop:curve-selection}
  Suppose $\XsubsetMn$ is \defnb and $a \in \fr{X}$ (the frontier of $X$, defined by to the closure of $X$ less $X$). Then, there is a \cont \defnb \inj $\funcdom{\gamma}{(0, \epsilon)}{X}$ for some $\epsilon > 0$ with
  \begin{align*}
    \lim_{t \to 0^{+}} \gamma (t) = a.
  \end{align*}
\end{proposition}

Predictably, this is going to use the result we just proved on \defnb choice, so we jump right in.

\begin{proof}[of Proposition \ref{prop:curve-selection}]
 Let $\abs{x} = \max \{ \abs{x_1}, \abs{x_2}, \hdots, \abs{x_n} \}$ . Since $a \in \fr{X}$, the set 
   \begin{align*}
     \Set{\abs{a - x}}{x \in X}
   \end{align*}
   is a \defnb set with arbitrarily small positive elements, and contains some interval $(0, \epsilon)$.
   
   If $t$ is in this $(0, \epsilon$), then the set 
   \begin{align*}
     \Set{x \in X}{\abs{a - x} = t}
   \end{align*}
  is \inhb. Thus, by DC, we get a \defnb $\funcdom{\gamma}{(0, \epsilon)}{X}$ such that $\abs{a - \gam(t)} = t$ for some $t$ in our interval. Clearly then, $t$ is \inj with limit $a$ as $t$ approaches 0 (on the right). As a throwback, we can apply the Monotonicity Theorem and reduce $\epsilon$ to reach the assumption that $\gamma$ is \cont.
\end{proof}

We've perhaps teased at it now for a bit, and the particularly knowledgable or prescient reader will have seen this coming, but it is at this point we move on to a spot of dimension theory. The discussion here is interesting in that we approach it both from the angle of definability (as expected) and algebraicity and see what comes to pass.

  %%%%%%%%%%%%%%%%%%%%% chapter.tex %%%%%%%%%%%%%%%%%%%%%%%%%%%%%%%%%
%
% sample chapter
%
% Use this file as a template for your own input.
%
%%%%%%%%%%%%%%%%%%%%%%%% Springer-Verlag %%%%%%%%%%%%%%%%%%%%%%%%%%
%\motto{Use the template \emph{chapter.tex} to style the various elements of your chapter content.}
\subsection{Definable Dimension: The First Go-Round}
\label{chap:defn_dimensionality} % Always give a unique label
% use \sectionmark{}
% to alter or adjust the chapter heading in the running head
\sectionmark{Some Introductions}

As so often one finds in mathematics, dimension is one of those ideas that it almost seems each mathematician has a notion of how it should be defined -- and more truthfully, it has hugely varying meanings across the vast spectra of mathematical disciplines. And perhaps even more, unfortunately, it so often seems the case that the average mathematician was raised on a dictionary of no more than 30 or so words -- which they go on to use and reuse and smash together into all-new words, for the most part equally inequivalently to how their office-mate might be doing the same thing in their preferred area of abstractness. To say the same in so many fewer words: we often find that the same words mean different things to different mathematicians -- Here, we are going to \emph{dimension} from the logician's point of view (at least in the context of \omy as we have been thus far). Little will be terribly new here, although we will encounter some interesting techniques, some longer proofs, and a fair deal of dallying about with unions, intersections, and differences of sets; all this to say, a set theorist's dream. Those of you who feel we are being a bit sly in our phrasing is correct, and you will come to see why, if you aren't already keenly aware of the big surprise we are half-heartedly keeping from you, in Section \ref{chap:alg_dimensionality} -- when a very exciting realization will come to pass. However, for now, we stick to the exciting realizations that come with continuing on along the logical path we have set for ourselves thus far.

\bigskip
\centerline{\rule{0.3333\linewidth}{.2pt}}
\smallskip

\begin{svgraybox}
  We continue in our assumption from the previous section -- that is, that we take $\M$ an \om expansion of an \emph{ordered} field, $(M, <, +, \cdot, 0, 1)$, and prove things about this construction in generality. Strictly speaking, this is not necessary here, even speaking less strictly than we have about this assumption earlier on. It is simply a matter of convenience and lack of loss of generality.
\end{svgraybox}


\subsubsection{So How Does One Define Dimension?}
\label{sec:defn_dim}

\noindent Tempting though it is to answer ``depends on who you ask'' and move on with our day, the question does bear significant thought. We start with the following definition.

\begin{definition}[Dimension]
  Suppose $\XsubsetMn$ is \defnb and \inhb. Then we set 
  \begin{align*}
    %\dim{X} = \max{ \left\{ \sum_{j = \vonevm{j}{n}} j \colon X \textrm{ contains a } \vmcell{j}{n} } \right\}
    \dim{X} = \max{ \left\{ \sum_{j \in \vonevm{j}{n}} j \colon X \textrm{ contains a } \vmcell{j}{n} \right\} }
  \end{align*}
  to be the dimension of our subset of $M^n$, with the dimension of $\emptyset$ being $- \infty$. Intuitively, this hopefully makes sense -- dimension being given by the largest object (of our current interest) sitting inside our space.
  \label{defn:defn_dim}
\end{definition}

At first blush, this seems to be a reasonable definition of dimension given the manner in which we've been defining the rest of our toolkit -- and with further blushing, we will come to find it even more reasonable than it may even have initially appeared. For those finding this \emph{unreasonable}, we would encourage a re-reading of some of the earlier definitions, specifically on cells and their decompositions, or, barring that, just setting fire to this manuscript and going on about your day. Either way, we say a little something about what subsets with non-empty interior may tell us.

\begin{lemma}
  \label{lemma:interior-open-cell}
  If $\XsubsetMn[n]$ has interior, $\inter{X}$, \inhb, then there is is a \defnb \inj map, $\funcdom{f}{X}{M^n}$ with image of $X$ under $f$ containing an open cell.
\end{lemma}
This lemma will go one to be quite useful in a moment when we wish to say some useful things about, for example (and in particular), the \emph{dimension invariant of \defnb \bijtions }.

\begin{proof}[of Lemma \ref{lemma:interior-open-cell}]
  We promise you now, barring life-threatening or otherwise dire circumstances, that this will be the last preface to a proof of this sort -- and trust you, dear reader, to recognize when we are setting up a proof by inductions on $n$. But for now, we proceed by induction on $n$.
  
  When $n = 1$, then $\XsubseteqMn$ is a \inhb subset of $M$, and so is infinite. Since $f$ is injective, $f(X)$ so too is it infinite (relying as well on \defnbty to conclude this), and so $f(X)$ contains an open interval -- which we know to be an open-cell.
  
  Suppose now that $n > 1$, and our inductive hypothesis holds for all $k$ between $1$ and $n -1$. By CD, we will simply assume that $X$ is itself already an open cell and that $f$ is continuous. So, by CD we can take some \cd $\fancyD$ of $f(X)$. Should one of the $C \in \fancyD$ be open, we are done -- and so we assume none are (to address all possible cases). Then, since $X$ is the union of the preimages of the cells in $\fancyD$, some $f^{-1}(C_j)$ contains an open cell, $C \subseteq M^n$. Say then that $C$ is contained in the preimage of, without loss of generality, $f^{-1}(C_1)$. Then, restricting $f$ to $C$, we have
  \begin{align*}
    \funcrestr{f}{C} \colon C \to C_1
  \end{align*}
  \cont, \defnb, and \inj. Recall the homeomorphic nature of the projection away from 0-coordinates. Similarly, in extensions of fiends (this is not about to be proven), open cells are homeomorphic to the ambient space, and so since we expand a field, any $A \in M^n$ are \defnbly homeomorphic to $M^m$. So, putting this together, we have $C$ \homeom to $M^n$ and $C_1$ \homeom to $M^{\ell}$ for some $\ell < n$ -- and so $C$ is \defnb \homeomic to $M^n$, $C_1$ for $M^{\ell}$, and thus we have a \cont \defnb \inj 
    \begin{align*}
      g &\colon M^n \to M^{\ell} \\
      h &\colon M^{\ell} \to M^{\ell}
    \end{align*}
    where we define $h$ by
    $$
      h \colon y \mapsto g(0, y)
    $$
    with, to be clear, $0$ coming from $M^{n - \ell}$ (in case things weren't seeming all above-board here). By our inductive hypothesis, we can finally conclude that the image $h(N^{\ell})$ \emph{has} interior. Letting some $b \in \inter{h(M^{\ell})}$ and $a \in M^{\ell}$ with $h(a) = b$, we can say by continuity of $g$ that for $x \in M^{n - \ell} \setminus \zeroset$ sufficiently small, that $g(x, a)$  will sit in the image of $h$, and so be achievable by some argument to $h$. We will call this $a^{\prime} \in M^n$ satisfying $f(a^{\prime}) = g(x, a) = g(0, a^{\prime})$ -- clearly contradicting \injtvty. 
    
    We have that $f(X)$ contains an open cell, and we can now pronounce our proposition proved.
    \smartqed
\end{proof}

\subsubsection{Dimension Under Functions}

Now, we will go on to show something very sensible indeed -- and something that, should it \emph{not} hold, cause great concern. That is, invariance of domain under definable bijection.

\begin{proposition}[A Bit on Definable Bijections]
  \begin{enumerate}
    \item \label{prop:defbij_1} If $X \subseteq Y \subseteq M^n$ all \defnb, then the dim
      \begin{align*}
        \dim{X} \leq \dim{Y} \leq n.
      \end{align*}
    \item \label{prop:defbij_2} If $X \subseteq M^m$, $Y \subseteq M^n$ are both \defnb and $\funcdom{f}{X}{Y}$ is a \defnb bijection between them
      \begin{align*}
        \dim{X} = \dim{Y}.
      \end{align*}
    \item \label{prop:defbij_3} If $X, Y \subseteq M^n$ are both \defnb, then
      \begin{align*}
        \dim{X \cup Y} = \max{\{ \dim{X}, \dim{Y} \}}
      \end{align*}
  \end{enumerate}  
\end{proposition}

Hopefully, these statements don't come across too contentious nor necessarily difficult to prove, but nonetheless, we state our own:
\begin{proof}
  \leavevmode
  \begin{enumerate}

    \item This is almost as free as it gets; by containment of $X$ in $Y$, we already have that no cell in $X$ can have dimension larger than one in $Y$. Similarly, we have bounding above of the dimension of any cells by the ambient space, which we know to have dimension $n$. That felt like perhaps a lot to say for what is quite clear.

    \item Assume we have proven \ref{prop:defbij_3}, which we do first. To be clear, 
      \begin{verbatim}
        GOTO 3
        COMPREHEND_3()
        GOTO 2
      \end{verbatim}
    And you're back! So now, let's prove \ref{prop:defbij_2}. Suppose $d = \dim{X}$ and $e = \dim{Y}$. By symmetry, it clearly suffices to show that $d \leq e$ (or vice versa) as one could then simply switch $X$ for $Y$.

    Since $X$ has dimension $d$, it must contain an $\vmcell{i}{m}$ with indices $\vonevm{i}{m}$ summing to $d$. Let a cell $C \subseteq X$ an $\vmcell{i}{m}$ of the same dimension, $d$. We have that $f$ sends $X$ to $Y$, and so composing with a \defnb \bij $C \to M^d$, we get a \defnb \injtion
    \begin{align*}
      g \colon M^d \to Y.
    \end{align*}
    Cell decomposing $Y$ into $\vonevm{C}{k}$ (each a cell, together having union $Y$), we have then that
    \begin{align*}
      M^d = \inv{g}(C_1) \cup \cdots \cup \inv{g}(C_k).
    \end{align*}
    By openness of $M^d$, as before we must have that some $\inv{g}(C_j)$ (without loss of generality, suppose $C_1$) contains some open cell, $D \subseteq M^d$, such that $D \subseteq \inv{g}(C_1)$. Since $C_1$ sits in $Y$, we generically say it is an $\vmcell{j}{n}$. For sake of contradiction, suppose these coordinates have sum less than $d$, and denote this quantity $\pri{e}$. Then, we have
    \begin{align*}
      \pri{h} \colon D \xrightarrow{g} C_1 \xrightarrow{\sim} M^{\pri{e}}
    \end{align*}
    a \defnb \inj function. We can imagine $M^{\pri{e}}$ as `sitting inside' $D$, since we know by our assumption that its dimension is necessarily lesser. So now, considering $0$ in $M^{d - \pri{e}}$, we have the map
    \begin{align*}
      \dpri{h} \colon D \to M^{\pri{e}} \times \left\{ 0 \right\} \subseteq M^d
    \end{align*}
    also both \defnb and \inj -- but then since we have an open cell, $D$, we contradict Lemma \ref{lemma:interior-open-cell}. Thus, we have that $d \leq e$, and so must also have $d = e$ by symmetry of the argument, and we are done.

    \item Let $d = \dim{X \cup Y}$. and let $C$ a cell in the union that witnesses the dimension, $d$ (which to be clear, and going forward, will refer to a $\vmcell{c}{n}$ with $\sum_{j=1}^{n} c_j$ = d). Now, let $\funcdom{\phi}{C}{M^n}$ a \defnb bijection. Then, we have (from before, and not non-obviously)
        \begin{align*}
          M^d = \phi(C \cap X) \cup \phi(C \cap Y)
        \end{align*}
        open (by $M^d$ open), and so one of the constituent parts of the union must be also. Hence, one of the two contains an open cell, $D$ (without loss of generality, suppose $\phi(C \cap X)$) such that 
        \begin{align*}
          \inv{\phi}(D) \subseteq X
        \end{align*}
        and so
        \begin{align*}
          \dim{\inv{\phi}(D)} = \dim{D} = d.
        \end{align*}
        So, then $\dim{X} \geq d \geq \dim{X}$, and the rest follows.
  \end{enumerate}
\end{proof}

Now with a bit under our belt about this notion of dimension that we would reasonably expect, we go on to prove a few more interesting facts that will come to use later on. In particular, we first show something about the additivity of the dimension of fibres.

\subsubsection{Concerning Fibres}

\begin{proposition}
  \label{prop:fibre_add}
  Suppose $X \subseteq \MmMn$ is \defnb, and some positive natural number $d \leq n$. Denote
  \begin{align*}
    X(d) = \Set{x \in M^n}{\dim{X_x} = d},
  \end{align*}
  recalling, of course, that $X_x$ denotes the fibre of $X$ over $x$ (for any who have forgotten). Then, $X(d)$ is \defnb and
  \begin{align*}
    \dim{ \bigcup_{x \in X(d)} \left( \{x\} \times X_x \right) } = \dim{X(d)} + d
  \end{align*}
\end{proposition}

\begin{proof}[of Proposition \ref{prop:fibre_add}]
  Suppose $\fancyD$ is a \cd of of $\MmMn$ \cmptble with $X$. If some $C \in \fancyD$ is an $\vmcell{i}{m+n}$, then its projection onto its first $m$ components, $\pi C$, is an $\vmcell{i}{m}$ and each fibre, $C_x$ is an $\vmncell{i}{m+1}{m+n}$ for $x \in \pi C$. Clearly, we have that
  \begin{align}
    \label{align:misc_1}
    \dim{C} = \dim{\pi C} + \dim{C_x}
  \end{align}
  for each $x \in \pi C$.

  Now, setting $\pri{C} \in \pi \fancyD$ (the cells given by this projection on all cells in $\fancyD$) and letting $\vmvnsupbr{C}{1}{k} \subset \fancyD$ the cells contained in $X$ whose projection is $\pri{C}$. Then for $x \in \pri{C}$, the fibre
  \begin{align*}
    X_x = \vmvnsupbr[\cup]{C_x}{1}{k}
  \end{align*}
  and so
  \begin{align*}
    \dim{X_x} &= \max{ \left\{ \ \vmvnsupbr{\dim{C_x}}{1}{k} \ \right\} }.
  \end{align*}
  Combining this with \ref{align:misc_1}, we have that
  \begin{align*}
    %\dim{X_x} &= \max{ \left\{ \ \dim{C^{(1)}} - \dim{\pri{C}}, \ \hdots, \ \dim{C^{(k)}} - \dim{\pri{C}} \ \right\} },
    \dim{X_x} &= \max{ \left\{ \ \dim{C^{(1)}}, \ \hdots, \ \dim{C^{(k)}} \ \right\} } - \dim{\pri{C}} ,
  \end{align*}
  which is completely independent of $X$ -- that is, \emph{where} we fibre makes no difference in the dimension. Let $d$ be this dimension, given by the maximum. Then $\dim{C_x} = d$ for all $x \in \pri{C}$, so $\pri{C} \subseteq X(d)$, and thus $X(d)$ is a finite union of cells and therefore, as desired, \defnb. So we have the \defnbty part of the proposition -- we now need to prove the given equality.

  We have that our $d$ is defined as above by
  \begin{align*}
    d &= \max{ \left\{ \ \dim{C^{(1)}}, \ \hdots, \ \dim{C^{(k)}} \ \right\} }  - \dim{\pri{C}}  \\
    d &= \dim{ \ \ \bigcup_{i=1}^{k} C^{(i)} \ \ }    - \dim{\pri{C}}  \\
  \end{align*}
  Recall, this union is the part of $X$ laying above $\pri{C}$, and so we can further improve to get
  \begin{align*}
    d &= \dim{ \ \ \bigcup_{x=\pri{C}} \{x\} \times C_x^{(i)} \ \ }  - \dim{\pri{C}}
  \end{align*}
  and so
  \begin{align*}
    \dim{ \ \ \bigcup_{x=\pri{C}} \{x\} \times C_x^{(i)} \ \ } = d + \dim{\pri{C}}
  \end{align*}
  which, taking the union over all $\pri{C} \in \pi \fancyD$ with $\pri{C} \subseteq X(d)$, gives us the result, knowing that the maximal dimension over all $\pri{C}$ is the dimension of $X(d)$. With that, we have gotten exactly what we want.
\end{proof}

As a corollary, we get something of a more general form (in several parts) of the previous proposition, which goes as follows:
\begin{corollary}
  \label{cor:dim_fibres}
  \leavevmode
  \begin{enumerate}
    \item \label{cor:dim_fibres_1} If $X \in M^{m + n}$ is \defnb, then $X$ has dimension given
      \begin{align*}
        \dim{X} &= \max_{0 \leq d \leq n}{ \left\{ \dim{X(d)} + d \right\} } \\
        &\geq \dim{\pi X}
      \end{align*}
      for $\pi$ as in the proposition.

    \item \label{cor:dim_fibres_2} Suppose $X \subseteq M^n$, $\funcdom{f}{X}{M^m}$ \defnb. Denote
      \begin{align*}
        X_f(d) = \left\{ \ x \in M^m \colon \dim{\inv{f}(x)} = d \ \right\}.
      \end{align*}
      Then $X_f(d)$ is \defnb and further,
      \begin{align*}
        \dim{\inv{f}(X_f(d))} = \dim{X_f(d)} + d
      \end{align*}
      and $\dim{X} \geq \dim{f(X)}$.

    \item \label{cor:dim_fibres_3} Consider now a special case of taking products, where $\XMn[X]{m}$, $\XMn[Y]{n}$ are \defnb. Then, perhaps predictably,
      \begin{align*}
       \dim{X \times Y} = \dim{X} + \dim{Y}.
      \end{align*}
    \end{enumerate}
\end{corollary}

Easy as they are, these proofs were left to the viewer, so please enjoy this unassessed attempt at just that. Note that as we know part \ref{cor:dim_fibres_3} to simply be a special case following from part \ref{cor:dim_fibres_2}, we only prove the first two parts of the corollary.

\begin{proof}[of Corollary \ref{cor:dim_fibres}]
  \leavevmode
  \begin{enumerate}
    \item That the dimension of $X$ is not lesser than that of its projection onto a subset of its coordinates is obvious and doesn't bear much consideration. Further, immediately applying Proposition \ref{prop:fibre_add}, simply noting that taking the union over fibres of subsets of $X$ of each possible dimension, $0 \leq d \leq n$, gives us this result exactly, we are done.

    \item In essence, here what we are doing is again applying Proposition \ref{prop:fibre_add}, but now to the graph of our $X$, $f(X)$. Again, by a similar argument as in the previous part of this proof, and the direct application of the formula as given in the proposition, the result we desire falls out.

    \item Follows from the previous.
    \end{enumerate}
\end{proof}


%\begin{exercise}
%  \label{exc:dim_exc}
%  \fix{Show that: Supposing $X, Y \subseteq M^{1 + n}$ with $Y$ \inhb, and that for each $x \in M$, either $X_x = \emptyset$ or $\dim{X_x} < \dim{Y_x}$, then $\dim{X} < \dim{Y}$.}
%\end{exercise}

\begin{proposition}
  \label{prop:dim_exc}
  Supposing $X, Y \subseteq M^{1 + n}$ with $Y$ \inhb, and å for each $x \in M$, either $X_x = \emptyset$ or $\dim{X_x} < \dim{Y_x}$, then $\dim{X} < \dim{Y}$.
\end{proposition}
This is stated without proof.
We now use this result to show the following, perhaps quite intuitive lemma.
\begin{lemma}
  \label{lemma:closures_fibres}
  Suppose $\XMn{n+1}$ is \defnb. Then
  \begin{align*}
    \left\{ x \in M \colon \cl{X_x} \neq \cl{X}_x \right\}
  \end{align*}
  is finite. That is, there are only finitely-many points at which the closure of the fibre is \textbf{not} the fibre of the closure.
\end{lemma}

\begin{proof}[of Lemma \ref{lemma:closures_fibres}]
  Note that we always have that the closure of the fibre is always contained in the fibre of the closure -- that is, $\cl{X_x} \subseteq \cl{X}_x$. Suppose, for purposes of contradiction, that the set defined above \emph{is} infinite. As such (with \defnbty), it contains an open interval which we shall call $I$. By definition of our set, for each $x \in I$, there is an open box, $\XMn[B]{n}$, witnessing the difference -- that is, that
    \begin{align*}
     B \cap X_x = \emptyset
    \end{align*}
  \emph{and simultaneously,}
    \begin{align*}
      B \cap \cl{X}_x \neq \emptyset
    \end{align*}
  We know the family of open boxes in $M^n$ to be \defnb, and so by definable choice (do recall that we said it would become important), we can get each such box as a \emph{\defnb function} (whose notation we overload by referring to it as $B$) of the prescribed $x \in I$. By monotonicity, we may assume that $B$ is continuous, and so taking
    \begin{align*}
      U = \left\{ \ (x, y) \in I \times M^n \colon y \in B(x) \ \right\},
    \end{align*}
  we have $U$ open in $I \times M^n$, and $U \cap X = \emptyset$. However, we must \emph{also} have then that $U \cap \cl{X} \neq \emptyset$ -- which is a clear contradiction. Thus, our original assumption of the infinitude of our set was incorrect, and the result is proven.
\end{proof}

Now, using this lemma and a bit from before, we can prove the following theorem:

\begin{theorem}
  \label{thm:fr_dim}
  Suppose $\XMn{n}$ is \inhb, and \defnb. Then, $\dim{\fr{X}} < \dim{X}$.
\end{theorem}

This one is quite intuitive, and its proof is not terribly involved -- but the result itself is quite useful and sees much use in inductive arguments about these sorts of ideas. We will not be using this result quite often, but later modules in this lecture series will make much more judicious usage.

\begin{proof}[of Theorem \ref{thm:fr_dim}]
  We start with the base case of $\XMn[X]{1}$. We trust that this follows without comment.

  Now consider $\XMn[X]{n + 1}$, and suppose that the result holds for $M^k$ for any $1 \leq k \leq n$ (in fact we only need weak induction here, but we assume this regardless for reasons of not mattering). For each coordinate, $\vonevm{i}{n+1}$, consider the set
    \begin{align*}
      \cl{X}_i = \left\{ \ x \in M^{n + 1} \colon x \in \cl{X \cap \inv{\pi_i}(\pi_i(x)) } \ \right\}
    \end{align*}
  for $\pi_i$ predictably the projection map onto the $i$-th coordinate. Note, of course, that $\pi_i(X) \subseteq \pi(X)$.

  Without any loss of generality, we make our arguments with respect to $x_1$ here -- although there is no reason they should not similarly hold for any other coordinate. Suppose we have some $x \in \cl{X} \setminus \cl{X}_1$ (a set defined similarly to the frontier of \emph{all} of $X$). Then we can write this $x$ as $x = (x_1, \pri{x})$ where $\pri{x} \in \cl{X}_{x_1}$ and further $\pri{x} \not\in \cl{X_x}$. By the above lemma (Lemma \ref{lemma:closures_fibres}) there are only finitely-many possible such $x_1$ -- and so the set difference above lies in a finite union of hyperplanes, each of which projecting onto a single point under $\pi_1$. So, there exist points $a_{1,1}, \ \hdots, \ a_{1, k_1} \in M$ such that
    \begin{align}
      \cl{X} \setminus \cl{X}_1 \subset \bigcup_{j=1}^{k_1} \inv{\pi_1}(a_{1, j}).
      \label{align:fr_proj}
    \end{align}
  As before, it the choice of coordinate $1$ was arbitrary, and this same argument holds for \emph{all} coordinates. Thus, permuting coordinates, we get that for each remaining coordinate, we get some $\left(  a_{i, 1}, \hdots, a_{i, k_i} \right)$, for $i$ varying over the remaining coordinates, such that \ref{align:fr_proj} holds for the respective $i$. Orthogonality of these planes is clear by the differing $i$'s, and so the closure minus the union of these closures sits within a set that can be characterized as
    \begin{align*}
      \cl{X} \setminus \bigcup_{i=1}^{n+1} \cl{X}_i \subseteq \left\{ (a_{1},\ j_{1},\ \hdots,\ a_{n+1},\ j_{n+1}) \ \ \colon \ \ \begin{aligned} & j_1 = 1,\ \hdots,\ k_1, \\ & j_{n+1} = 1,\ \hdots,\ k_{n+1} \end{aligned} \right\}.
    \end{align*}
    \emph{The} thing to take notice of here is that this set on the right is \emph{finite}, and so our difference in the closure of $X$ and union of closures of the projections onto each coordinate is contained in a finite set.

    Recalling what we proved about the dimension of the union of a set of sets (Proposition \ref{prop:fibre_add}), we get that
      \begin{align*}
        \dim{\fr{X}} &= \dim{\cl{X} \setminus X}\\
                     &\leq \max{ \left\{ \dim{\cl{X}_i \setminus X}, \ 0 \right\} }.
      \end{align*}
      The problem is now reduced to showing simply that $\dim{\cl{X}_i \setminus X}$ has dimension less than $X$ for each $i$. This doesn't turn out to be too difficult, and we jump right in with little fuss.

      Without loss of generality (a phrase we are beginning to believe may have a nice initialism), take $i = 1$, and some point $a \in M$. Then
        \begin{align*}
          \dim{ \cl{X \cap \inv{\pi_1}(a)} \setminus \left( X \cap \inv{\pi_1}(a) \right) } < \dim{\inv{\pi_1}(a)}
        \end{align*}
      so long as this is \inhb, by the inductive hypothesis. Otherwise, if it \emph{is empty}, then we consider the whole faff above, by which we of course mean
        \begin{align*}
          \cl{X \cap \inv{\pi_1}(a)} \setminus \left( X \cap \inv{\pi_1}(a) \right) = \left( \cl{X}_1 \setminus X \right) \cap \inv{\pi_1}(a),
        \end{align*}
      and so (and we don't claim obviousness necessarily) for each $a \in M$, with $X \cap \inv{\pi_1}(a) \neq \emptyset$, we have that
        \begin{align*}
          \dim{\cl{X}_1 \setminus X} \cap \inv{\pi_1}(a) < \dim{X \cap \inv{\pi_1}}.
        \end{align*}
      Notice that the left half of this inequality can be identified by the fibre over $a$, and the right as the fibre over $x$, so that
        \begin{align*}
          \dim{\cl{X}_1 \setminus X}_a < \dim{X_a}.
        \end{align*}
      %So, by Exercise \ref{exc:dim_exc}, we get that
      So, by Proposition \ref{prop:dim_exc}, we get that
        \begin{align*}
          \dim{\cl{X}_1 \setminus X} < \dim{X}.
        \end{align*}
      Pulling the usual and, to be honest at this point, tired and tiring rabbit out of our (frankly garish) hat of permuting coordinates, we see (in a twist we are certain no one saw coming) that this argument works for \emph{all} coordinates, not just $i = 1$ -- and so the result holds for $i$ from 2 up to $n + 1$. We now have that one of the following holds:
        \begin{enumerate}
         \item $\dim{\cl{X} \setminus X} < \dim{X}$
         \item $\dim{X} = 0$
       \end{enumerate}
      The former is exactly what we wanted, and should the dimension of $X$ be 0, then $X$ is finite and closed -- and we are finished.
      \smartqed
\end{proof}

And now, we are done with dimensionality! Sort of. In a way. But in another way, not at all -- and for those who gave the introduction more than a cursory skim or have seen the Table of Contents, you will know that this has only been the first half of our investigation into the idea of dimension. Without belabouring beyond reason, as we will be sure to do so in the next section as well, prepare now to investigate a \emph{seemingly} less motivated (at least from the previous material) definition and discussion on dimension -- but one that will become vital to our proofs to come.

  %%%%%%%%%%%%%%%%%%%%% section.tex %%%%%%%%%%%%%%%%%%%%%%%%%%%%%%%%%
%
% sample section
%
% Use this file as a template for your own input.
%
%%%%%%%%%%%%%%%%%%%%%%%% Springer-Verlag %%%%%%%%%%%%%%%%%%%%%%%%%%
%\motto{Use the template \emph{section.tex} to style the various elements of your section content.}
%\section{Algebraic Dimensionality: A Second, and  (Perhaps) More Enlightening Approach}
\subsection{Model-Theoretic Dimension: Once More with Feeling}
\label{chap:alg_dimensionality} % Always give a unique label
% use \sectionmark{}
% to alter or adjust the section heading in the running head
\sectionmark{Some Introductions}

What we just worked through was broadly a very natural extension of the ideas we'd been toying with so far, and the idea of defining dimensionality by cells is a very sensible one indeed. However, the antsy and number theoretically inclined amongst you may be wary of how little of that sort has cropped up so far. Fear not then, our brethren and sistren in the fine and noble study of numbers and the little things they do -- as we now examine the idea of dimensionality from an \emph{algebraic} perspective and begin to work on the ideas that will come to unite the logical world of \omy with that of our own (not to expose our biases too blatantly). We will come to the wonderful conclusion that, in fact, these two notions are one and the same. We can start to drop the veil of pretense we have been vaguely shrouding this whole affair in -- although you would be in the know had you read the introduction in its entirety or even just the Table of Contents. Importantly, this idea will lead us to one of the important tools that allows the \pwt to work at all; that being this connection between algebra, \nt and \omy that together can put together something bigger and more wonderful than each constituent on its own. So, dear reader, please enjoy a rehashing of much of what you saw just a section ago, but now through an entirely different lens -- one that will perhaps expose you to some ideas and inner workings you'd not thought of or seen previously.

\bigskip
\centerline{\rule{0.3333\linewidth}{.2pt}}
\smallskip

\begin{svgraybox}
  As in Sections \ref{chap:connectedness} and now \ref{chap:defn_dimensionality}, we will continue in our assumption that we have $\M$ an \om expansion of an \emph{ordered} field, $(M, <, +, \cdot, 0, 1)$, and prove things about this construction in generality.
\end{svgraybox}

Although at no point will we be diving deeply into the depths of model theory and all the fun/horror that this may entail (depending on who you ask), this is going to be the part of the course that relies most heavily on \emph{some} knowledge of model theory. There's little to do to get around that besides having little heuristics in one's head that make things \emph{kind of make sense} enough to not worry about it too much. All this said, the reader who's not even heard the term `model theory' in their life should not be intimidated, as we will be doing our best to hold your hand (or not as you see fit) through the dicier parts of this section, and so much as we hate to keep using this phraseology, belabouring that which may be very obvious to one of a different background. While this may not have been mentioned before, as a series of papers (or more likely document on a computer of some variety), it is easy but vastly rewarding to jump over sections you feel eminently comfortable about with a reckless abandon. This is true for all, but I hope the one professor I hope does get to at least the vast majority of words written here. But for anyone else -- skip judiciously, and skip as if you've much better things to do (which you likely do).

\subsubsection{So How Does One Define Dimension?}
\label{sec:alg_dim}

\noindent Tempting though it is to answer ``depends on who you ask'' and move on with our day, the question does bear significant thought. We start with the following definition (and promise we are not trying to induce d\'ej\`a Vu):

\begin{definition}[Algebraic Dimension]
  \
  Unlike in the previous section, we will \emph{not} be starting with a definition of the algebraic dimension and working from there -- you simply are not yet prepared, given the content discussed so far. We now back up a little bit and will return to this topic when more sufficiently prepared. Suffice it to say for now that our definition of dimension will be reliant on the idea of \emph{algebraic} (and then, equivalently, \emph{\defnb}) closures (in the model-theoretic sense).
  \label{defn:alg_dim_fake}
\end{definition}

\subsubsection{Some Recollections or Introductions}

Here we recall for some, and for others introduce for the first time, some of the rudimentary ideas of model theory, and in particular of how one goes about defining a model-theoretic or algebraic closure of a set -- and then go on similarly to before to use this to define dimensionality.

\subsubsection{On Closures of Various Types}

\begin{definition}[Algebraic Closure]
  We say some $A \subseteq M$, then the \textbf{model-theoretic algebraic closure of $A$}, which we will denote by $\acl{A}$, is the union of all finite $A$-\defnb subsets (in $\M$) of $M$.
\end{definition}

\begin{definition}[Definable Closure]
  Again, given $A \subseteq M$, then the \textbf{\defnb closure of $A$}, denoted $\dcl{A}$, is given by the union of all $A$-\defnb singletons.
\end{definition}

Note perhaps a more careful attention to parameters being paid here than in previous sections -- precision here concerning the topic is, not to say, of greater importance than elsewhere, but can much more easily lead to confusion if not properly, clearly, and rigorously attended to.


In our setting, in particular in reference to the order-structure our models possess, these seemingly disparate definitions of closure actually coincide -- that is,
\begin{align*}
  \acl{A} = \dcl{A}.
\end{align*}

For convenience, we will, in general, simply work with the definable closure.

\pagebreak

\subsubsection{Some Easy Properties}

\begin{proposition}[Basically Free]
\leavevmode
Let $A \subseteq M$. Then
  \begin{enumerate}
    \item $A \subseteq \dcl{A}$
    \item $\dcl{\dcl{A}} = \dcl{A}$
    \item $\dcl{A} = \bigcup \left\{ \ \dcl{\{a_1, \hdots, a_n\}} \colon n \in \N, \ a_j \in A \ \right\}$
  \end{enumerate}
\end{proposition}

The third of these properties is referred to as having \emph{finite character}. These each feel reasonably obvious enough that we can go without proof, and the interested reader should have little to no trouble coming up with one themselves. The \emph{non}-trivial property, however, that we \emph{will} be proving is referred to as the \emph{Exchange lemma}, which we will curiously notate as a theorem.

\subsubsection{A Less Easy Property}

\begin{theorem}[Exchange (Pillay \& Steinhorn \cite{pillay_definable_1986})]
  \label{thm:exchange}
  Suppose $A \subseteq M$, and $a, b \in M$. Then if $b \in \dcl{A \cup \{ a \}}$, but $b \not\in \dcl{A}$, then we may `swap' $a$ and $b$, concluding that $a \in \dcl{A \cup \{ b \}}$.
\end{theorem}

Whilst this may seem apropos of nothing and not terribly well-motivated, with a bit of squinting, head-scratching, and perhaps dusting off a copy of \emph{Linear Algebra and Applications 6e} or some other such undergraduate textbook, you may realize that this is, in fact, a generalization of a fact you know quite well -- the Steinitz Exchange Lemma; that is, that any two bases of isomorphic (finite) vector spaces contain the same number of elements. Much like many model-theoretic results, and is, in a sense, sort of the point of model theory, we see this sort of thing often: where something abstract-feeling and seemingly far-removed has immediate consequences should specification to any particular model be made. Now let's prove this, relying heavily on the \Mt to do so.

\begin{proof}[of Exchange (Theorem \ref{thm:exchange})]
  Adding constants for elements of $A$, we may suppose that $A = \emptyset$. So, then every $b \in \dcl{\{a\}}$ and \emph{not} in $\dcl{\emptyset}$. Since $b \in \dcl{\{a\}}$, then there is an $\esdefnb$ function, $f$, with $a \in \dom{f} \subseteq M$ and such that $f(a) = b$ (by definition).

  Since $\dom{f}$ is $\esdefnb$, it is a finite union of $\esdefnb$ points and $\esdefnb$ intervals. Supposing $a$ was one such point, then $a$ would be $\esdefnb$, and by $b$ in the \defnbcl of $a$, we would also have $b$ $\esdefnb$ -- but we know it not to be by assumption, and so this is a contradiction. Thus, $a$ is \emph{not} a point in $\dom{f}$, and we can thus assume $\dom{f}$ to be an $\esdefnb$ open interval, $I$, with $a \in I$. By \Mt, we can assume that f is strictly monotonic (or constant) on $I$. If constant, then
    \begin{align*}
      b = f(a) = f(m)
    \end{align*}
  for $m$ the midpoint of $I$ (which must be $\esdefnb$ by I $\esdefnb$), and further by $f \esdefnb$, so too is $b \esdefnb$. Again, this contradiction implies $f$ is non-constant, and so must be strictly monotonic.

  Then, $\inv{f}(b) = a$ is well defined, and so $a$ is \defnb from $b$ -- that is, $a \in \dcl{\{b\}}$ as desired.
  \smartqed
\end{proof}

Having those easy properties above (in particular finite character and idempotence of $\dclop$) along with exchange means that the concept of \defnbcl is what is referred to as a \emph{\pregeom}. That is, $\dclop$ is a \pregeom in any model of the theory of $\M$. For those of us who have not previously encountered the idea of \pregeoms, this is significant because a \pregeom comes equipped with a notion of \emph{dimension} -- which is exactly what we're after right now. Even if you are unfamiliar with the term, we can assure you, dear reader, with almost complete certainty that you are intimately familiar with several examples of these objects. Vector spaces, projective and affine spaces, and algebraically closed fields are all examples of \pregeoms -- although it should be noted that these are of quite different classifications (within the realm of this area of study) from one another \cite{pillay_geometric_1996}.

Suppose $A \subseteq M$, $\aMn[a]{n}$. We define
  \begin{align*}
    \dim{\sfrac{a}{A}} = \min{ \left\{ \ \card{\pri{a}} \ \colon \ \pri{a} \textrm{ a subtuple of } a \textrm{ and } a \in \dcl{A \cup \pri{a}}^n \ \right\} }.
  \end{align*}

An alternative characterization is as follows: we call some $X \subseteq M$ \emph{independent} over $A$ if for every $x \in X$, we have $x \not\in \dcl{A \cup (X \setminus \{ x \})}$. Then, the dimension as first defined is given by the \emph{maximum} cardinality of a subtuple which is independent over $A$. That is,
  \begin{align*}
    %\sfrac{a}{A} = \max{ \left\{ \card{X} \ \colon \ X \subseteq M \textrm{ is independent over } A \right\} }
    \dim{ \sfrac{a}{A} } = \max{ \left\{ \card{X} \ \colon \ \forall x \in X \ . \ x \not\in \dcl{A \cup (X \setminus \{ x \})} \right\} }
  \end{align*}

We assume that $\M$ is sufficiently saturated and that all parameter sets are small relative to this saturation claim. We will not be discussing at any great (or really even short) length what this means, but we will note that this is actually a stronger condition than we necessarily need to take -- it is just a convenience that will make things a bit easier down the line. Consider now the following definition:

\begin{definition}
  Suppose $\XMn[X]{n}$ is \defnb over $A$. Then we define the dimension of $X$ to be, predictably,
    \begin{align*}
      \dim{X} = \max{ \left\{ \ \dim{\sfrac{a}{A}} \ \colon \ a \in X \ \right\} }.
    \end{align*}
    \label{defn:alg_dim}
\end{definition}

\begin{remark}
  Without the assurance of sufficient saturation, the points $\left\{ \sfrac{a}{A} \right\}$ may not even exist, and so instead, we could quantify over all elementary extensions of our model -- but for our purposes, we acknowledge and go on to ignore this small potential snag.
\end{remark}

Although we have been referring to some set of parameters, $A$, throughout, it turns out that as long as you take a ``small'' set (again, relative to the saturation), then the choice of $A$ does not truly affect the dimension of the definable $X$. To be clear, we will not be clear about quantifying appropriate ``smallness'' relative to the saturation in this course and simply sweep the issue under the rug for perhaps a secondary, more rigorous course on the topic.

Informally, we can envision that for two sufficiently small sets of parameters, $A \subseteq B \subseteq M$, we can clearly see that for \defnb $\XMn{n}$, we get $\dim{X}_B \leq \dim{X}_A$. Supposing the dimension with respect to $A$ to be $k$ with point $a \in X$ witnessing the dimension, without loss of generality we can take the first $(\vonevm{a}{k})$ to be independent over $A$ -- but then by saturation, we can similarly find $k$ independent such $(\vonevm{b}{k})$ over $B$ with type matching that for our $a$. Then, our $(\vonevm{b}{k})$ extends to $b \in X$, and so $\dim{X}_B \geq k$ as well. Thus, the choice of parameters is irrelevant (again, given this caveat).

\subsection{Reconciling Dimensionality}

Ultimately, our goal here is to show that the dimension we are currently discussing is, in fact, equivalent to that discussed in the previous (previous) section. To this end, consider the following lemma, which topologically characterizes the independence of points.

\begin{lemma}
  Suppose that $\aMn[a]{n}$, and $\AM[A]$ (small). Then, the coordinates, $(\vonevm{a}{n})$ of $a$ are independent over $A$ if and only if every $\Adefnb$ set $\XMn[X]{n}$ with $a \in X$ has \inhb interior.
  \label{lemma:indep_top}
\end{lemma}

\begin{proof}[of Lemma \ref{lemma:indep_top}]
  We start with the easy direction, and use a proof by contradiction. Suppose $\vonevm{a}{n}$ are \emph{dependent} over $A$. Without loss of generality, take $a_n \in \dcl{A \cup \{ \vonevm{a}{n-1} \}}$. Then, our language contains some formula, $\phi$, with parameters from $A$, such that $a_n$ is the unique $x$ satisfying $\phi(\vonevm{a}{n-1}, x)$ in $\M$. Now, let
    \begin{align*}
      X = \left\{ \ \aMn[x]{n} \ \colon \ \phi(\vonevm{x}{n}) \textrm{ holds and } x_n \textrm{ uniquely satisfies } \phi(\vonevm{x}{n}) \ \right\}.
    \end{align*}
  By uniqueness of $x_n$, we must have that $X$ has an empty interior and $a \in X$. Now, for the other direction, we use \cd by induction on $n$.

  If $n = 1$, and if $a_1$ is independent over $A$, then $a_1$ is not in any finite $\Adefnb$ set. So, if $a_1 \in X$ for $X$ $\Adefnb$, then $X$ is infinite, and so contains an open interval. Thus, it has \inhb interior.

\pagebreak 

  Supposing this holds for $M^n$, we take $\aMn[a]{n+1}$ with coordinates $(\vonevm{a}{n+1})$ independent over $A$, and suppose $\XMn[X]{n+1}$ to be $\Adefnb$ and having $a \in X$.
  \begin{svgraybox}
    It bears mentioning here, as we are about to use this fact but have neither mentioned, proved, nor intend to prove it, but the following is true: If we have a set of cells decomposed over some set of parameters, $A$, then we can take each of those cells to be defined over that same set of parameters.
  \end{svgraybox}
  Recall that by \cd, we get to suppose that $X = C$ is a cell defined over $A$. As such, $C$ may either be
    \begin{itemize}
      \item $C = \graph{f}$ for $\funcdom{f}{\pri{C}}{M}$ $\Adefnb$ -- in which case $a_{n+1} = f(\vonevm{a}{n})$, leading to a contradiction of independence over $A$; \ or

      \item $C = (f, g)_{\pri{C}}$ for $\funcdom{f,g}{\pri{C}}{M}$ \cont and $\Adefnb$ with $f < g$ (or one of the functions $\pm \infty$). This latter case must be true by exhaustion.
    \end{itemize}
  So, we get that $(\vonevm{a}{n}) \in \pri{C}$. We have that $\vonevm{a}{n}$ are independent over $A$ and that $\pri{C}$ is an $\Adefnb$ set containing them -- and so must have interior. By the inductive hypothesis, $\pri{C}$ is an open-cell, and then so too must be $C$, meaning $X$ has \inhb interior.

  With both directions proved, we are now finished.
  \smartqed
\end{proof}

Immediately putting that lemma to work, we can show the following:
\begin{proposition}
  Let $\XMn[X]{n}$ an $\Adefnb$ set, and $k \leq n$ a possible dimension for $A$. Then $\dim{X} \geq k$ if and only if there is a coordinate-projection, $\funcdom{\pi}{M^n}{M^k}$ such that the projection $\pi(X)$ has \inhb interior.
  \label{prop:inhb_coord_proj}
\end{proposition}

\begin{remark}
  The maximum such possible $k$ is sometimes referred to as the \emph{topological dimension} of the set -- although (and as we have bemoaned before) this term is also used to refer to a variety of inequivalent concepts. It is also perhaps worth noting that, in our setting, this topological dimension agrees with the definable (or cellular) dimension; just as we are about to show, so too does the algebraic dimension we have been discussing here.
\end{remark}

\begin{corollary}[Agreement of Dimension Definitions]
  For $\XMn[X]{n}$ a \defnb set, the notion of $\dim{X}$ as described in this section (Definition \ref{defn:alg_dim}) agrees with that of the previous section (Definition \ref{defn:defn_dim}).
  \label{cor:dim_agree}
\end{corollary}
This is stated without proof, which shouldn't be even near out of the reach of the reader who has been feeling confident with the content thus far. It would not be unwise to take a go at it as a test of one's familiarity and skill with the ideas and the sorts of proofs we have seen thus far.

Now, we turn our attention back to proving the previously neglected proposition.
\begin{proof}[of Proposition \ref{prop:inhb_coord_proj}]
  We start with the forward direction. Suppose $\dim{X} \geq k$. Then for our $a \in X$ of interest, there is an $n$-tuple that, without loss of generality, can be taken to be $(\vonevm{a}{n})$ with first $k$ coordinates non-trivial -- ie. that has coordinates independent over $A$. Then, should we take the projection onto the first $k$ coordinates, we know that $(\vonevm{a}{k})$ lies in that projection, and so by the lemma we just proved, that projection has \inhb interior.

  Now for the other direction, we suppose such a projection, $\funcdom{\pi}{M^n}{M^k}$ exists such that $\pi(X)$ has \inhb interior. Again, we suppose for convenience that this occurs for the first $k$ coordinates. Then, $\pi(X)$ is definitionally $\Adefnb$, and having non-empty interior, must contain an open $\Adefnb$ box, $U = I_1 \times \hdots \times I_k$. Using saturation, we inductively find $(\vonevm{a}{k}) \in U$ independent over $A$. Then, taking $a \in X$ such that $\pi(a) = (\vonevm{a}{k})$, we have that this witnesses the dimension of $X$ is no less than $k$. So, $\dim{X} \geq k$.
\end{proof}



  %%%%%%%%%%%%%%%%%%%%% section.tex %%%%%%%%%%%%%%%%%%%%%%%%%%%%%%%%%
%
% sample section
%
% Use this file as a template for your own input.
%
%%%%%%%%%%%%%%%%%%%%%%%% Springer-Verlag %%%%%%%%%%%%%%%%%%%%%%%%%%
%\motto{Use the template \emph{section.tex} to style the various elements of your section content.}
%\section{Algebraic Dimensionality: A Second, and  (Perhaps) More Enlightening Approach}
\subsection{A Slew of Results on \SCDs}
\label{chap:smooth_cell} % Always give a unique label
\sectionmark{Some Introductions}

To wrap up now what may have felt like a very long first section, we state several results on \emph{\scds} -- in particular starting with a definition of what makes a \cd smooth in any case. For the beleaguered reader, just champing at the bit for discussion of \pw to start properly: we promise, this is the last bit between you and that carrot that's been dangling in front of you for that past while. For the reader, however, feeling that more detail should be wrung out before moving on, it is with great disappointment that we inform you that you have chosen the wrong text to follow and that it took this many sections for you to come to that conclusion. Either way, make room somewhere in your mind for a few more definitions, \lemmas, propositions, and theorems, and we will quickly be on our way to a not \emph{too} involved proof of the theorem it feels we set out to prove so long ago.

\bigskip
\centerline{\rule{0.3333\linewidth}{.2pt}}
\smallskip

This section, in particular, is more of a survey of useful results and ideas than the preceding sections, with much less attention paid to proving the claimed statements either at all or with much degree of rigour. For those interested in such detail, they are encouraged to reference (as one can for most that we have done and will do here) \emph{Tame topology and \Om structures} \cite{dries_tame_1998}.

\subsubsection{Smoothness}
Recall our setting; we are in an \om expansion of an ordered field, $\M = (M, <, +, \cdot, 0, 1, \hdots)$, and so much of what you'll recall from your classes in calculus and analysis apply -- for example, the notion of differentiability of a univariate function on a point in the interior of its domain can be well-defined. As a result, we use this ordered-field structure to define continuous differentiability in almost exactly this way.

\newpage

\begin{svgraybox}
  Note that we hand-wave quite a lot of the calculus-centric content here, as is it (reasonably) presumed to be relatively obvious how one would extend or make any arguments presented either more rigorous or work in higher dimensions than we discuss. We hope, for those ardent fans of the basics of calculus, that this is not too much of a disappointment.
\end{svgraybox}


\begin{theorem}
  Suppose $\funcdom{f}{(a, b)}{M}$ is \defnb, and $r \in \N$. Then $f$ is $C^{r}$ except at no more than finitely-many points (where $C^r$ is $r$ \contdfblty)
  \label{thm:cont_diff}
\end{theorem}

\begin{proof}[or rather, a bit of reasoning about the idea]
Much like in the calculus one should find themselves familiar with, we can define $C^r$ broadly as usual, and we will heuristic our way through how one does this when $r=1$. We have the limit definition of the derivative of $f$ and view the limit at each point from below and above due to the ordering of our field. We should note, of course, that this holds necessarily only on some subset of the domain rather than its entirety by necessity. By \Mt, we can assume continuity of $f$, and then if we can show the limit from above exists and is positive on some interval, then $f$ is strictly increasing (vice versa for decreasing). Then, after a slight bit of hand-waving, if we can show the two functions (limits from above and below) are $M$-valued (non $\pm \infty$), then $f$ is actually differentiable and can only be $\pm \infty$ at finitely-many points. Again, this is not claimed to be rigorous by any means.
\end{proof}

This is apropos of something to come, rather than something to worry about much now, but still bears mentioning, especially in the \textit{calculus-y} bit of the note. In the same way that we apply may the ``regular'' calculus constructions in our model, the same sorts of theorems follow in that model. In particular, note that the Mean Value theorem holds. This is mentioned as we will be using it without proof, just a bit later on.

%Equivalently, we can say that on some $\XMn[A]{n}$ \defnb, a \defnb map, $\funcdom{f}{A}{M^m}$, is $C^r$ on $A$ if there is some open $U \subseteq M^n$ and \defnb

\subsubsection{Smooth Cells and Decompositions}

Of course, to get to our usual definition (or rather, its analogue), we could follow all the standard methods (mostly) one sees in their first calculus course, but for our purposes, we move right along to talking about \emph{cells} again. 

\begin{definition}[$C^r$-cells]
A $\Crcell[r]$ is defined almost precisely the same way we did a cell originally, but now with the requirement that all functions involved (as in, as part of definitions and proceeding results) be $C^r$ themselves. For brevity, we will not re-enumerate the definition here.
\end{definition}

Very quickly then, we move on to \emph{smooth} \cd but won't prove it -- again in our excitement and closeness of what is to come.

\begin{theorem}[Smooth Cell Decomposition]
  Suppose $r \in \N$, then (just as in the \CDt there are two parts)
    \begin{enumerate}
      \item \label{item:smooth_cell_1} Suppose $\vonevm{X}{k} \subseteq M^n$ are \defnb. Then, there exists a \cd, $\fancyD$, \cmptble with $\vonevm{X}{k]}$ such that each $C \in \fancyD$ is $C^4$.

      \item \label{item:smooth_cell_2} (In analogy to piecewise-continuity) If $\funcdom{f}{X}{M}$ is \defnb, $\XMn[X]{n}$, then there is a \cd, $\fancyD$ into $\Crcells$ such that the restriction of $f$ to $C$ is $C^r$ for each $C \in \fancyD$.
    \end{enumerate}

  \label{thm:smooth_cell}
\end{theorem}

\begin{proof}[or rather, how one might prove Theorem \ref{thm:smooth_cell}]
  No proof is provided, although it doesn't vary too much from that of the proof of \CD \emph{without} smoothness. As before, one proceeds by induction on $n$ (we know, small promise broken), although now, similarly to how in the proof of the \CD theorem we proved two inductive statements back and forth, here we do a similar thing where we prove that, for the given function in \ref {item:smooth_cell_1}, we prove the set of points at which that function is \cont must be dense in $X$. Again, we proved something very similar before -- just now, this proves the existence of derivatives of certain type. In actuality, we get to use the \CD theorem proper to make things easier on ourselves, making this proof (arguably) a bit less difficult.
\end{proof}

And now, finally (which must feel shockingly early given the length of this section), we will finish off with a statement that we will come to use but do nothing here to prove or really discuss at much length.

Recalling that our expansion is over an ordered real field, we have been given quite a bit of specificity already -- so we may ask ourselves whether we can improve upon the $r$ in our $C^r$ specification (as in, taking it to be $\infty$ rather than just finite), or even better, just analytic. We may have set up the question a bit too cheerfully there then because the answer is \textbf{no}, we may not. While the original result is attributable to Pila and Wilkie, the results we present here are of Le Gal, and Rolin \cite{gal_o-minimal_2009}

\begin{proposition}[Le Gal et Rolin \cite{gal_o-minimal_2009}]
  Given an \om structure $\tilde{\R} = (\overline{\R}, \ \hdots)$ which \textbf{does not} have $C^{\infty}$ \cd \textbf{as defined in the usual way}, then the structure does not have it for any consistent definition of the concept.
\end{proposition}

We bold that second part of the (very informally given) statement, in particular, to emphasize that there is an \emph{normal} way of doing things, as inspired by the calculus we know on the reals. It is to that \textbf{usual way} that is being referred.

Funnily enough, however, one bumps into and finds themselves interested in structures that \emph{do} have $C^{\infty}$ \cd quite often, which is quite a nice thing indeed. And it is with this we come to what may seem an abrupt conclusion (of Part I).


\subsubsection{Saying Goodbye to the Easy Bit}

We covered a lot here, much of each piece building on the last -- so one would be forgiven (praised even) for giving the several previous sections a quick once, twice, or thrice-over just to make sure everything is solidly in place in their mind map. In the next section, we start on the \pwt proper and the two broad constituents that make it up -- each of them a piece of meaningful machinery in its own right. This first section should have been a (hopefully) reasonably good primer on what one needs to know about \omy, \cds, \scds, and dimensionality to fully understand the results we are about to come upon and the proofs we use to ensure their correctness.


%%%%%%%%%%%%%%%%%%%%%part.tex%%%%%%%%%%%%%%%%%%%%%%%%%%%%%%%%%%
% 
% sample part title
%
% Use this file as a template for your own input.
%
%%%%%%%%%%%%%%%%%%%%%%%% Springer %%%%%%%%%%%%%%%%%%%%%%%%%%

\begin{partbacktext}
\part{The \pwT}
\noindent Now we start properly on the promise we set out to make good on from the start: an honest proof of the \pwt. The previous part, in no means brief and right to the point, was hopefully either a good refresher or first introduction to the necessary machinery we are about to put to use to prove this eponymous theorem. As mentioned before, there are two main parts to this proof: \fix{thing 1, and thing 2}. We now have everything we need to put those together in full gruesome (meant in a good way, of course) detail -- albeit using more modern proofs than originally used by Pila and Wilkie.

We use \fix{idk the source on the 1 for thing 1} first, using \fix{blank or something or the other instead of blank} for \fix{the first part}. Then, we use the similarly recent proof of the result of \fix{fuck me i forget but it was 2021 and was something-Gromov} to get \fix{Resulto numero 2}. From there, \pw itself is far from a stretch, and we will soon have built it up in (almost) all its full spectacle.

\end{partbacktext}

  %%%%%%%%%%%%%%%%%%%%% section.tex %%%%%%%%%%%%%%%%%%%%%%%%%%%%%%%%%
%
% sample section
%
% Use this file as a template for your own input.
%
%%%%%%%%%%%%%%%%%%%%%%%% Springer-Verlag %%%%%%%%%%%%%%%%%%%%%%%%%%
%\motto{Use the template \emph{section.tex} to style the various elements of your section content.}
\subsection{Finally: A New Beginning}
%\sectionmark{Starting Again}
\label{starting_pw} % Always give a unique label
% use \sectionmark{}
% to alter or adjust the section heading in the running head
%\sectionmark{Some Introductions}

One can imagine the collective sighs of all those read through the previous section to get to this, the \emph{exciting} part of the module, almost deafening. Here we will first start our proper discussion of the \pwt (and how we told a \emph{bit} of a fib earlier about the specific formulation we prove), along with some bits and bobs as to how we will prove it, and some previews of what it entails. There are two broad parts to the proof we will discuss -- although each is deserving of a chapter (or perhaps even more) on its own, and so their perhaps a bit more rushed than deserved proofs are outsourced to a section of their own each. For now, we start by dipping our toes into what we've been building to throughout the last entire first part.

\bigskip
\centerline{\rule{0.3333\linewidth}{.2pt}}
\smallskip

\subsection{How We're Going to Prove It}

As mentioned before -- but bears repeating -- our approach to proving \pwt (in our special little case) broadly follows that of Bhardwaj, and van den Dries \cite{bhardwaj_pilawilkie_2022}; a recent paper that only came out this year (should you be reading this in 2022). Their major contribution includes using \sacds to simplify several arguments of the original proof by Pila and Wilkie. They also provide a full treatment of a recent variant \cite{binyamini_yomdingromov_2021} of the (classical) Yomdin-Gromov theorem \cite{gromov_entropy_1987} (again, assuming your readership takes places not too soon after these words reach the page) in 2021, as well as a result of Bombieri and Pila \cite{bombieri_number_1989}, which is also made use of. In our case, we are going to be following the (recent) treatment of what we will come to colloquially call \emph{parameterization} by \cite{binyamini_yomdingromov_2021}.


%\centerline{\rule{0.6665\linewidth}{.2pt}}
%\footnote{Please note that this horizontal line is \emph{not} intended to cause any calming effects, as we have used a \textit{similar} (but distinct) line for previously. Delineation is all we intend this to be used for, and it is all that the reader is permitted to take from it. Observe with caution and most importantly, integrity.}

%\bigskip

\subsubsection{Some Recollections and Definitions}

Let us now begin with a recollection of our setting. We are no longer going to speak to the generality of any ordered field, and instead, an \om expansion $\Rtildefull$ of the ordered field of $\overline{\R}$. Ultimately, the \pwt is about rational points on definable sets -- so taking definable sets over this expansion (supposing points have $n$ dimensions) and intersecting with $\Q^n$, then \pw tells us something about the number of such points we can expect; in short, how \emph{few} of them we should expect up to a certain \emph{height}. This concept was briefly touched upon early in the first part. Still, to recap, we can define certain functions that characterize the height of a number (\`a la multiplicative height for rationals) and then bound the size of this set from above as a function of that height parameter. Without the use of a height, there could, of course, be infinitely-many points, so this concept is crucial (and broadly the \emph{point} of the theorem).

Let's begin by again defining the multiplicative height on $\Q$.
\begin{definition}[Multiplicative Height on $\Q$]
  Suppose some $q = \sfrac{a}{b}$ for $a, b$ coprime, $a, b \in \Z$, and $b \neq 0$. Then, the height function is denoted by $\funcdom{H}{\Q}{\Zgeq{0}}$, and defined
    \begin{align*}
      H(q) = \max{ \left\{ \ \norm{a}, \ \norm{b} \ \right\} }
    \end{align*}
    for $\norm{\cdot}$ simply the absolute value on $\Q$.
  \label{defn:Q_height}
\end{definition}

\subsubsection{A Little Idiosyncrasy}

\begin{svgraybox}
  For reasons of preserving syntax used consistently (rather than some off-handed syntax we have been replacing) throughout this part of the course, we overload the use of $H$ as both a function on the rational numbers (and later just any height function) that maps some $q \in \Q^n$ to its height, \emph{as well as} the integral height bound or value itself. You will often find sentences of the form `fixing some $H \geq 1$, we take $X(S, H)$ to be the set $\{ s \in S \ \colon \ H(s) \leq H \}$'. While a bit confusing, especially at first blush, this choice for syntactic overloading was made due to its persistence throughout Dr. Jones' lectures.
\end{svgraybox}


\begin{corollary}
  Fixing some height, $H$, there are only finitely many $q \in \Q$ with height $H$.
\end{corollary}
\begin{proof}
  This should be clear.
\end{proof}

As usual, we now address the multidimensional-case
\begin{definition}[Multiplicative Height on $\Q^n$]
  Suppose some $q = (\vonevm{q}{n}) \in \Q^n$ for $n \in \N$, all in lowest terms as before, then we define the height function to be the coordinate-wise maximum:
    \begin{align*}
      H(q) = \max{ \left\{ \ H(q_j) \colon 1 \leq j \leq n \ \right\} }.
    \end{align*}
  \label{defn:Qn_height}
\end{definition}

With that in mind, we have the following definition.
\begin{definition}
  Let $\XRn[X]{n}$ and $H \in \N$. Denote the set
    \begin{align*}
      \XQH[H] = \left\{ \ q \in X \cap \Q^n \ \colon H(q) \leq H \ \right\}
    \end{align*}
    and further, refer to the algebraic and transcendental parts of $X$ and their corresponding sets such that we have
    \begin{align*}
      \XQH[H] \setminus \XalgQH[H] = \XtrQH[H],
    \end{align*}
    where we trust the superscript notations to be clear.
  \label{defn:height_sets}
\end{definition}

To rephrase what we said above, the \pwt gives us an upper bound on
  \begin{align*}
    \card{\XtrQH[H]},
  \end{align*}
  with $H$ as a parameter controlling the cardinality of this set. Ideally, we want good bounds on this, especially as $H \to \infty$ -- which quite nicely, \pw gives us.

\subsubsection{Building to \pw}

A special case of this coming from Pila (building on an earlier paper he had worked on with Bombieri) is the following:

\begin{theorem}[A Special Case by Pila]
  Suppose $\funcdom{f}{[0, 1]}{\R}$ is analytic and transcendental, and further that $X = \graph{f}$. Then, for all $\eps > 0$, there exists some $c > 0$ such that for any $H \in \N$,
    \begin{align*}
      \card{\XQH[H]} \leq c \cdot H^{\eps}
    \end{align*}
\end{theorem}

In general, a \defnb set may contain a rational line (with interior, that is), in which case the number of points is enormous. Here is where the idea of algebraic and transcendental parts of $X$, again denoted $\Xalg[x]$ and $\Xtr[X]$, becomes vital. However, we now do you the service of definition, such that anything useful can be done with them.

\begin{definition}[Algebraic Part of $X$]
  The \emph{algebraic part} of some $\XRn[X]{n}$, denoted $\Xalg[x]$, is given by the union of all connected, infinite, $\overline{\R}$-\defnb subsets of $X$. That is, definable in the real field (with parameters).
\end{definition}

By our definition a few above, it is clear then that the set of interest to us is $\XtrQH[H]$, where we take all of $X$ and throw away all the `junk' bits we identify with $\Xalg[X]$. Notice that we assume our $X$ to be definable, but these partitions into algebraic and transcendental sets can be \emph{far} from definable, especially $\Xalg[x]$. Indeed, understanding each of these two sets are different problems from one another, requiring different approaches to say meaningful things. Here, we focus on $\Xtr[X]$.

\subsubsection{The Two Ingredients in Proving the \pwT Proper}

We stated it in the Introduction so as to give something to look forward to, but we now do so again -- without further ado, the \pwt:

  \begin{theorem}[The \pwT]
    Suppose $\XRn[X]{n}$ is a \defnb set (in our \om structure). Then for all $\eps > 0$, there exists some $c > 0$ such that for all $H \in \N$,
      \begin{align*}
        \card{\XtrQH[H]} \leq \bound[H].
      \end{align*}
      \label{thm:pwt}
  \end{theorem}

  Now, this right-hand side might be (should be) looking very familiar from just a moment ago -- but recall that that was for an example on \emph{all} points in the set and a special case. Here, their result is much more general and only pays attention to the transcendental parts of $X$.

  Nice though it would be just to put a quick proof underneath and sate ourselves of this theorem, it does, unfortunately, require a bit more work than that. We are going to go through in the next few sections and prove the necessary constituents to our proof of \pw. The two main bits are as follows:
  \begin{enumerate}

    \item \textbf{The \om bit}: Suppose we have a \defnb set $X \subset (0, 1)^n$ (by normalization) and some $r \in \N$. Then there exists a set of functions, $\{ \vonevm{\phi}{k} \}$ where $\funcdom{\phi_j}{(0, 1)^{\dim{X}}}{X}$ that are $C^r$ \defnb maps such that all derivatives of each map up to degree $r$ are bounded (in modulus) by $1$ and whose union is $X$.
    \label{pw_proof:pt1}

    \item \textbf{The number theory bit}: Let $k, n \in \N$, $k < n$. Then for each $d \in \N$, there exists some $r \in \N$ and $c, \ \eps > 0$ with the property that if $\funcdom{\phi}{(0, 1)^k}{(0, 1)^n}$ is a $C^r$ map with derivatives bounded by $1$ up to order $r$ (note: not necessarily \defnb), and $X = \image{\phi}$, then for $G \in \N$, the set $\XQH[H]$ is contained in at most $c \cdot H^{\eps}$ algebraic hypersurfaces of degree at most $d$. Further, this $\eps \to 0$ as $d \to \infty$.
    \label{pw_proof:pt2}

  \end{enumerate}

In essence, this first point means that definable sets possess a certain type of parameterization, which we will come to know as \emph{cellular $r$-parameterizations}. In reference to the work we follow, the Yomdin-Gromov theorem was mentioned (which was not actually a joint work, but work continued by Gromov, inspired by Yomdin), which essentially shows this result over particular special cases. We use a newer variant, as mentioned before, which will allow us to go on to show that \pw generalizes the previously given -- first to more general \om structures, and then to further generalizations we will note (but not discuss) later on.

\begin{svgraybox}
  Just as a point of note for the reader who may not have caught on, it is this second part of the proof that we have (almost \textit{ad nauseam}) been mentioning having its roots in the work between Bombieri and Pila \cite{bombieri_number_1989}.
\end{svgraybox}

Unfortunately, unlike we have seen with a couple of other multi-part results in Part I, the \pwt does \emph{not} immediately and obviously follow, even once we have these two results -- so that will take a bit of doing in and of itself. So, for the next three odd sections, we will preoccupy ourselves with proving first the \emph{\om bit}, then the \emph{\nt bit}, and finally, pulling it all together. The third and final part of this set of notes will then mention, at a very high level and without much detail, some extensions, generalizations, and applications of the \pwt and how we have seen it being used since its inception.

\subsubsection{Just a Couple of Remarks Before we Begin}

To not faff on too long, we enumerate a few remarks on what we've just started and a bit of what is to come. These remarks will not be crucial to what follows but are nonetheless of significant interest.
\begin{remark}[A Series of Remarks]
\begin{enumerate}
  \item The bound given in the \pwt of $\bound[H]$ is the \emph{best} possible bound. It is already the best in the special case of Pila on curves given above -- and so this implies that $\pw$ cannot be improved over the reals. In general, the proof involves a bit more.

  \item However, we often can do better in `effective' and common examples. The first point merely points out that, in general, we can come up with structures that can do no better than the general case -- but in many particular restrictions, we can do better. A good example is in $\R_{\textrm{exp}}$, where Pila conjectured that we can bound by $c \log{(H)^k}$. As well, Binyamini and Novikov \cite{binyamini_wilkies_2017} prove such a bound for $(\overline{\R}, \ \operatorname{exp}_{[0, 1]}, \ \operatorname{sin}_{[0, 1]})$, the restricted $\operatorname{exp}$ and $\operatorname{sin}$ functions. These sorts of (improved) bounds are known for many classes of sets, though very often they are \emph{not} the \defnb sets of some \om structure, requiring more restriction than we assume here.

  \item The $c$ in the bound $\bound$ is \emph{not} effective -- which is to say, not easily or reasonably describable or calculable. Given that the nature of this work is often \emph{not} computational, this is not necessarily a terrible thing -- although, should it be effective, it would be the odd mathematician to be unhappy about it.

  \item There are various stronger results (eg. bounds on number field degrees over $\Q$, rather than rational points), where the height function is no longer rational but algebraic. For example, consider
      \begin{align*}
        \card{ \ \Xtr (d, H) \ } &= \card{ \ \alpha \in X \ \colon \ [\Q(\alpha) \colon \Q] \leq d, \ H(\alpha) \leq H \ } \\
                                 &\leq \bound[H] \textrm{ when } c = c(d).
      \end{align*}
    Here what we hope for (in nice structures, not in general), is something of the form
      \begin{align*}
        c \cdot d^k \cdot \log{(H)}^{\ell}.
      \end{align*}
    Such bounds also have applications that generally differ from \pw (some of which we will mention later on) To spoil that just a tinge, there is a great deal we can say about elliptic curves (and other algebraic curves, of course) in light of the \pwt.

    \item Finally, there has been a good bit of work put in (and is still being put in) to the \emph{parameterization result}. This was the actual result by Yomdin that Gromov later worked on. As it turned out, of course, Pila and Wilkie could use that result (in conjunction with a good bit else) to prove the theorem this course is primarily interested in. Further improvements to these parameterization results have effective application such as, for example, the \emph{number} of $\vonevm{\phi}{k}$ needed, as a function of $r$, as stated in part (ingredient) \ref{pw_proof:pt1} in our proof of the \pwt. Or, for example, when can we do better than having these functions be $C^r$ -- such as being $C^{\infty}$ with some bound (not necessarily 1, as we have) on derivatives of all orders.
  \end{enumerate}
\end{remark}

If any of these topics seemed of particular interest to you, then it will be a bit of a disappointment that we do  \emph{not} discuss any of them in great detail here -- and those we do, we do so at more of a surface level than the other content presented. Of course, the references provided (and still currently worked on literature in the area) are a great place to look for further information on these topics.

  %%%%%%%%%%%%%%%%%%%%% chapter.tex %%%%%%%%%%%%%%%%%%%%%%%%%%%%%%%%%
%
% sample chapter
%
% Use this file as a template for your own input.
%
%%%%%%%%%%%%%%%%%%%%%%%% Springer-Verlag %%%%%%%%%%%%%%%%%%%%%%%%%%
%\motto{Use the template \emph{chapter.tex} to style the various elements of your chapter content.}
\subsection{The O-Minimality Bit (Parameterization)}
\sectionmark{Basics and Basic Cells}
\label{chap:om_bit} % Always give a unique label
% use \sectionmark{}
% to alter or adjust the chapter heading in the running head
%\sectionmark{Some Introductions}

We initially and have for a bit now been calling this section `the \om bit' to differentiate it from the number-theoretical part that is to come next. However, in the future, and as was alluded to a bit in the last section, this `\om bit' is about \param, and we shall generally go on to refer to this section and result as the `\param result,' or simply `\param.'

\bigskip
\centerline{\rule{0.3333\linewidth}{.2pt}}
\smallskip

\subsubsection{Setting}

In this section, we step back a bit from the specificity of $\R$ and work over a more general \om expansion of an ordered field, $\M = \Mltarithplus$. We let $I = (0, 1)$ the open unit interval, interpreted in our model, $\M$. For the next little while, we have Binyamini and Novikov \cite{binyamini_yomdingromov_2021} to thank for what is to come. We begin with some introduction to the world where we will find ourselves for the next little bit.

\begin{definition}
  A \emph{\bc}, $\CIn[C]{n}$ is a set given by a product of copies of $I$ and singletons containing $0$, such that there are in total $n$ factors of this set product. On such a cell, $C$, a \cont map, $\funcdom{\phi}{C}{I^n}$ is called \emph{cellular} if
    \begin{enumerate}
      \item For $\phi = (\vonevm{\phi}{n})$ with each $\phi_j$ only dependent on the first $j$ coordinates.

      \item For each $j$, and each $(\vonevm{x}{j - 1})$ in the projection of $C$ onto the the first $j - 1$ coordinates, the function $x_j \mapsto \phi_j(\vonevm{x}{j})$ is strictly increasing.
    \end{enumerate}

\end{definition}

This next remark is one to remember, as it is a fact that we will be using and making references too often.


\begin{remark}
  Suppose $\funcdom{\phi}{C}{I^n}$, $\funcdom{\psi}{\pri{C}}{C}$ are both cellular. Then so too is their composition, $\funcdom{\phi \circ \psi}{\pri{C}}{I^n}$.
\end{remark}
Checking that the two necessary conditions hold is rather trivial here and not included. This next bit is mostly just notational.

\begin{definition}[and notation for the norm]
  Given $\XRn[U]{n}$ and $\funcdom{f}{U}{\R^m}$ a $C^r$ map, we set the $r$-norm of $f$ to be
    \begin{align*}
      \rnorm[r]{\ f \ } = \left( \max_{\abs{\alpha} \leq r}{ \sup{ \abs{ \ ( D^{(\alpha)} f_j(x) \ } } } \right) \cdot \cfrac{1}{\alpha !}
    \end{align*}
\end{definition}

Note that the scaling by factorial $\alpha$ is not strictly relevant but more of a convenience. Ultimately, just so it is entirely clear what the point of all this is, what we are aiming to find here is parameterizations into the cellular maps on which these norms are bounded. To that end, consider:

\begin{definition}
  Let $r \in \N$ and $\CIn[x]{n}$ definable. Then a \emph{\cellrparam} (CRP) is a finite set, $\Phi$, of \defnb $C^r$ \cellr maps such that each
    \begin{align*}
      \rnorm[r]{\phi} \leq 1
    \end{align*}
  for each $\phi \in \Phi$, and that the union of their images covers $X$ -- that is,
    \begin{align*}
      \bigcup_{\phi \in \Phi} \image{\phi} = X.
    \end{align*}
\end{definition}
Suppose now we want a \cellrparam between definable sets -- what does that require?
\begin{definition}
  Suppose we have some $\CIn[Y]{m}$, and $\funcdom{f}{X}{Y}$ \defnb. Then. \cellrparam[r] of $f$ is a \cellrparam[r] $\Phi$ of $X$, with the \emph{extra} property that
    \begin{align*}
      \rnorm[r]{f \circ \phi} \leq 1
    \end{align*}
    for all $\phi$ in $\Phi$.
\end{definition}

\subsubsection{Do These Even Exist?}

As is often amusing to those outside of mathematics, and even those \emph{in} maths (at least from the perspective of a graduate student in the area), so often we find ourselves setting up a series of definitions and such defining objects that we wish to study -- with a capstone theorem being that they exist at all. While this may seem the completely backwards way to go about doing things, it is often one that ends up working quite well -- but not always. We're sure, of course, that we don't need to recite the (likely apocryphal) story of the aspiring mathematician who worked for quite a while in this way on H\"older continuous maps with $\alpha > 1$ -- having only before seen the case with $\alpha$ in the unit interval. It was only at his defence that he was asked to show an example of such a \emph{non-constant} map -- which, of course, does not exist. Of course, similar apocrypha exist about anti-metric spaces and other seemingly fun ideas that end off in nothing of much use. Luckily, or perhaps even more accurately, \emph{skillfully} and intuitively from our definitions, it turns out that our setup \emph{does} end up working out for us as desired. We now go on to state and prove the theorem that these objects we have described above do, in fact, exist and are non-trivial.

\subsubsection{Yes -- And Here's a Proof}

Just like with the proof of $\CD$, we are going to have two inductive towers, where we again employ the back-and-forth method to prove the theorem for all $n$. Note as another similarity that the first statement is one about sets and the second of functions.

\begin{theorem}[Parameterization (by Binyamini \& Novikov)]
  %Let $n \in \N$. Just like with the proof of $\CD$, we are going to have two inductive towers that we wish to prove to get the theorem for all $n$. Unlike \textbf{that} proof, however, we don't employ the back-and-forth method we used there, as that will prove unnecessary here. Similarly, though, we see that the first statement is about sets and the second of functions. The two statements in our theorem are
  Let $n \in \N$. Then
    \begin{enumerate}[label={}]
    \item[$\In$ ] If $r \in \N$ and $\CIn[X]{n}$ is \defnb, then $X$ has a \cellrparam.
    \item[$\IIn$ ] If $r, \ m \in \N$ and $\CIn[X]{n}$, $\CIn[Y]{m}$, and $\funcdom{f}{X}{Y}$ \defnb, then $f$ has a \cellrparam.
  \end{enumerate}
  Of course, it is clear that $\IIn$ is the more difficult of the two to prove, and in fact, $\IIn$ can then be used to prove $\In$ (so we have perhaps presented them in a bit of a disordered way, in absence of the reflection with the \CD theorem).
  \label{thm:existence}
\end{theorem}

Before we start the proof, we give provide a quick remark on these \cellrparams, and that is that we can \emph{almost} compose these \cellrparams. This is to say essentially the following:

\begin{remark}
  Suppose $\Phi$ is a \cellrparam of some $\CIn[X]{m}$ and that for each $\phi \in \Phi$, given $\funcdom{\phi}{C}{X}$, we have a \cellrparam $\Psi_{\phi}$ of $C$. What we then would like to do is take all the $\phi$ and compose with all corresponding $\Psi_{\phi}$, and receive cellular maps. We \emph{almost} get this result. For example, taking some $\psi \in \Psi_{\phi}$, We can compute to show
    \begin{align*}
      \rnorm[r]{\phi \circ \psi} \leq c_{n, \ r},
    \end{align*}
    which, although not necessarily 1, is finite and so bounded. Of course, we can \emph{make} this constant to be $1$ by doing a bit of extra parameterization. Start by covering $\pri{C} = \dom{\psi}$ with $(c + 1)^k$ boxes, where we take $k = \dim{\psi}$, we let $c \coloneqq c_{n, \ r}$, and each box is itself a translate of $(0, \sfrac{1}{c})^k$. This is justified due to the fact that on each such box, we have a natural linear map from $(0, 1)^k$ to $(0, \sfrac{1}{c})^k$ -- and in so applying these natural maps, we end up with $c_{n, \ r} = 1$ after the computation, for each $\phi$ and $\psi$. So now, at the expense of ending up with a whole bunch more maps, we end up with a \cellrparam. What we just described -- the process of going from a constant of $n$ and $r$ to 1 using linear maps will hence be referred to as \textbf{linear substitution}. Going on, and in the next proof, we are going to be justifying parts \emph{by linear substitution} quite a bit, so what may have seemed like a brief detour is actually quite important
\end{remark}


\begin{remark}
This may seem completely apropos of nothing, but now and for the rest of these notes, we assume our model, $\M$, to be $\aleph_0$-saturated. If this idea is familiar to you: great. Otherwise, don't worry too much about it. Consider the idea of saturation as somewhat analogous to compactness in a sense; where the specified saturation determines the basis for a particular topological construction. Essentially this assumption allows us just a bit more stringency to get our result more cleanly than we may otherwise. If you want a little more explanation, refer to the snazzy little box below (or up on the top of the next page, as it may be).
\end{remark}

\pagebreak

\begin{svgraybox}
  \textbf{For those a little more interested} \\
  In essence, $\M$ being $\omega$-saturated for $\omega$ a cardinal means that for $A \subseteq M$ with cardinality less than $\omega$, the $A$-\defnb subsets of $M^n$ form a basis for a topology (called the $A$-topology) on $M^n$ which is compact Hausdorff, and for which $A$-\defnb sets are the open and closed sets. We take $\aleph_0$-saturation (allowing only finite, not even countable $A$ sets), but this allows us to use definable selection to get a uniform version of a proposition we need in our proof. Further, but not relevant here, $\aleph_0$-saturated models can be elementarily extended to $\aleph_1$-saturated models, where work can be done that may not be possible in the original model -- but can then be applied back due to being an elementary extension. Saturation is actually quite important if one is to discuss this topic in more generality and without making some of the assumptions we do, but for our purposes, this is a \textit{sweep-under-the-rug} kind of idea.
\end{svgraybox}

Not to keep the eager reader too antsy about getting to the proof of the above theorem, but we are first going to start with a lemma (and then another -- which some may even say is starting with two \lemmas) that will make the proof of $\IIn$ \emph{quite} a bit nicer for ourselves.

\subsubsection{A Few \Lemmas to Get us Going}
\sectionmark{Some Useful Results}

\begin{lemma}
  Let $n, r \in \N$, and suppose that every $\funcdom{F}{\pri{X}}{I}$ \defnb on a \defnb $\pri{X}$ of dimension $n$ has a \cellrparam. Then for every definable $\funcdom{G}{X}{I^m}$ with $m \in \N$ and such that $\dim{X} = n$ has a \cellrparam.
  \label{lem:crp-pt2}
\end{lemma}

\begin{proof}[of Lemma \ref{lem:crp-pt2}]
  Fix some $G = (\vonevm{G}{m})$ and let $\Phi_1$ be a \cellrparam of $G_1$ (which we know we can find by assumption). It is then enough to find a \cellrparam of $G \circ \phi$ for $\phi \in \Phi_1$, and then use \emph{linear substitution} as described it before. So, we can suppose $\rnorm{G_1} \leq 1$. Recall of course that by our assumption, $G_1$ depended only on its first indeterminate -- and so we can similarly get a \cellrparam, $\Phi_2$ of $G_2$. Taking some $\phi \in \Phi_2$ then evidently (as before) $G_2 \circ \phi$ is as desired -- but now we need to worry about $G_1 \circ \phi$. Generically, we get that $\rnorm{G_1 \circ \phi} \leq c_{r, \ n}$. We can then keep doing this for all $3 \leq j \leq m$, and then for each have that $\rnorm{G_j} \leq c_{r, \ n}$. At this point, we can do a whole bunch of our linear substitutions to get a bound of $1$ for each $j$.
\end{proof}

\begin{remark}
  It may not be immediately obvious why the above lemma is useful to us. Principally, its use will be in proving $\IIn$ in the theorem, where it essentially says that we get to treat our $\funcdom{f}{X}{Y}$ for $\CIn[Y]{n}$ equivalently to some $\funcdom{\hat{f}}{X}{\hat{Y}}$ for $\hat{Y} \subseteq I$ -- essentially preventing us from having to have to deal with vectorial image, and only consider the case of one variable in the codomain. This is quite nice indeed.
\end{remark}

\begin{proof}[of Theorem \ref{thm:existence}]
  It should be clear that we induct on $n$ to prove this. The statement $\In[1]$ is immediately obvious from what we've said earlier, so we need to prove $\IIn[1]$ to finish the base case. Unfortunately, this is nowhere near as immediate -- so much so that we appeal to a lemma.

    \begin{lemma}[Gromov]
      Suppose $r \geq 2$, $\funcdom{f}{I}{I}$ \defnb and $C^{r-1}$ with $\rnorm[r-1]{f} \leq 1$. Then $f$ has a \cellrparam.
      \label{lem:gromov_r2}
    \end{lemma}

    \begin{proof}[of Lemma \ref{lem:gromov_r2}]
      First, by applying the \SMt, we can break up into intervals on each of which we assume that our $f$ is $r$-times differentiable -- and further that its $r$-th derivative is strictly monotonic (if the derivative is 0, then there's nothing further needed). Without loss of generality, we assume one of these intervals is $I$ itself. We take $f^{(r)}$ strictly decreasing on $I$, and $f^{(r)} > 0$. Then, by applying the mean value theorem (which we took for granted as part of the clarity of implementing calculus-type concepts in our model), for all $x \in I$ there is some $\xi_n \in (0, x)$ such that
         \begin{align*}
            \cfrac{c_r}{x} &\geq \cfrac{f^{(r-1)}(x) - f^{(r-1)}(0)}{x} & = f^{(r)}(\xi_n) \\
            & &\geq f^{(r)}(x)
          \end{align*}
        for $c_r$ some constant depending on $r$. Let $\funcdef{g}{x}{f(x^2)}$. Then we already have that $\rnorm[r-1]{g}$ is bounded, and so it suffices to check for $r$. Computing this, we get
          \begin{align*}
            g^{(r)}(x) = a + c_r \cdot f^{(r)}(x^2) \cdot x^r,
          \end{align*}
        for $a$ simply the sum of some bounded terms (about which needn't worry). But, by the above,
          \begin{align*}
            f^{(r)}(x^2) \cdot x^r \leq c_r \cdot x^{r - 2}
          \end{align*}
          and since $r$ was presumed to be at least $2$, then this is bounded as desired. So, we get that $\rnorm{g} < c_r$, which we can again make 1 by linear substitution.
    \end{proof}

    Gromov's lemma gave us the case where $r$ was at least 2, so now we handle the case where $r$ is 1.

    \begin{lemma}
      Suppose $\funcdom{f}{I}{I}$ is a \defnb function, and $r \in \N$. Then there exist \defnb $C^r$ functions, $\vonevm{\phi}{k}$ with each given $\funcdom{\phi_j}{I}{I^2}$, whose images cover $\graph{f}$ such that that $\rnorm{\phi_j} \leq 1$, and each coordinate of each $\phi_j$ is either constant or strictly monotonic.
      \label{lem:existence_r1}
    \end{lemma}

    \begin{proof}[of Lemma \ref{lem:existence_r1}]
      By \sm, we can assume that $f$ is $C^1$ on $I$ and either constant or strictly monotonic with derivative $\pri{f}$ sitting in one of the following intervals over $I$:
        \begin{itemize}
          \item  $(\ - \infty, \ 1 \ )$
          \item  $[\ -1, \ 0 \ )$
          \item  $( \ 0, \ 1 \ ]$
          \item  $( \ 1, \ \infty \ )$.
        \end{itemize}
      We can then assume that $0 < \pri{f} \leq 1$ by using $\inv{f}$ in place of $f$ if necessary, or reversing the orientation of $I$ (depending on which of the follow above intervals $\pri{f}$ sits in). Then, we simply apply the previous lemma $r - 1$ times. to get a parameterization, $\Phi$ of $I$ such that $\rnorm{f \circ \phi} \leq 1$, and so we take $\left\{ \ (\phi, \ f \circ \phi) \ \colon \ \phi \in \Phi \ \right\}$ as our parameterization of the graph.
    \end{proof}

    So, we've now proved (through the previous two \lemmas) the parameterization of the graph of $f$, and what we want to prove now is the existence of a \cellrparam. We are now equipped to do so (recall that we are currently proving $\IIn[1]$). It suffices to do this for a \defnb $\funcdom{f}{I}{I}$. We apply the previous lemma to get a parameterization, $\Phi$ of the graph of $f$. We address each $\phi$ in the parameterization; if $\phi \in \Phi$ has strictly decreasing first coordinate, we compose with $1 - x$ (making it increasing), and otherwise leave it be; so when we take the set
      \begin{align*}
        \left\{ \ \phi_1 \ \colon \ \phi = (\phi_1, \ \phi_2) \in \Phi \right\}
      \end{align*}
    we are ensured that it is a \cellrparam of $I$. Now all we need to check is composition with $f$ holding as expected. Recall that $\rnorm{f \circ \phi_1} = \rnorm{\phi_2} \leq 1$, and so we do have a \cellrparam of $f$.

    That may have felt like quite a lot, especially for just the base case, and so again, we introduce our old friend, the horizontal line, in hopes that it brings us a respite in what may be an overwhelming time. If not at all distressed, feel free to ignore it entirely.

    \centerline{\rule{0.6667\linewidth}{.2pt}}
    \smallskip

    Now isn't that \emph{pleasant}? Try not to spend too long, and we will now move on to the inductive stage of this proof.

    We assume that $\In[m]$ and $\IIn[m]$ hold for $m \leq n$. We start by proving the (much) easier of the two parts, $\In[n+1]$. By \cd, we can suppose $\CIn[X]{n+1}$ is a cell, and so either the graph of a function or the space between two functions.

    If the former, that is, $X = \graph{f}$ for $\funcdom{f}{\pri{X}}{I}$ a \cont \defnb function on some $\CIn[\pri{X}]{n}$, then applying $\IIn[n]$ to this function, we get a \cellrparam of $f$, $\Phi$ -- and so taking some $\phi \in \Phi$, we have that $\funcdom{\phi}{\pri{C}}{I^n}$ for $\pri{C}$ some basic cell in $I^n$. Simply set $C = \pri{C} \times \{ \ 0 \ \}$ and define
      \begin{align*}
        \psi &\colon C \to I^{n + 1} \\
        \psi &\colon (x, 0) \mapsto (\phi(x), \ f \circ \phi(x)).
      \end{align*}
    Since $\rnorm{f \circ \phi} \leq 1$, taking all such $\psi$ gives a \cellrparam of $X$ as desired.

    Supposing now instead that $X = (f, \ g)_{\pri{X}}$ is the space between two graphs, we apply $\IIn[n]$ to the map $(f, \ g)$ (note the perhaps confusing notation; the non-subscripted $(f, g)$ is a \emph{map}, and not the space between graphs) we get a \cellrparam, $\Phi$ of $(f, \ g)$. Again, taking some $\phi \in \Phi$ we have that $\funcdom{\phi}{\pri{C}}{I^n}$ for $\pri{C} \subseteq I^n$ a basic cell. Similarly to the last case (and in analogue to our definition of cells), set $C = \pri{C} \times I$ and define
      \begin{align*}
        \psi &\colon C \to I^{n + 1} \\
        \psi &\colon (\pri{x}, x_{n+1}) \mapsto (\phi(\pri{x}), \ g \circ \phi(\pri{x}) \cdot x_{n + 1} + (1 - x_{n+1}) \cdot f \circ \phi(\pri{x}) ).
      \end{align*}
    Since we have already that $\rnorm{g \circ \phi} \leq 1$ and $\rnorm{f \circ \phi} \leq 1$, we easily see that $\rnorm{\psi} \leq c$, which can be changed to a bound of 1 by linear substitution. Thus again, we get a \cellrparam of $X$ -- and now, we have proven $\In[n+1]$ with all cases exhausted. We hate to remind the reader that this was the easy one, and not just by a little bit. Just because this author contends that we all deserve it, \emph{please} once, and perhaps finally, enjoy the following line.

    \centerline{\rule{0.6667\linewidth}{.2pt}}
    \smallskip

    We assume $\In[m]$ for $m \leq n$, and $\IIn[m]$ for $m < n$, and show that $\IIn[n]$ holds to start the induction. But first, we need that uniform version of $\IIn[n-1]$ that we alluded to earlier (should you have read the very snazzy box).

    \subsubsection{A Uniform Version of One of Our Statements}
    \sectionmark{Doing the Induction}

    %\begin{svgraybox}
    % The following was presented as an exercise to the viewer, but we will present it as a proposition in these notes, following our own (hopefully correct) proof.
    %\end{svgraybox}

    As it turns out, it will be necessary (in the way we prove the next bit) to use a \emph{uniform} version of $\IIn[n-1]$, which we will present as a proposition. This is where the importance of our model being $\aleph_0$-saturated comes into play, and one uses this, along with \defnb choice, to prove that the uniform result follows from the non-uniform. We do not include a detailed proof here, but for details on this, the reader is directed to \cite{bhardwaj_pilawilkie_2022}, where this is addressed in Corollaries 7.1 and 7.2. For particularities, the latter portion of Appendix B can be referenced.

    \begin{proposition}[Uniform Version of $(\textrm{II})_{n-1}$]
      Supposing $\IIn[n-1]$ holds, and that $\funcdom{f}{I \times I^{n - 1}}{I}$ is \defnb, there is a partition, $\vonevm{I}{k}$ of $I$ into \defnb sets such that for each $j = 1, \hdots, k$, there is a uniform parameterization of $f$, ($\Phi_j$ a set of \defnb maps given $\funcdom{\phi}{I_j \times C}{I^{n-1}}$ for $C$ a basic cell in $I^{n - 1}$ depending on $\phi$) such that if $a \in I_j$, then $\Phi_{j, \ a} = \left\{ \ \phi(a, \ \hdots) \ \colon \ \phi \in \Phi_j \ \right\}$ is a \cellrparam of $f(a, \ \hdots)$.
      \label{prop:uniform_param_exc}
    \end{proposition}


    \subsubsection{Finally Inducting}

    We are now ready to prove $\IIn[n]$ with that under our belt. We do this by a series of reductions that will make a seemingly more difficult result much more within reach. First, suppose $\funcdom{f}{X}{I}$, with $\CIn[X]{n}$ \defnb. What we want then is a \cellrparam for $f$. By $\In[n]$, we can reparameterize $X$ and work with each chart separately. As such, we can assume that $X$ is a basic cell. If this has any 0 coordinates, we can use $\IIn[m]$ for some $m < n$, and we are done. So, we assume that all coordinates of $X$ are non-zero. This, of course, then means that, because it is a basic cell, it \emph{must} be $I^n$ itself. So, we are now considering the function $\funcdom{f}{I^n}{I}$. This is our first reduction. Our next reduction is going to be showing that we can assume $f$ to be $C^r$ and for each $a$ in our interval, evaluating with $a$ as our first coordinate gives a map (on $I^{n-1}$) with bounded derivative  (that is, $\rnorm{f(a, \ \hdots)} \leq 1$).

    This is where the \emph{uniform} version of $\IIn[n-1]$ becomes useful. We apply the proposition to $\funcdom{f}{I \times I^{n-1}}{I}$ What the \emph{uniform} version told us was that we get \emph{finitely-many} of these parameterizing families, $\Phi_j$ as above, and a partition $(\vonevm{I}{k})$. Going even one level deeper, for each such $I_j$, we can reparameterize (as we did earlier) to assume that each $I_j$ is just $I$. Thus, partitioning and rescaling, we can take each $I_j = I$ (after which point we now drop the $j$ from the notation). We then fix some $\funcdom{\phi}{I \times C}{I^{n-1}}$ for $C = I^{n - 1}$. Notice we ignore the immediately solvable case (as before) where would could just appeal to one of the assumed statements for $\IIn[m]$, $m < n$ in our strong induction. Setting now
     \begin{align*}
       G(\vonevm{x}{n}) = f(x_1, \phi(\vonevm{x}{n}))
     \end{align*}
     by the parameterization from (a good bit) earlier, we get that for each $a \in I$, we have that $\rnorm{G(a, \ \hdots)} \leq 1$. By \scd, there is a \defnb $\CIn[Z]{n}$ having dimension less than $n$, and such that the function restriction $G \ \vert_{I^n \setminus Z}$ is $C^r$. Using $\In[m]$ for $m < n$, we find a \cellrparam, $\Phi_Z$ for $Z$ (which we must be able to since $Z$ has dimension less than $n$). Since $\dim{Z} < n$, if $\phi \in \Phi_Z$ has domain $C$ and $C$ has a 0 coordinate, then we can apply $\IIn[n-1]$ to each $G \circ \phi$, for each $\phi$. To summarize, in particular because it is easy to get lost in this, what we have done is reparameterize $f$ on $Z$ -- and now we go on to worry about the complement of $Z$ in $I^n$.

     That is all to say, we are left again with the restricted function $G \ \vert_{I^n \setminus Z}$ -- which the keen amongst you should notice (the restriction of) is a set in $\In[n]$. Even for the less keen, we know from $\In[n]$ that we can parameterize such sets -- and so the end is (perhaps) within sight! Suppose we have a \cellrparam, $\Psi$ of $I^n \setminus Z$. Fixing some $\psi \in \Psi$, we can define
      \begin{align*}
        t &\colon C \to I^n \setminus Z,
      \end{align*}
    and we presume that $C$ is \emph{not} $I^n$ for reasons of non-triviality.

    We know that, by definition, $\psi$ maps into the space where $G$ is $C^r$, and so $\comp{G}{\psi}$ is $C^r$. For $\alpha \in (\N)^{n-1}$ a multi-index, with order $\abs{\alpha} \leq r$, for any $a \in I$, we have to check that, being cellular (and so depending only on the first-coordinate, $a$),
      \begin{align*}
        \abs{D^{\alpha} \ (\comp{G}{\psi})(a, \ \hdots)} &= \abs{D^{\alpha} \left( \comp{(G(\psi_1(a), \ \hdots)}{(\vmvn{\psi}{2}{n})} \right)}
      \end{align*}
    since $\psi$ is cellular. And so, we can conclude that
      \begin{align*}
        \abs{D^{\alpha} \left( \comp{(G(\psi_1(a), \ \hdots)}{(\vmvn{\psi}{2}{n})} \right)} \leq c_{n, \ r}
      \end{align*}
    since $\rnorm{G(b, \ \hdots)} \leq 1$ for each $b \in Z$ (in particular where $b = \phi_1(a)$) and $\rnorm{\psi} \leq 1$ -- and so as ever, by linear substitution, we can make our $c_{n, \ r}$ equal to 1.

    So, we can assume that $\rnorm{\comp{G}{\psi}(a, \ \hdots)} \leq 1$ for each $a \in I$. We are now so close to \emph{almost} done.

    So, should we replace $f$ with $\comp{G}{\psi}$ for $\phi_1$, we may then assume that $\funcdom{f}{I^n}{I}$ is $C^r$, and that for each $a \in I^{n - 1}$, we have $\rnorm{f(a, \ \hdots)} \leq 1$.

    \subsubsection{A Final Lemma (For This Proof)}

    We now need one quick lemma to finish off (though we will not be proving it until a bit later).
    \begin{lemma}
      Suppose $\funcdom{f}{I^n}{I}$ is \defnb, and $C^1$, and suppose as well that
      \begin{align*}
        \left\vert \ \cfrac{\partial f}{\partial x_j}(x) \ \right\vert \leq 1
      \end{align*}
      for each $x \in I^n$, and $j = 2, \ \hdots, n$. Then, the set of $a \in I$ such that $\cfrac{\partial f}{\partial x_1}(a, \ \hdots)$ is unbounded is finite

      \label{lem:fin_a_unbounded}
    \end{lemma}

    The proof of this lemma is to come later, and for now, we will take this fact for granted.
    %\begin{proof}[of Lemma \ref{lem:fin_a_unbounded}]
    %\end{proof}

    Recalling where we were in our proof, we can take $\N^n$ and order it first by degree and then lexicographically. We take then that $\alpha \in \N^n$ is at least in this ordering such that
      \begin{align*}
        \left\vert D^{\alpha}(f) \right\vert > 1
      \end{align*}
    -- that is, its derivatives are `big.' If there is no such $\alpha$, we are finished! Otherwise, we parameterize $f$ to bound \emph{this} derivative, and such that the resulting (next) function has increased $\alpha$, and continue this inductively. We now show that this is, in fact, something we \emph{can} do -- that is, ensure that we increase $\alpha$ -- and then finish by induction.

    First note that our multi-index, $\alpha$, living in $\N^n$ must have $\alpha_1 \geq 1$. By the above (unproven) lemma, there are only finitely-many $a \in I$ such that $D^{\alpha} f(a, \ \hdots)$ is unbounded. As such, we can use $\IIn[n-1]$ to handle these cases, and so assume that for each $a \in I$, we do have $D^{\alpha} f(a, \ \hdots)$ bounded.

    Now, we define
      \begin{align*}
        S = \left\{ \ x \in \I^n \ \colon \left\vert D^{\alpha} f(x) \right\vert \geq \cfrac{1}{2} \sup_{\pri{x} \in I} \left( D^{\alpha} f(x, \pri{x}) \right) \ \right\}.
      \end{align*}
    By our assumption, we know this supremum to be bounded, and so this set is actually definable (not in the \om sense, just in that it makes sense to define). By \defnb (this time in the \om sense) choice, there is a \defnb $\funcdom{\gamma}{I}{S}$ with $\gamma_1(t) = t$ simply the identity. Consider the map
      \begin{align*}
        t \mapsto (\gamma(t), \ D^{\pri{\alpha}} f (\gamma(t)))
      \end{align*}
    where $\pri{\alpha} = (\alpha_1 - 1, \alpha_2, \ \hdots, \ \alpha_n)$. By $\IIn[1]$, we get a \cellrparam, $\Phi$, of this map. Take some $\phi \in \Phi$, and consider
      \begin{align*}
        G(x_1, \ \hdots, \ x_n) = f(\phi(x_1), \ x_2, \ \hdots, \ x_n ).
      \end{align*}
    This makes sense because, as part of a \cellrparam of a map on the unit interval, $\phi$ maps into the unit interval itself.
    
    \begin{svgraybox}
      A pithy little aside that is just \emph{too} good not to mention here is Dr Jones' comment about how ``one should never differentiate in public'' -- which we find quite amusing.
    \end{svgraybox}

    Computing for $\beta < \alpha$, we get
      \begin{align*}
        \left\vert \ D^{\beta} \ G \ \right\vert < \cnr
      \end{align*}
    for $c_{n, \ r}$ a constant depending on $n$ and $r$ as usual -- so this is fine. What we need to worry about is $\alpha$. When we consider $D^{\alpha} \ G$, we are differentiating $f$ to some order \emph{other} than specified by $\alpha$ (but smaller, which is okay), but we may run into problem cases where we take the $\alpha$ derivative of $f$, and then a bunch of $\phi$s come out. So, computing $D^{\alpha} \ G$, we get some number of terms dependent on $n$ and $r$, and bounded by $\cnr$ plus some term
      \begin{align*}
        \pri{\phi}(x_1)^{\alpha_1} \ \cdot \ \left( D^{\alpha} f \right) \ (\phi_1(x_1), \ x_2, \ \hdots, \ x_n)
      \end{align*}
    with the left-most term coming from taking the derivative of the expression above in $\phi$. By our definition of $S$ and $\gamma$, we have that
      \begin{align*}
        \left\vert \  \pri{\phi}(x_1)^{\alpha_1} \ \cdot \ (D^{\alpha} f) \ (\phi(x_1), \ x_2, \ \cdots, x_n) \ \right\vert &\leq 2 \cdot \left\vert \ \pri{\phi}(x_1) \ \right\vert^{\alpha_1} \ \cdot \left\vert \ (D^{\alpha} f) \ (\gamma(\phi(x_1))) \ \right\vert \\
                    &\leq 2 \cdot \left\vert \ \pri{\phi}(x_1) \ \right\vert \ \cdot \left\vert \ (D^{\alpha} f) \ (\gamma(\phi(x_1))) \ \right\vert
      \end{align*}
    since $\alpha_1 \geq 1$, and $\card{\pri{\phi}} \leq 1$. Thus, we need to bound this right-hand side of the inequality. To do so, we compute
      \begin{align}
        \cfrac{\partial}{\partial x_1} \left( \left( D^{\pri{\alpha}} f \right) \left( \gamma (\phi(x_1)) \right) \right) &= \pri{\phi}(x_1) \cdot \left( D^{\alpha} f \right) \left( \gamma (\phi(x_1)) \right) \label{eqn:param_deriv_1} \\
        &\quad \quad+ \pri{\phi}(x_1) \cdot \sum_{j=2}^{n} \cfrac{\partial \gamma_j}{\partial t} \ (\phi(x_1)) \ \cdot \ \left( D^{{\alpha}^{(j)}} f \right) \ \left( \gamma(\phi(t)) \right)
        \label{eqn:param_deriv_2}
      \end{align}
    note that we ignore the $\pri{\gamma_1}$ since it is just the identity map. We have that $a^{(j)} = \pri{\alpha} + (0, \ \hdots, \ 0, \ 1, \ 0, \ \hdots, \ 0)$ with $1$ at the $j$-th coordinate. We claim that this is sufficient to control (prove the bound) as claimed.

    Recall that $\Phi$ is a \cellrparam of the map taking $t$ to $(\gamma(t), \ D^{\pri{\alpha}} f(\gamma(t)))$, so substituting in our various $\phi$, we get bounded derivatives. That is, the left-hand side of Equation \ref{eqn:param_deriv_1} is bounded by $\Phi$, a \cellrparam of $\left( D^{\pri{\alpha}} f \right)(\gamma(t))$. Similarly, we have that each partial derivative in the sum in Equation \ref{eqn:param_deriv_2} is also bounded, again by $\Phi$, a \cellrparam of $\gamma$ -- and the other incidence of $\pri{\phi}(x_1)$ in Equation \ref{eqn:param_deriv_2} is bounded for the same reason. The final factor in Equation \ref{eqn:param_deriv_2} is finally also bounded because $\alpha^{(j)} < \alpha$, and $\alpha$ was assumed to be minimal such that the derivative exceeded $1$. Hence, putting this all together, we get that the right-hand side of Equation \ref{eqn:param_deriv_1} \emph{must be bounded}; to be clear, that is
      \begin{align*}
        \pri{\phi}(x_1) \cdot \left( D^{\alpha} f \right) \left( \gamma (\phi(x_1)) \right)
      \end{align*}
    is bounded.

    Now what to do with all this? Well, we have that (skipping right past the linear substitution step), that
      \begin{align*}
        \lvrv{ \pri{\phi}(x_1) \cdot \left( D^{\alpha} f \right) \left( \gamma (\phi(x_1)) \right) } \leq 1,
      \end{align*}
      and finally we are finished with the proof.
    \end{proof}

\sectionmark{Nearly There Now}

That is, of course, just a very funny joke we cruelly taunt the reader with -- as if you'll remember correctly, we never actually went on to prove that last lemma, whose proof we left for some poor sob down the road. It's perhaps just unfortunate then that that poor sob is, in fact, ourselves. However, before we prove that lemma directly, we first prove another lemma in what is becoming an increasingly ridiculous chain of \lemmas, each feeding into one other, almost and perhaps with no end in sight. But that's mathematics for you.

\begin{lemma}
  Suppose $\funcdom{f}{M \times I}{I}$ a family of functions mapping from the unit interval into itself a \defnb family of $C^1$ functions. Then, there is a $c > 0$ such that for any $a \in M$, and $B > 0$, the set
    \begin{align*}
      T = \left\{ \ t \in I \ \colon \ \lvrv{ \pri{f_a}(t) > B } \ \right\}
    \end{align*}
    has $\mu(T) < \cfrac{c}{B} \ $, for $\mu$ the sum of lengths of intervals function.
    \label{lem:small_lemma_in_support}
  \end{lemma}

  \begin{proof}[of Lemma \ref{lem:small_lemma_in_support}]
    Consider the two sets whose union compose that in the original statement,
      \begin{align*}
        T_{+} &= \left\{ \ t \in I \ \colon \ \pri{f_a}(t) > B \ \right\} \\
        T_{-} &= \left\{ \ t \in I \ \colon \ \pri{f_a}(t) < -B \ \right\}.
      \end{align*}
    These are both given by finite unions of open intervals, and that number of intervals, by \om, is bounded uniformly in $a$ and $B$. So, each of these intervals has length not exceeding $\frac{1}{B}$ given we have that $\funcdom{f_a}{I}{I}$ (by mean value theorem as before). Of course, then, the result almost immediately follows because we may take our constant to be the number of intervals in these two sets.
  \end{proof}

We \emph{now} use this last lemma to prove the lemma we took for granted during the more arduous proof of parameterization. Because it's been a bit, we recall the theorem (mostly in full) before providing a proof.

  \begin{lemma}[Really Lemma \ref{lem:fin_a_unbounded}]
    Suppose $\funcdom{f}{I^n}{I}$ is \defnb, and $C^1$, and suppose as well that
    \begin{align*}
      \lvrv{ \cfrac{\partial f}{\partial x_j}(x) } \leq 1
    \end{align*}
    for each $x \in I^n$, and $j = 2, \ \hdots, n$. Then, the set of $a \in I$ such that $\cfrac{\partial f}{\partial x_1}(a, \ \hdots)$ is unbounded is finite
  \end{lemma}

  \begin{proof}[of Lemma \ref{lem:fin_a_unbounded}]
    Supposing otherwise, the this set must contain an interval, and we can assume that interval to be $I$ -- that is,
      \begin{align*}
        \pderiv[f]{x_1} (a, \ \hdots)
      \end{align*}
    is unbounded for all $a \in I$. By choice and \sm, we can assume that we get a $C^1$ $\funcdom{\gamma}{(0, \ \infty) \times I}{I^{n - 1}}$ such that for all $B > 0$ and $t \in I$, we have
      \begin{align*}
        \lvrv{ \pderiv[f]{x_j} (t, \gamma_B(t)) } > B,
      \end{align*}
    and is \defnb. Applying the previous lemma (Lemma \ref{lem:small_lemma_in_support}) to the coordinates of $\gamma_B$ and to the family $f(t, \ \gamma_B(t))$, there is a constant, $C$, such that outside a set of measure $\leq \frac{C}{B}$, we have
      \begin{align*}
        \lvrv{ \deriv{t} f(t, \gamma_{B}(t)) } \leq \cfrac{B}{3}
      \end{align*}
    and coordinate maximum,
      \begin{align*}
        \max_{j = 1, \ \hdots, \ n-1} { \lvrv{ \pri{\gamma}_{B, \ j} (t) } } \leq \cfrac{B}{3n}.
      \end{align*}

    Outside of this set (that is, where we have these bounds), we have, by computation, that
      \begin{align*}
        \deriv{t} f(t, \gamma_B(t)) = \pderiv[f]{x_1} (t, \gamma_B(t)) + \sum_{j=2}^{n} \pderiv[f]{x_j} (t, \gamma_B(t)) \cdot \pri{\gamma}_{B, \ j-1} (t),
      \end{align*}
    which is claimed to be less than $\cfrac{B}{3}$. In the sum, we have that the $\pri{\gamma}_{B, \ j-1}(t)$ are each less than $\cfrac{B}{3n}$, and the derivatives in that same product are each less than $1$ -- making the entire sum at most $\cfrac{B}{3}$ -- but the first summand is greater than $B$ regardless, and so we have a contradiction. That is, as soon as this set of small measure isn't everything, this contradiction occurs. Thus, once $B > C$, we run into trouble.
  \end{proof}

  Now with that, we truly are done (this section)! Depending on your particular interests, this rather technical proof may have seemed either garish or quite snazzy -- but either way, we are now left with a powerful result, and the first of the major two we will be using to prove the \pwt. If this sort of technicality is not exactly your wheelhouse -- well, then perhaps the model theory arena isn't the path to go down -- but you might be happy to hear that the next section is a fair bit different, and it would not be inconceivable that those not so fond of this material find themselves quite enjoying what's coming up. Depending on the sort of number theory you study/have studied, some of how we go about things may strike you as a bit odd in the approach to number-theoretic proofs, but only time will tell. Much like we went on to refer to the \emph{\om bit} as \textbf{parameterization}, we will similarly start calling this number-theoretic bit the \textbf{diophantine part} -- which we note that we \emph{do not} capitalize, following the convention of the lectures.

  %%%%%%%%%%%%%%%%%%%%% chapter.tex %%%%%%%%%%%%%%%%%%%%%%%%%%%%%%%%%
%
% sample chapter
%
% Use this file as a template for your own input.
%
%%%%%%%%%%%%%%%%%%%%%%%% Springer-Verlag %%%%%%%%%%%%%%%%%%%%%%%%%%
%\motto{Use the template \emph{chapter.tex} to style the various elements of your chapter content.}
\subsection{The Number Theory Bit (Diophantine Part)}
\label{chap:nt_bit} % Always give a unique label
\sectionmark{Finally Something For the Cool Kids}
% use \sectionmark{}
% to alter or adjust the chapter heading in the running head
%\sectionmark{Some Introductions}

As we near the end (though not too closely, mind) of our journey through a proof of \pw, we find ourselves now at the part that the na\"ive amongst us may have expected to come much sooner -- and that's the number-theoretic part. As mentioned in just the last section, this is also called the \emph{diophantine part}, the treatment of which we will be following from Habegger \cite{habegger_diophantine_2016} as introduced there (though not at the inception of the concept), this didn't investigate the rational points on definable curves as we have been throughout, but rather points that are \emph{very close} to them. \textbf{To be clear, this is not what we are going to be doing} -- but using several of the ideas from the proof in that paper, we get to `skip' our way along in the proof of \pw to looking at points of bounded degree, rather than just rational points.

\bigskip
\centerline{\rule{0.3333\linewidth}{.2pt}}
\smallskip

\begin{svgraybox}
  Again, note that `diophantine' is purposefully left uncapitalized as a stylistic choice to follow the material presented in the course. There is nothing more to be read into it than that.
\end{svgraybox}

We start with a proof sketch before getting into all the minutiae proper and broadly describe how we will proceed throughout this section.

\subsubsection{On Points of Bounded Height}

As we investigate \pbh, we naturally need a height function (just as we did for the rationals, as discussed earlier). We define one as follows: Suppose $q \in \Qbar$, and let $P \in \Z[x]$ the unique irreducible polynomial with $P(q) = 0$ having coprime coefficients and leading coefficient, $a_0 \geq 1$. Then, we define the height function of $q$ as follows:
\begin{definition}[Height of Algebraic Points]
  With the defined variables as above, we let
  \begin{align*}
    H(q) = \left( a_0 \cdot \Pi_{z \in \C \colon P(z)=0} \max{ \{ \ 1, \lvrv{z} \ \} } \right)^{\sfrac{1}{\deg{P}}}
  \end{align*}
\end{definition}

\begin{remark}
  Note, of course, that should we restrict ourselves to \emph{only} $\Q$ rather than all of $\Qbar$, then this definition lines up with the one we gave earlier about rational points. Further, note that an alternative definition is given by embedding $\Q(\alpha)$ in various $p$-adic fields -- although we will not be formally addressing that here. We will be using this idea a bit but not delving into proofs on the matter for reasons of brevity.
\end{remark}

Without enough fuss to prove any of these things, we can say a couple of facts about this height function.

\begin{proposition}[Some Facts About the Algebraic Height Function]
  Let $q, \pri{q} \in \Qbar$. Then
    \begin{enumerate}
      \item $H(q + \pri{q} \leq 2 \cdot H(q) \cdot H(\pri{q}))$
      \item $H(q \cdot \pri{q}) \leq H(q) \cdot H(\pri{q})$
      \item $H(\sfrac{1}{q}) = H(q)$
    \end{enumerate}
    These should all, even if you don't see a proof immediately, feel intuitive if you have a feeling for how this height function is supposed to work out. In particular, the third is a good example that, if it causes confusion, should lead one to look more into height functions as a concept.
\end{proposition}

\subsubsection{A Few More References}

For some further references on this area, see
\begin{enumerate}
  \item Bombieri-Gabler: `Heights in Diophantine Geometry'
  \item Masser: `Auxiliary Polynomials in Number Theory'
\end{enumerate}

\subsubsection{The Diophantine Proposition Proper}

Now, suppose we are given some $\XRn[X]{n}$, $e \geq 1$, and $n \geq 1$. Then we put
  \begin{align*}
    \XeH{e} = \left\{ q \in X \cap \Qbar \ \colon \ H(q) \leq H, [\Q(q) \ : \ \Q] \leq e \right\}.
  \end{align*}

\begin{proposition}[Diophantine Proposition (Habegger)]
    Suppose $k, n, e \in \N$ with $k < n$, and $d \geq (e + 1) \cdot n$ are unique. Then there exist $c, \ \eps > 0$, $r \in \Z$ with the property that, supposing $\funcdom{\phi}{(0, 1)^k}{(0, 1)^n}$ has $\lvrv{D^{\alpha} \phi } \leq 1$ for all $\alpha \in \N^k$ with $\norm{\alpha} = r$, and $X = \image{\phi}$, then for any $H \geq 1$, the set $\XeH[H]{e}$ is contained in the union of at most $C \cdot H^{\eps}$ hypersurfaces of degree at most $d$. Further, as $\eps \to \infty$, $d \to 0$.
    \label{prop:dioph}
\end{proposition}

You'll (or rather, you should) recognize this as a very similar statement to earlier -- the only difference now is that we are taking algebraic points of bounded height rather than restricting them to rational points. Also, note that \emph{this} is that second `ingredient' to which we've been referring for quite some time now; this is the second part we're going to use to put together the \pwt, and we're going to do so in such a way that we get to skip some of the steps found in the original proof, that necessitated proving the statement first for rational numbers, and then for algebraic numbers. Here, in one fell swoop, we get \emph{all} algebraic numbers due to advances in the proofs used in the years since Pila and Wilkie's original result.

The `proof' here will not be as rigorous as in the previous section (and whether you breathed a sigh of relief there or one of disappointment says quite a lot) and acts more like a sketch of a full proof. It will still be extensive enough to hold our attention for a (hopefully good) long while. In part, the choice to not go through with a full proof is that this part of the proof of \pw is not predicated or reliant on \om -- and so, in the scheme of this course, falls on a less important rung of the latter than bother other and later elements.

We start with a spot of notation. For $d \geq 0$, and $n \geq 1$, set $D_n(d) = \binom{n+d}{n}$. In the plethora of ways to interpret this value, we are interested in it to count the number of monomials in $n$ indeterminates of total degree not exceeding $d$. The following two propositions are stated without proof.

\begin{proposition}
  Suppose $q \in \Q$ has $H(q) \leq H$ and $q \neq 0$. Then $\norm{q} \geq \frac{1}{H}$.
\end{proposition}

We trust you can see why a proof felt unnecessary here. The following lemma generalizes this idea.

\subsubsection{Some Results on Bounds}

\begin{lemma}
  Suppose $q \in \R^n$ with $[\Q(q) : \Q] \leq e$, $H(q) \leq n$, and $f \in \Z[x_1, \ \hdots, \ x_n]$ has degree not exceeding $d$, with $f(q) \neq 0$. Then
    \begin{align*}
      \lvrv{f(q)} \geq \cfrac{1}{(D_n(d) \cdot \lvrv{f} \cdot H^{\alpha \cdot n})^e}.
    \end{align*}
\end{lemma}

One would be forgiven for not immediately seeing that this generalizes the previous proposition or immediately how to go about proving it -- but rest assured, dear reader, that both these facts are true. While we do not provide a proof here, the reader is directed to Habegger \cite{habegger_diophantine_2016} as before. At the same time, we may discontinue saying so as we go on. If something is even not fully (or even just not \emph{well}-explained, then Habegger's paper is a wonderful resource on the matter).

Ultimately, what we want out of this theorem is polynomials (these hypersurfaces) that gives the points (by vanishing) on exactly $\XeH[H]{e}$ -- the manner of which we do so being by showing that these sets are smaller than
  \begin{align*}
    \cfrac{1}{(D_n(d) \cdot \lvrv{f} \cdot H^{\alpha \cdot n})^e},
  \end{align*}
and so the presumption of $f(q) \neq 0$ fails at these points.

The second idea that we're going to be using here is the following lemma:
\begin{lemma}
  Letting $M, N \in \N$, $M < N$, and $A$ an $M \times N$ matrix with rows denoted by $\vonevm{a}{m}$ such that each has 2-norm, $\twonorm{a_j} \geq 1$, we set $\Delta = \Pi_{j=1}^{n} \twonorm{a_j}$. If $Q \geq 2 \sqrt{N} \Delta^{\sfrac{1}{N}}$, then there exists $f \in \Z^N \setminus \zeroset$ with
    \begin{align*}
      \norm{f} \leq Q \textrm{ and } \norm{A \cdot f} \leq (2 \sqrt{N})^{\sfrac{N}{M}} \cdot Q^{1 - \sfrac{N}{M}} \cdot \Delta^{\sfrac{1}{M}}
    \end{align*}
\end{lemma}

Depending on the amount of algebraic and transcendental number theory you've been exposed to, the thought that may immediately jump to mind is that of Minkowski's theorem -- and you be entirely correct to do so. If feeling up to it, give it a go now, and we will later come back and give a proof sketch of this lemma. We now are getting very close to what we actually want.

\sectionmark{Something Similar, Again and Again}

\begin{proposition}
  Suppose $k, d, e, n, b$ are positive integers with $k < n$, $D_n(d) \geq (e + 1) D_k(b)$. Then there is some $c > 0$ with the following property: Let $\funcdom{\phi}{I^k}{I^n}$ a $C^{b + 1}$ function with $\norm{\phi^{(\alpha)}(x)} \leq 1$ for all $\norm{\alpha} \leq b + 1$ and $x \in I^{k}$ with $X = \image{\phi}$. Then, we have that for any $H \geq 1$, there is some $N \leq c \cdot H^{(k+1) \cdot n \cdot e \cdot \frac{d}{b}}$ and polynomials $\vonevm{f}{N} \in \Z[\vonevm{x}{n}] \setminus \zeroset$ of degree $\leq d$ such that if $q \in \XeH{e}$, then $f_j(q) = 0$ for some $j$.
  \label{prop:pre_dioph}
\end{proposition}

Notice how close this comes to actually being what we want to prove. As we said before, we will come back to prove this proposition (sort of) later on, but for now, we will go ahead and give a proof of the diophantine proposition from everything we've seen so far (proved or otherwise).

\subsubsection{Proving the Diophantine Proposition, Supposing the Hard Part}

\begin{proof}[of Diophantine Proposition (Proposition \ref{prop:dioph})]
  Suppose $k, n, e, $ and $d$ with $k < n$ and $d \geq (e + 1) \cdot n$ are all positive integers. We have $D_k(b)$ strictly increasing in $b$, so then there must be a unique $b$ such that
    \begin{align*}
     (e + 1) \cdot D_k(b) \leq D_{k + 1}(d) < (e + 1) \cdot D_k(b + 1).
   \end{align*}
  Fixing this $b$, we do a bit of computation using the bound on $d$ in our assumptions to get that
    \begin{align*}
      e + 1 > \cfrac{d + 1}{k + 1} \cdot \left( \ \cfrac{d}{b} \ \right)^{k}
    \end{align*}
  and so, we can stare at this long and hard enough that we end up rearranging and getting that
    \begin{align*}
      \cfrac{d}{b} < \left( \cfrac{(e + 1)(k + 1)}{d + 1} \right)^{\frac{1}{k}}.
    \end{align*}
  We can then see that as the right-hand side, $\left( \cfrac{(e + 1)(k + 1)}{d + 1} \right)^{\frac{1}{k}} \to \infty$, as $d \to 0$. Then, applying the above proposition with $r = b+1$, $\eps = (k+1) \cdot n \cdot e \cdot \frac{d}{b} \to 0$ as $d \to \infty$.
\end{proof}

What we did there was then a bit unfair -- take a proposition that was \emph{almost} what we wanted, skipped out on our bill of a proof, and then basically use that to prove our desired theorem. To the end of making things up to the undoubtedly angry waitstaff of Proposition \ref{prop:pre_dioph}, we provide at the very least a reasonable sketch of the proof of the proposition.

\subsubsection{Proving the Hard Part}

\begin{proof}[sketch of Proposition \ref{prop:pre_dioph}]
  Suppose $H \geq 1$, $c$, $\pri{c}$, etc. all independent of $H$. Set our $r$ to be given
    \begin{align*}
      r = \cfrac{\pri{c}}{H^{\frac{k + 1}{k} \cdot n \cdot e \cdot \frac{d}{b}}}
    \end{align*}
  where $\pri{c}$ is small. We have that $I^k$ is contained in the union of $N \leq (1 + \frac{1}{r})^k \leq 2^k \cdot (\pri{c})^{-k} \cdot H^{(k+1) \cdot n \cdot e \cdot \frac{d}{b}}$ closed boxes of side length $r$. Notice that $2^k \cdot (\pri{c})^{-k}$ is a $c$, and in fact, the boundary one on $N$. So now, this $N$ is going to be the $N$ as given in the proposition (i.e. the number of necessary polynomials), and its bound is of this form. Now, we just need to find a polynomial that works on each box generically.

  Let $V$ be any such box. We can find $f = f_V$ such that $f(q) = 0$ for $q \in \XeH[H]{e}$, $q = \phi(z)$ for some $z \in V \cap I^k$. We then can vary $V$ to get $\vonevm{f}{N}$. To explicitly find these $f$, we write
    \begin{align*}
      f(\vonevm{x}{n}) = \sum_{\norm{j} \leq d} f_j \cdot x_1^{j_1} \cdot \ \cdots \ \cdot x_n^{j_n}
    \end{align*}
  for coefficients $f_j \in \Z$ to be determined. Fix a point $a \in V \cap I^k$. Now, for $\alpha \in \N^k$ with $\norm{\alpha} \leq b$, let $A_{\alpha}$ be the vector given by
    \begin{align}
      \cfrac{r^{b - \norm{\alpha}}}{\left\vert\left\vert \left( \cfrac{D^{\alpha} \phi^{j} (a)}{\alpha !} \right)_{\norm{j} \leq \alpha} \right\vert\right\vert_{2} } \cdot \cfrac{D^{\alpha} \phi^{j}(a)}{\alpha !}
      \label{eqn:nasty_1}
    \end{align}
  where $\phi^{j} = \phi_1^{j_1} \cdot \ \cdots \ \cdot \phi_n^{j_n}$. We get $M = D_k(b)$ rows given by $A_{\alpha}$, $N = D_A(d)$ the columns, we may apply the lemma proved (pretty far) above to this matrix, for which we need to know $\Delta$ and $Q$ -- which a behind-the-scenes calculation shows that we can take to be
    \begin{align*}
      \Delta &= r^{\frac{-b}{k+1} \cdot D_k(b)}  \\
      Q &= r^{- \cfrac{b + k + 1}{(e + 1)(k + 1)}}.
    \end{align*}
    We can now verify that $Q \geq 2 \sqrt{N} \Delta^{\frac{1}{N}}$, supposing that th  $\pri{c}$ we started with is sufficiently small. By the lemma, there are $\{f_j\}$ not all zero with $\norm{f} \leq Q$ and 
      \begin{align*}
        \norm{A \ f} \leq \cfrac{c \cdot \Delta^{\sfrac{1}{D_K(b)}}}{Q^c}.
      \end{align*}
      Recall that $\norm{A \ f}$ was that nasty sum before, expression (\ref{eqn:nasty_1}). What we now wish to do is show that this choice of $f$ \emph{works}. We start by setting
        \begin{align*}
          \funcdef{g}{t}{f(\phi_1(z), \ \hdots, \ \phi_n(z))}.
        \end{align*}
      The intention now is to show that the modulus of $g$ is small compared to the sort we saw in the Liouville-type lemma earlier, since if $z$ is in a `little' box such that $\phi(z)$ is algebraic of degree at most $e$ and height at most $H$, then we must have $g(z) = 0$. We can start at this by looking at the Taylor expansion of $g$. This gives us
        \begin{align*}
          g(t) = &\sum_{\norm{\alpha} \leq b} \left( \sum_{\norm{j} \leq d} \cfrac{f_j \cdot D^{\alpha} \phi^{j}(a)}{\alpha !} \right) (z - a)^{\alpha} \\
                  &\quad \quad + \sum_{\norm{\alpha} = b+1} \left( \sum_{\norm{j} \leq d} \cfrac{f_j \cdot D^{\alpha} \phi^{j}(\xi_{z}) }{\alpha !} \right) (z - a)^{\alpha}
        \end{align*}
      where $\xi_{z}$ is on the line segment connecting $a$ and $z$. We have that
        \begin{align*}
          &\lvrv{ \sum_{\norm{\alpha} \leq b} \left( \sum_{\norm{j} \leq d} \cfrac{f_j \cdot D^{\alpha} \phi^{j}(a)}{\alpha !} \right) (z - a)^{\alpha} } \\
          & \quad \quad \quad \quad \leq c 
          \cdot \cfrac{\Delta^{\frac{1}{D_k(b)}}}{Q^e} \cdot r^b \cdot \sum_{\norm{\alpha} \leq b} \left( \sum_{\norm{j} \leq d} \cfrac{D^{\alpha} \phi^{j}(a)}{\alpha !} \right)^{\frac{1}{2}}.
        \end{align*}
      Due to our bound on derivatives, all our Euclidean terms (the right-most sums) are small, and so we get that the above is less than
        \begin{align*}
          c \cdot r^{\sigma}
        \end{align*}
      for
        \begin{align*}
          \sigma = e \cfrac{b + d + 1}{(e + 1)(k + 1)} + \cfrac{b \cdot k}{k + 1}.
        \end{align*}
      For the remainders in the series, we get
        \begin{align*}
          \norm{ \ \cdot \ } \leq c \cdot \norm{f}_{\infty} \cdot r^{b + 1}
        \end{align*}
      and since each $f$ has norm bounded above by $Q$, we again get the bound
        \begin{align*}
          \norm{ \ \cdot \ } \leq c \cdot r^{\sigma}
        \end{align*}
      for the remainder in the Taylor series. In whole, this gives us that 
        \begin{align*}
          \norm{g(z)} \leq c \cdot r^{\sigma}.
        \end{align*}
      
      Supposing now that $q = \phi(z) \in \XeH[H]{e}$ and $z \in V$. Then, if $f(q) \neq 0$, by the Liouville-type lemma on heights, we have $f$ bounded \emph{below} by something of the form
        \begin{align*}
          f(q) \geq \cfrac{1}{\left(D_n(d) \cdot Q \cdot H^{d \cdot n}\right)^e}
        \end{align*}
      (by $\norm{f} \leq a$), and so then
        \begin{align*}
          c \cdot r^{\sigma} \geq \cfrac{1}{\left(D_n(d) \cdot Q \cdot H^{d \cdot n}\right)^e}
        \end{align*}
      then rearranging, 
        \begin{align*}
          \cfrac{1}{c D_n(d)} &\leq r^{\sigma} \cdot Q^e \cdot H^{d \cdot n \cdot e} \\
                              &= {c^{\prime}}^{\frac{b \cdot k}{k + 1}}
        \end{align*}
      the equality part of which should be noted being \emph{independent} of $H$. Thus, taking $\pri{c}$ sufficiently small, this inequality cannot hold with necessity -- and so we reach a contradiction. It's been a long ride, so the reader would be forgiven for having forgotten we were even seeking out a contradiction -- but either way, this then implies the theorem by the assumption shown to give rise to contradiction; namely, that $f(q) = 0$, and so we can vary over all the points we need it to, and then we can vary over all the boxes to get each of the $f_j$ in the proposition.
\end{proof}

That was quite a lot, especially for those uninducted into the hallowed halls of these monstrosities of proofs -- but in fact, the techniques used here, namely to show that some polynomials must vanish at certain points given sufficient smallness, are quite common and would not be a bad tool to add to one's repertoire. And with that, we are now finished with the diophantine part of the proof going into the \pwt! This means we are now ready to face the beast itself in the (near) final part of this lecture series, where we are to prove a \emph{little bit} different version of the \pwt than we have been talking about throughout -- where instead we prove the statement for \emph{families} of \defnb sets, rather than just singular \defnb sets. After that, some talk about applications, further improvements, and active research areas on this and surrounding topics will be discussed, all more relaxed. But for now, let's take out a second, well-deserved pat on the back for getting this close to the finish line.
  %%%%%%%%%%%%%%%%%%%%% section.tex %%%%%%%%%%%%%%%%%%%%%%%%%%%%%%%%%
%
% sample section
%
% Use this file as a template for your input.
%
%%%%%%%%%%%%%%%%%%%%%%%% Springer-Verlag %%%%%%%%%%%%%%%%%%%%%%%%%%
%\motto{Use the template \emph{section.tex} to style the various elements of your section content.}
\subsection{Sticking it all Together}
\sectionmark{The End of our Journey (at last)}
\label{chap:chap:all_together} % Always give a unique label
% use \sectionmark{}
% to alter or adjust the section heading in the running head
%\sectionmark{Some Introductions}

Here, we start by introducing the \pwt for \emph{families} of \defnb sets, which is what we will actually be proving (and, of course, implies the version for a single \defnb set), and then going on to provide a full proof of how the last two sections' results (the \emph{parameterization} and \emph{diophantine} results) allow us to come to the theorem we've been holding our breath for this whole time. For the strong-lunged amongst us, this will be a nice capstone to everything we've seen so far and should hopefully pull everything together in a manner that is both satisfying and elucidative for all who have made it to this point.

\bigskip
\centerline{\rule{0.3333\linewidth}{.2pt}}
\smallskip

\subsubsection{The Familial Version of the \pwT}
  Suppose that we are now working over $\Rtilde = (\overline{\R}, \ <, \ \hdots)$, an \om expansion of the real field.
  \begin{theorem}[\pw]
    Suppose $X \subseteq \R^m \times \R^n$ is \defnb, and let $e \geq 1$, $\eps > 0$. Then, there is a $c > 0$ such that for all $a \in \R^m$ and $H \geq 1$,
      \begin{align*}
        \card{ \  \XtraeH \ } \ \leq \ c \cdot H^{\eps}.
      \end{align*}
\end{theorem}

The proof we give follows Bhardwaj and van den Dries \cite{bhardwaj_pilawilkie_2022}, which has one relatively easy but fundamental lemma.

\subsubsection{A Fundamental Lemma}

\begin{lemma}
  Suppose we have some $S \subseteq \R^n$ is \sa (\defnb in the real field, $\overline{\R}$), and a map, $\funcdom{f}{S}{\R^m}$ which is \sa and \inj. If $X \subseteq S$ such that $\funcrestr{f}{X} \colon X \to Y = f(X)$ is a homeomorphism, then $f(\Xalg[X]) = \Xalg[Y]$ and so $f(\Xtr[X]) = \Xtr[Y]$.
  \label{lem:pw_key_lemma}
\end{lemma}

\begin{proof}[of Lemma \ref{lem:pw_key_lemma}]
  It is clear that when we map $\Xalg[X]$ under $f$ it is contained in $\Xalg[Y]$, so what suffices to be shown is that all of $\Xalg[Y]$ is mapped to by $\Xalg[X]$. We argue this based on the definition of \saty. Suppose we have some $C \subseteq Y$ a connected, infinite, \sa set. Then, as $f$ is invertible on $Y$, we can consider $\inv{f}(C)$, which must be \sa by both $f$ and $C$ \sa. Further, by \injtvty of $f$, we have that $\inv{f}(C)$ is contained in $X$, and also that it is connected by $C$ connected and $f$ continuous and infinite. Thus, it is contained in an algebraic box of $X$, and so $\inv{f}(C) \subseteq \Xalg[X]$. So, $\inv{f}(\Xalg[Y]) \subseteq \Xalg[X]$ and $f(\Xalg[X]) \subseteq \Xalg[Y]$ -- and so clearly they are equal.
\end{proof}


\subsubsection{Our Final Proof}

We are now prepared to prove \pw -- something that may feel like it deserves a bit more fanfare than we are giving it presently; however, since our calming horizontal lines are already bravely serving their purpose on the battlefields of the more painful of proofs, we hope the reader can be sufficiently emboldened and inspired by a short vertical line.

\medskip
\centerline{$\mid$}
\medskip

Now you don't see one of those every day, do you? With spirits now ablaze, we begin with our final big proof of this set of notes. In truth, this is not so difficult a proof and will more be an exercise on the part of you, our dear reader, in recalling the two ingredients of the last two sections. You would not be ill-advised to go back and just recall the final result of each (the proofs are not so important) and the uniform version of the result, if appropriate (as is so for parameterization).

\begin{proof}[of the \pwT]
  We will, of course, (and \emph{all together now}) proceed by induction on $n$, the dimension of the ambient fibre space. As we saw in the number-theoretic proofs, we are going to perform a series of reductions that will make the proof more tractable and reduce it to a case that we can use our known results to conclude. This first reduction will be to show that we can reduce our discussion from subsets of $\R^n$ to subsets of $I^n$, which is quite fine indeed. To do so, we take the map for $x \in \R$ given $x \mapsto \pm x^{\pm 1}$ and take images of sets in $\R^n$ under these maps. Since these preserve being algebraic of degree at most $e$ and with height at most $H$, this is fine to do and keeps us in $[0,  \ 1]^n$ as then, by induction, we can handle the faces of the box, leaving us in $I^n$ as we desired. This, of course, sets us up to use the parameterization result. Note that this means we can also assume the parameter space, $\R^n$ is just $I^n$ (what we said above was just about subsets, but it applies to the whole space), and similarly for $\R^m$. That is, our $X \subseteq I^m \times I^n$; this completes the first reduction.
  
  Using the diophantine proposition from the previous section (we won't recall it here for the sake of brevity) for each $k < n$, we can take $d$ so large that for whatever $\eps (k, n, d, e)$ (given by the diophantine proposition) is less than $\eps / 2$ (for the $\eps$ given in the setup of \pw). This is a bit confusing, we realize but bear with us. We then take $r, c$ to be the maximum of the resulting $r$ and $c$ given by each application of the diophantine proposition. Then, using that $r$, we apply the parameterization result (the uniform version), and so get some $N$ such that each fibre, $X_a$ for $a \in I^m$, has a \cellrparam consisting of no more than $N$ maps. Note that $N$ depends on $X$, $e$, and $\eps$ (the \pw $\eps$).
  
  Consider now just \emph{one} of these maps in some \cellrparam for some $X_a$. The diophantine proposition tells us that in the image, the points of degree at most $e$ and height at most $H$ lay on at most $c \cdot H^{\frac{\eps}{2}}$-algebraic hypersurfaces. Putting this all together, the number of such points for $a \in I^m$ and $H \geq 1$ lay on at most $N \cdot c \cdot H^{\frac{\eps}{2}}$-algebraic hypersurfaces -- or equivalently,
    \begin{align*}
      X_a(e, \ H) \subseteq \textrm{union of at most } N \cdot c \cdot H^{\frac{\eps}{2}} \textrm{ hypersurfaces of degree at most } d.
    \end{align*}
    So, it suffices to show that there exists some $c$ such that if a polynomial $P \in \R[x_1, \ \hdots, \ x_n] \setminus \zeroset$ has degree $\leq d$, then transcendental part of the zero-locus of $P$ intersected with the fibre has fewer than $c \cdot H^{\frac{\eps}{2}}$ points. That is, 
    \begin{align*}
      \card{ \ (X_a \cap \mathbb{V}(P) )^{\textrm{tr}} \ } \ \leq \ c \cdot H^{\frac{\eps}{2}},
    \end{align*} 
    as then we have
    \begin{align*}
      \card{ \ X_a^{\textrm{tr}} (e, \ H) \ } \ &\leq \ N \cdot c \cdot H^{\frac{\eps}{2}} \ \cdot \ c \cdot H^{\frac{\eps}{2}} \\
                                                &\leq \ c \cdot H^{\eps}
    \end{align*}
    which is our desired result.
    
    To prove this sufficiency, we let $F$ be the set of all such non-zero polynomials in $n$ indeterminates of degree at most $d$, viewed as an $\overline{\R}$-\defnb set. Then, take $\varv \subseteq F \times \R^n$ to be the family of all $\overline{\R}$-\defnb hypersurfaces of degree at most $d$ whose fibre over $P$ is just the zero-locus of $P$. That is, $\varv_P = \mathbb{V}(P)$ for each $P \in F$. Now, we may apply \cd in the real field to get \sa sets, $\vmvnsupbr{C}{1}{k} \subseteq F \times \R^n$, such that the union of their fibres over $P$ is $\varv_P$ and for each $j$, there is an $i^{(j)} \in \{0, \ 1\}^n$ not all $1$ (so as not to have interior), such that for each $P \in F$, the fibre $C^{(j)}_P$ is either empty, or an $i^{(j)}$-cell. This idea should be very familiar now, we are just seeing it in a slightly different context than we have previously. 
    
    Now, for $j$ and $\isupj$ as above, we define the natural projection,
    \begin{align*}
      \pi_{\isupj} \colon \R^n \to \R^{n_j}
    \end{align*}
    for $n_j$ the number of non-zero coordinates in $\isupj$. As before, this is just the projection away from the zero-coordinates of the cell (this should again be familiar), which is a homeomorphism onto its image. This, of course, preserves the degree and height of points, so we needn't worry about that. Now, we appeal to that lemma we proved before starting. Let $Y^{(j)} \subseteq I^m \times F \times \R^{n_j}$ be the \defnb family such that for some $(a, P) \in I^m \times F$,
    \begin{align*}
      Y^{(j)}_{(a, P)} = \pi_{\isupj} \left( X_a \cap C_{P}^{(j)} \right).
    \end{align*}
    By Lemma \ref{lem:pw_key_lemma}, the transcendental part of $Y^{(j)}_{(a, P)}$ is equal to the image under $\pi_{\isupj}$ of the transcendental part of $\left( X_a \cap C_{P}^{(j)} \right)$ -- and since these projections preserve the degree and height of points, this is also true of the unprojected set.
    
    We also have that 
    \begin{align*}
      (X_a \cap \mathbb{V}(P))^{\textrm{tr}} \ \subseteq \ \bigcup_{j=1}^{k} \ (X_a \cap C_P^{(j)})^{\textrm{tr}}
    \end{align*}
    for each $(a, P)$. By the induction hypothesis, we can do our counting by the $(Y_{(a, P)}^{(j)})^{\textrm{tr}}$ -- and by taking the maximum over all $j$, 
    By the induction hypothesis, there is a $c > 0$ such that for each $j$ and each $(a, P)$, and $H \geq 1$, we have
    \begin{align*}
      \left\vert \ \left( Y^{(j)}_{(a, P)} \right)^{\textrm{tr}} (e, \ H) \ \right\vert \ \leq \ c \cdot H^{\frac{\eps}{2}}.
    \end{align*}
    Hence, for each $(a, P)$ and $H \geq 1$, we have
    \begin{align*}
      \card{ \  (X_a \cap \mathbb{V}(P))^{\textrm{tr}} (e, \ H) \ } \ &\leq \ k \cdot c \cdot H^{\frac{\eps}{2}} \\
                                                                      &\leq \ c \cdot H^{\frac{\eps}{2}}
    \end{align*} 
    by independence of $k$ from $H$. But notice now that this is exactly what we determined to be sufficient, and so our proof is complete!
    \smartqed
\end{proof}

As you can see, what really makes this proof \emph{tick} is this method of \sacd, combined with the lemma we proved that allows us to maintain the partition of algebraic from transcendental points, given particular conditions on our function. While it may be a bit disingenuous to call this proof `simple,' it really is a simplification of the original proof given by Pila and Wilkie. 
  
%%%%%%%%%%%%%%%%%%%%%part.tex%%%%%%%%%%%%%%%%%%%%%%%%%%%%%%%%%%
% 
% sample part title
%
% Use this file as a template for your input.
%
%%%%%%%%%%%%%%%%%%%%%%%% Springer %%%%%%%%%%%%%%%%%%%%%%%%%%

\section{Part III: That Which Has Been Neglected}

\noindent It really says it all on the tin; this is a brief, final section where we point out some parts from the lecture not included (importantly, the applications lecture), and wrap everything up.

We point out some points \emph{post}-proof, as we will describe this new era in which we find ourselves, that these notes do \emph{not} discuss, but were included in the lectures. This is simply for the sake of completeness in light of this incompleteness and not to taunt our dear reader with all that they are missing out on. Although if the latter reasoning helps the reader deal with this loss, then they are more than welcome to accept it as the factual and intended inclusion of this section (more of a paragraph or two, really).

  %%%%%%%%%%%%%%%%%%%%% chapter.tex %%%%%%%%%%%%%%%%%%%%%%%%%%%%%%%%%
%
% sample chapter
%
% Use this file as a template for your input.
%
%%%%%%%%%%%%%%%%%%%%%%%% Springer-Verlag %%%%%%%%%%%%%%%%%%%%%%%%%%
%\motto{Use the template \emph{chapter.tex} to style the various elements of your chapter content.}

\subsection{Some References and Applications}
\sectionmark{What We Don't See}
\label{chap:unfinished}

And that's about it, really. After this point in the course, there were two main applications discussed and in relatively (but not excessively) considerable detail. For the reader's sake, we do include the references given in the lectures, should they be interested in straying away on their own. These are
  \begin{itemize}
    \item Thomas Scanlon (\textit{Note: Not T.M. Scanlon, despite sharing a given and family name}) who has written a number of papers in this area, but whom we cite for his survey paper of the topic in general \cite{Scanlon2012CountingSP}.
    \item Philipp Habegger (who you'll recall from before) and Martin Orr, each of whom has writings in the proceedings from a summer school at the University of Manchester (and for which we do not have a citation here).
  \end{itemize}

There is a good bit of discussion of elliptic curves and manages to cover such theorems as the Ax-Lindeman-Weierstrass theorem, the Manin-Mumford conjecture, and the Andr\'e-Oort conjecture. Further on, there is some discussion of results to do with Galois bounds (bounds on Galois orbits of torsion points) broadly following results originally devised by Schmidt \cite{schmidt2018} (but relying on \pw, of course). The applications lecture is chock full of wonderfully interesting results that are omitted here only for reasons of time (on the part of the authorial team) and not relevance or interest. The interested reader is highly encouraged to take a watch, should they find the time, of the final archived lecture.

\subsection{\textit{Venimus, Vidimus, Vicimus} (mostly) -- and Now We Are Done}
This has been quite a journey, and from where we stand now, it may feel as if we spent much of it wading about in the shallow end before a jam-packed and speedy tour through the deep end that actually led us directly into proving the \pw theorem proper. However, for the young graduate student with little experience in the area, this author, at least, can say definitively that as much was to be gained in those slower, early sections of Part I as was in the whirlwind proofs of Part II. Overall, we hope these notes are useful, elucidative, interesting, or at the very least, searchable for any who may find themselves in want or need of a write-up of this first of three modules in the lectures series on \omy and the \pw theorem. Thank you for bearing with us, and deepest apologies for what is left out in what might otherwise have been a very interesting part IV: Applications.
  
  