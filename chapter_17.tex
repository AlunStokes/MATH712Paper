%%%%%%%%%%%%%%%%%%%%% chapter.tex %%%%%%%%%%%%%%%%%%%%%%%%%%%%%%%%%
%
% sample chapter
%
% Use this file as a template for your own input.
%
%%%%%%%%%%%%%%%%%%%%%%%% Springer-Verlag %%%%%%%%%%%%%%%%%%%%%%%%%%
%\motto{Use the template \emph{chapter.tex} to style the various elements of your chapter content.}
%\chapter{Algebraic Dimensionality: A Second, and  (Perhaps) More Enlightening Approach}
\chapter{A Slew of Results on \SCDs}
\label{chap:smooth_cell} % Always give a unique label
%\chaptermark{Some Introductions}

\abstract*{To wrap up now what may have felt like a very long first section, we state a number of results on \emph{\scds} -- in particular starting with a definition of what makes a \cd smooth in any case. For the beleaguered reader, just champing at the bit for discussion of \pw to start properly: we promise, this is the last bit between you and that carrot that's been dangling in front of you for that past while. For the reader, however, feeling that more detail should be wrung out before moving on, it is with great disappointment that we inform you that you have chosen the wrong text to follow, and that it took this many chapters for you to come to that conclusion. Either way, make room somewhere in your mind for a few more definitions, \lemmas, propositions, and theorems, and we will quickly be on our way to a not \emph{too} involved proof of the theorem it feels we set out to prove so long ago.}

\abstract{To wrap up now what may have felt like a very long first section, we state a number of results on \emph{\scds} -- in particular starting with a definition of what makes a \cd smooth in any case. For the beleaguered reader, just champing at the bit for discussion of \pw to start properly: we promise, this is the last bit between you and that carrot that's been dangling in front of you for that past while. For the reader, however, feeling that more detail should be wrung out before moving on, it is with great disappointment that we inform you that you have chosen the wrong text to follow, and that it took this many chapters for you to come to that conclusion. Either way, make room somewhere in your mind for a few more definitions, \lemmas, propositions, and theorems, and we will quickly be on our way to a not \emph{too} involved proof of the theorem it feels we set out to prove so long ago.}

\medskip

This chapter in particular really is more a survery of useful results and ideas than the preceding chapters, with much less attention paid to proving the claimed statements either at all, or with much degree of rigour. For those interested in such detail, they are encouraged to reference (as one can for most that we have done and will do here) \emph{Tame topology and \Om structures} \cite{dries_tame_1998}.

\section{Smoothness}
Recall our setting; we are in an \om expansion of an ordered field, $\M = (M, <, +, \cdot, 0, 1, \hdots)$, and so much of what you'll recall from your classes in calculus and analysis apply -- for example, the notion of differentiability of a univariate function on a point in the interior of its domain can be well-defined. As a result, we use this ordered-field structure to define continuous differentiability in almost exactly this way.

\begin{svgraybox}
  Note that we hand-wave quite a lot of the calculus-centric content here, as is it (reasonably) presumed to be relatively obvious how one would extend or make any arguments presented either more rigorous or work in higher dimensions than we discuss. We hope, for those ardent fans of the basics of calculus, that this is not too much a of a disappointment.
\end{svgraybox}


\begin{theorem}
  Suppose $\funcdom{f}{(a, b)}{M}$ is \defnb, and $r \in \Zgeq{1}$. Then $f$ is $C^{r}$ except at no more than finitely-many points.
  \label{thm:cont_diff}
\end{theorem}

\begin{proof}[or rather, a bit of reasoning about the idea]
delete me
\end{proof}


