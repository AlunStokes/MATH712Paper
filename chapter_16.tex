%%%%%%%%%%%%%%%%%%%%% chapter.tex %%%%%%%%%%%%%%%%%%%%%%%%%%%%%%%%%
%
% sample chapter
%
% Use this file as a template for your own input.
%
%%%%%%%%%%%%%%%%%%%%%%%% Springer-Verlag %%%%%%%%%%%%%%%%%%%%%%%%%%
%\motto{Use the template \emph{chapter.tex} to style the various elements of your chapter content.}
%\chapter{Algebraic Dimensionality: A Second, and  (Perhaps) More Enlightening Approach}
\chapter{Algebraic Dimension: One More Time Around}
\label{chap:alg_dimensionality} % Always give a unique label
% use \chaptermark{}
% to alter or adjust the chapter heading in the running head
%\chaptermark{Some Introductions}

%\abstract*{As so often one finds in mathematics, dimension is one of those ideas that it almost seems each individual mathematician has a notion of how it should be defined -- and more truthfully, it has hugely varying meanings across the vast spectra of mathematical disciplines. And perhaps even more, unfortunately, it so often seems the case that the average mathematician was raised on a dictionary of no more than 30 or so words -- which they go on to use and reuse and smash together into all-new words, for the most part equally inequivalently to how their office-mate might be doing the same thing in their preferred area of abstractness. To say the same in so many fewer words: we often find that the same words mean different things to different mathematicians -- and here, we are going to study two interpretations of one such word -- that being \emph{dimension}. We look at both the algebraist's and logician's definition (at least in the context of \omy) of the dimension of a structure and then happily go on to find out that they actually intersect. This goes on to be fundamental, and this fact is part of what allows us to connect the seemingly non-number theoretic ideas examined hence, to those that allow us to go on to prove a theorem about points of bounded height on curves; a very number theoretic idea indeed.}

\abstract*{What we just worked through was broadly a very natural extension of the ideas we'd been toying with so far, and the idea of defining dimensionality by cells is a very sensible one indeed. However, the antsy and number theoretically-inclined amongst you may be getting a little weary of just how little of that sort has cropped up so far. Fear not then, our brethren and sistren in the fine and noble study of numbers and the little things they do -- as we now examine the idea of dimensionality from an \emph{algebraic} perspective, and begin to work in the ideas that will come to unite the logical world of \omy with that of our own (not to expose our biases too blatantly). We will come to the wonderful conclusion that, in fact, these two notions are one and the same, and we can start to drop the veil of pretense we have been vaguely shrouding this whole affair in -- although you would be in the know had you read the introduction in its entirety, or even just the Table of Contents. Importantly, this idea will lead us to one of the important tools that allows the \pwt to work at all; that being this connection between algebra, \nt and \omy that together can put together something bigger and more wonderful than each constituent on its own. So, dear reader, please enjoy a rehashing of much of what you saw just a chapter ago, but now through an entirely different lens -- one that will perhaps expose you to some ideas and inner workings you'd not thought of or seen previously.}

%\abstract{As so often one finds in mathematics, dimension is one of those ideas that it almost seems each individual mathematician has a notion of how it should be defined -- and more truthfully, has hugely varying meanings across the vast spectra of mathematical disciplines. And perhaps even more, unfortunately, it so often seems the case that the average mathematician was raised on a dictionary of no more than 30 or so words -- which they go on to use and reuse and smash together into all-new words, for the most part equally inequivalently to how their office-mate might be doing the same thing in their preferred area of abstractness. To say the same in so many fewer words: we often find that the same words mean different things to different mathematicians -- and here, we are going to study two interpretations of one such word -- that being \emph{dimension}. We look at both the algebraist's and logician's definition (at least in the context of \omy) of the dimension of a structure and then happily go on to find out that they actually intersect. This goes on to be fundamental, and this fact is part of what allows us to connect the seemingly non-number theoretic ideas examined hence, to those that allow us to go on to prove a theorem about points of bounded height on curves; a very number theoretic idea indeed.}

\abstract{What we just worked through was broadly a very natural extension of the ideas we'd been toying with so far, and the idea of defining dimensionality by cells is a very sensible one indeed. However, the antsy and number theoretically-inclined amongst you may be getting a little weary of just how little of that sort has cropped up so far. Fear not then, our brethren and sistren in the fine and noble study of numbers and the little things they do -- as we now examine the idea of dimensionality from an \emph{algebraic} perspective, and begin to work in the ideas that will come to unite the logical world of \omy with that of our own (not to expose our biases too blatantly). We will come to the wonderful conclusion that, in fact, these two notions are one and the same, and we can start to drop the veil of pretense we have been vaguely shrouding this whole affair in -- although you would be in the know had you read the introduction in its entirety, or even just the Table of Contents. Importantly, this idea will lead us to one of the important tools that allows the \pwt to work at all; that being this connection between algebra, \nt and \omy that together can put together something bigger and more wonderful than each constituent on its own. So, dear reader, please enjoy a rehashing of much of what you saw just a chapter ago, but now through an entirely different lens -- one that will perhaps expose you to some ideas and inner workings you'd not thought of or seen previously.}


\begin{svgraybox}
  As in Chapters \ref{chap:connectedness} and now \ref{chap:defn_dimensionality}, we will continue in our assumption that we have $\M$ an \om expansion of an \emph{ordered} field, $(M, <, +, \cdot, 0, 1)$, and prove things about this construction in generality.
\end{svgraybox}

Although at no point will we be diving deeply into the depths of model theory and all the fun/horror that this may entail (depending on who you ask), this is going to be the part of the course that relies most heavily on \emph{some} knowledge of model theory, and there's little to do to get around that besides having little heuristics in one's head that make things \emph{kinda make sense} enough to not worry about it too much. All this said, the reader who's not even heard the term `model theory' in their life should not be intimidated, as we will be doing our best in holding you hand (or not as you see fit) through the dicier parts of this section, and so much as we hate to keep using this phraseology, belabouring that which may be very obvious to one of a different background. While this may not have been mentioned before, as a series of papers (or more likely document on a computer of some variety) it is not only easy, but vastly personally rewarding to jump over sections you feel eminently comfortable about with a reckless abandon. Well, this is true for all but I hope the one professor whom I hope does get to at least the vast majority of words written here. But for anyone else -- skip judiciously, and skip as if you've much better things to do (which you likely do).

\section{So How Does One Define Dimension?}
\label{sec:alg_dim}

\noindent Tempting though it is to answer ``depends on who you ask'' and move on with our day, the question does bear significant thought. We start with the following definition (and promise we are not trying to induce amnesia):

\begin{definition}[Algebraic Dimension]
  \
  Unlike in the previous chapter, we will \emph{not} be starting with a definition of algebraic dimension, and working from there -- you simply are not yet prepared, given the content discussed so far. We now back up a little bit, and will return to this topic when more sufficiently prepared.
  \label{defn:alg_dim_fake}
\end{definition}

\section{On Closures of Various Types}
We hope that the mention of `closure' brings no worry to your mind, as we have for several chapters now been using the \emph{standard} idea of a closure of a set in our definitions, propositions, \lemmas, proofs, and whatever such delineations one may come up with. As we are sure you are aware, however, `closure' is one of those terms that mathematicians have a difficult time letting go of once they've sunk they're teeth in, and it has come to mean different things to several different branches of mathematical discipline. Perhaps the most well-known (to any undergraduate having taken an analysis class) is that of the \emph{metric closure} -- that is, taking a set, and adding all its limit-points with respect to a given metric. Think $\R$ from $\Q$ or local $p$-adic equivalents. Were you to ask a student fresh out of modern algebra, they would likely tell you of \emph{algebraic closure}, wherein one adds all the roots of univariate-polynomials with coefficients in a given field. Think $\overline{\Q}$ from $\Q$ -- notice that this is a \emph{different} closure than one gets from $\Q$ under any $p$-norm.

While this is likely all well-known to a reader of your calibre, we believe it bears mentioning before we intoduce the two ideas we are about to be playing about with for the next little bit: those of \emph{definable closure} ($\operatorname{dcl}$) and \emph{algebraic closure} ($\operatorname{acl}$).
