%%%%%%%%%%%%%%%%%%%%% chapter.tex %%%%%%%%%%%%%%%%%%%%%%%%%%%%%%%%%
%
% sample chapter
%
% Use this file as a template for your own input.
%
%%%%%%%%%%%%%%%%%%%%%%%% Springer-Verlag %%%%%%%%%%%%%%%%%%%%%%%%%%
%\motto{Use the template \emph{chapter.tex} to style the various elements of your chapter content.}
\chapter{Dimensionality}
\label{chap:dimensionality} % Always give a unique label
% use \chaptermark{}
% to alter or adjust the chapter heading in the running head
%\chaptermark{Some Introductions}

\abstract*{As so often one finds in mathematics, dimension is one of those ideas that it almost seems each individual mathematician has a notion of how it should be defined -- and more truthfully, it has hugely varying meanings across the vast spectra of mathematical disciplines. And perhaps even more, unfortunately, it so often seems the case that the average mathematician was raised on a dictionary of no more than 30 or so words -- which they go on to use and reuse and smash together into all-new words, for the most part equally inequivalently to how their office-mate might be doing the same thing in their preferred area of abstractness. To say the same in so many fewer words: we often find that the same words mean different things to different mathematicians -- and here, we are going to study two interpretations of one such word -- that being \emph{dimension}. We look at both the algebraist's and logician's definition (at least in the context of \omy) of the dimension of a structure and then happily go on to find out that they actually intersect. This goes on to be fundamental, and this fact is part of what allows us to connect the seemingly non-number-theoretic ideas examined hence, to those that allow us to go on to prove a theorem about points of bounded height on curves; a very number-theoretic idea indeed.}

\abstract{As so often one finds in mathematics, dimension is one of those ideas that it almost seems each individual mathematician has a notion of how it should be defined -- and more truthfully, has hugely varying meanings across the vast spectra of mathematical disciplines. And perhaps even more, unfortunately, it so often seems the case that the average mathematician was raised on a dictionary of no more than 30 or so words -- which they go on to use and reuse and smash together into all-new words, for the most part equally inequivalently to how their office-mate might be doing the same thing in their preferred area of abstractness. To say the same in so many fewer words: we often find that the same words mean different things to different mathematicians -- and here, we are going to study two interpretations of one such word -- that being \emph{dimension}. We look at both the algebraist's and logician's definition (at least in the context of \omy) of the dimension of a structure and then happily go on to find out that they actually intersect. This goes on to be fundamental, and this fact is part of what allows us to connect the seemingly non-number-theoretic ideas examined hence, to those that allow us to go on to prove a theorem about points of bounded height on curves; a very number-theoretic idea indeed.}


\section{Dimension: So How Does One Define That?}
\label{sec:dim-defn}

\noindent Tempting though it is to answer ``depends on who you ask'' and move on with our day, the question does bear significant thought.
